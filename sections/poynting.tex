\section{Poynting's theorem}
%
We consider \(\mathrm{M}_\mathrm{III}\) and \(\mathrm{M}_\mathrm{IV}\) and take
their scalar product with, respectively, \(\vbfi\) and \(\vefi\).
\[\vbfi \cdot \of{\crl{\vefi}} = - \frac{\vbfi}{\gammaF} \cdot \pderiv{\vbfi}{t}
\qquad
\vefi \cdot \of{\crl{\vbfi}} = \alphaA \, \vefi \cdot \vec{j} + \frac{\vefi}{\gammaM} \cdot \pderiv{\vefi}{t}\]
We then subtract the second equation from the first.
\[\vefi \cdot \of{\crl{\vbfi}} - \vbfi \cdot \of{\crl{\vefi}} = \alphaA \, \vefi \cdot \vec{j} + \frac{\vefi}{\gammaM} \cdot \pderiv{\vefi}{t} + \frac{\vbfi}{\gammaF} \cdot \pderiv{\vbfi}{t}\]
The left hand side of this equation is the divergence of \(\vefi \times \vbfi\).
\[- \dvg{\of{\vefi \times \vbfi}} = \alphaA \, \vefi \cdot \vec{j} + \frac{\vefi}{\gammaM} \cdot \pderiv{\vefi}{t} + \frac{\vbfi}{\gammaF} \cdot \pderiv{\vbfi}{t}\]
We now isolate the term containing \(\vefi \cdot \vec{j}\).
\begin{align*}
- \vefi \cdot \vec{j} &= \dvg{\of{\frac{\vefi \times \vbfi}{\alphaA}}} +
\frac{\vefi}{\alphaA \, \gammaM} \cdot \pderiv{\vefi}{t} + \frac{\vbfi}{\alphaA \, \gammaF} \cdot \pderiv{\vbfi}{t}\\
- \vefi \cdot \vec{j} &= \dvg{\of{\frac{\vefi \times \vbfi}{\alphaA}}} +
\frac{\vefi}{\alphaG \, \kappac} \cdot \pderiv{\vefi}{t} + \frac{\vbfi}{\alphaA \, \kappaf} \cdot \pderiv{\vbfi}{t}
\end{align*}
We then use the identity for the derivative of a product to rewrite the second and
third terms in the right hand side.
\begin{align*}
- \vefi \cdot \vec{j} &= \dvg{\of{\frac{\vefi \times \vbfi}{\alphaA}}} +
\frac{1}{2} \pderiv{}{t} \of{\frac{\vefi \cdot \vefi}{\kappac\alphaG} + \frac{\vbfi \cdot \vbfi}{\kappaf\alphaA}}\\
- \vefi \cdot \vec{j} &= \dvg{\of{\frac{\vefi \times \vbfi}{\alphaA}}} +
\frac{1}{2\kappac} \pderiv{}{t} \of{\frac{\vefi \cdot \vefi}{\alphaG} + \frac{\kappac}{\kappaf}\frac{\vbfi \cdot \vbfi}{\alphaA}}\\
- \kappac \vefi \cdot \vec{j} &= \dvg{\of{\frac{\kappac}{\alphaA}\vefi \times \vbfi}} +
\frac{1}{2} \pderiv{}{t} \of{\frac{\vefi \cdot \vefi}{\alphaG} + \frac{\kappac}{\kappaf}\frac{\vbfi \cdot \vbfi}{\alphaA}}
\end{align*}
The left hand side is the power density dissipated due to the Joule--Lenz effect;
this suggests that the term in the derivative is an \textbf{energy density},
and the vector of which we are taking the divergence is its flux, called the
\textbf{Poynting vector} \(\vpoy\).
\begin{equation}
\vpoy = \frac{\kappac}{\alphaA} \vefi \times \vbfi \qquad
u = \frac{1}{2} \of{\frac{\vefi \cdot \vefi}{\alphaG} + \frac{\kappac}{\kappaf}\frac{\vbfi \cdot \vbfi}{\alphaA}}
\end{equation}
In these terms, Poynting's theorem states the following.
\begin{equation}
\D{u}{t} = \pderiv{u}{t} + \dvg{\vpoy}
\end{equation}
%
\subsection{Maxwell's stress tensor}
%
We start with the expression of the Lorentz force density and replace the charge
and current densities with expressions resulting from Maxwell's equations.
\[\vec{f} = \rho \vefi + \frac{\kappac}{\kappaf} \vec{j} \times \vbfi = \frac{\dvg{\vefi}}{\alphaG}\vefi + \frac{\kappac}{\kappaf} \of{\frac{\crl\vbfi}{\alphaA} \times \vbfi
  - \frac{1}{\alphaA \gammaM} \pderiv{\vefi}{t} \times \vbfi}\]
We can then replace the last term with its expression in terms of \(\vefi \times \vbfi\).
\[\vec{f} = \frac{\dvg{\vefi}}{\alphaG}\vefi + \frac{\kappac}{\kappaf} \of{\frac{\crl\vbfi}{\alphaA} \times \vbfi
  - \frac{1}{\alphaA \gammaM} \pderiv{}{t} \of{\vefi \times \vbfi} + \frac{1}{\alphaA\gammaM} \vefi \times \pderiv{\vbfi}{t}}\]
Then we can use Faraday--Maxwell's equation to replace the derivative of \(\vbfi\).
\[\vec{f} = \frac{\dvg{\vefi}}{\alphaG}\vefi + \frac{\kappac}{\kappaf} \of{\frac{\crl\vbfi}{\alphaA} \times \vbfi
  - \frac{1}{\alphaA \gammaM} \pderiv{}{t} \of{\vefi \times \vbfi} - \frac{\gammaF}{\alphaA\gammaM} \vefi \times \of{\crl{\vefi}}}\]
\[\vec{f} = \frac{\dvg{\vefi}}{\alphaG}\vefi + \frac{\kappac}{\kappaf} \of{- \frac{\vbfi \times \of{\crl{\vbfi}}}{\alphaA}
  - \frac{1}{\gammaM} \pderiv{}{t} \of{\frac{\vefi \times \vbfi}{\alphaA}} - \frac{\kappaf}{\kappac} \frac{1}{\alphaG} \vefi \times \of{\crl{\vefi}}}\]
\[\vec{f} + \frac{\kappac}{\gammaM \kappaf} \pderiv{}{t} \of{\frac{\vefi \times \vbfi}{\alphaA}} = \frac{\vefi \of{\dvg{\vefi}} - \vefi \times \of{\crl{\vefi}}}{\alphaG}
  - \frac{\kappac}{\kappaf} \frac{\vbfi \times \of{\crl{\vbfi}}}{\alphaA}\]
We notice that a term is missing to achieve symmetry, this term would contain the divergence
of \(\vbfi\), which is zero, so it can be freely added to the expression.
\[\vec{f} + \frac{\kappac}{c^2} \pderiv{}{t} \of{\frac{\vefi \times \vbfi}{\alphaA}} = \frac{\vefi \of{\dvg{\vefi}} - \vefi \times \of{\crl{\vefi}}}{\alphaG}
  + \frac{\kappac}{\kappaf} \frac{\vbfi \of{\dvg{\vbfi}} - \vbfi \times \of{\crl{\vbfi}}}{\alphaA}\]
We recognise the Poynting vector and use \(\vec{a} \times \of{\crl{\vec{a}}} = \frac{1}{2} \grd{\of{\vec{a}\cdot\vec{a}}} - \of{\vec{a} \cdot \grd{}}\vec{a}\).
\[\vec{f} + \frac{1}{c^2} \pderiv{\vpoy}{t} =
  \frac{\vefi \of{\dvg{\vefi}} + \of{\vefi \cdot \grd{}} \vefi - \frac{1}{2} \grd{\of{\vefi\cdot\vefi}}}{\alphaG}
  + \frac{\kappac}{\kappaf} \frac{\vbfi \of{\dvg{\vbfi}} + \of{\vbfi \cdot \grd{}} \vbfi - \frac{1}{2} \grd{\of{\vbfi\cdot\vbfi}}}{\alphaA}\]
We use the identity \(\dvg{\of{\vec{a} \otimes \vec{a}}} = \vec{a} \of{\dvg{\vec{a}}} + \of{\vec{a} \cdot \grd{}}\vec{a}\).
\[\vec{f} + \frac{1}{c^2} \pderiv{\vpoy}{t} =
  \frac{\dvg{\of{\vefi\otimes\vefi}} - \frac{1}{2} \grd{\of{\vefi\cdot\vefi}}}{\alphaG}
  + \frac{\kappac}{\kappaf} \frac{\dvg{\of{\vbfi\otimes\vbfi}} - \frac{1}{2} \grd{\of{\vbfi\cdot\vbfi}}}{\alphaA}\]
By recalling that we can write the gradient of a scalar function as the divergence
of the same function multiplied by the identity tensor \(\vec{1}\), we can write the right hand
side of this expression as the divergence of a tensor called the \textbf{Maxwell stress tensor}.
\[\vmxw = \of{\frac{\vefi\otimes\vefi}{\alphaG} + \frac{\kappac}{\kappaf} \frac{\vbfi\otimes\vbfi}{\alphaA}}
  - \frac{\vec{1}}{2} \of{\frac{\vefi \cdot \vefi}{\alphaG} + \frac{\kappac}{\kappaf}\frac{\vbfi \cdot \vbfi}{\alphaA}}\]
We obtain the following expression, which expresses conservation of momentum (recall
that force is the time derivative of momentum).
\[\D{\vec{p}}{t} + \frac{1}{c^2} \pderiv{\vpoy}{t} = \dvg{\vmxw}\]
Looking at the expression for \(\vmxw\), we notice that it can be written as a tensor
minus half its trace. Moreover, the trace part is exactly the energy density for
the electromagnetic field which we found in Poynting's theorem, for this reason
we denote this tensor by \(\vec{u}\) and define it by including the \(\frac{1}{2}\)
factor, so the trace is exactly the energy density.
\[\vec{u} = \frac{1}{2} \of{\frac{\vefi\otimes\vefi}{\alphaG} + \frac{\kappac}{\kappaf} \frac{\vbfi\otimes\vbfi}{\alphaA}}
\qquad \trace{\vec{u}} = u\]
In terms of this new tensor, the Maxwell stress tensor can be written as follows,
making it evident that its trace is, once again, the energy density.
\[\vmxw = 2 \, \vec{u} - \of{\trace{\vec{u}}} \vec{1} \qquad \trace{\vmxw} = u\]
%
%
