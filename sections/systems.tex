\section{Systems}
%
A system of units is said to be \textbf{rationalised} when no factors of \(4 \pi\)
appear in Maxwell's equations.
This is useful because the absence of such factors implies that equations relating
to specific charge distributions will end up having factors related to the symmetry
of the system, such as \(4 \pi\) for spherical symmetry or \(2 \pi\) for cylindrical
symmetry.
\subsection{The International System -- SI}
The International System of units makes the following choice for the couplings.
\[\alphaG = \frac{1}{\epsilon_0} \qquad \alphaA = \mu_0 \qquad \kappac = 1 \qquad \kappaf = 1 \qquad \gammaM = c^2 \qquad \gammaF = 1\]
Here \(\epsilon_0\) and \(\mu_0\) are constants determined by experiment; the current (2020) CODATA recommended values are
\[\epsilon_0 = \SI{8.8541878128(13)e-12}{\farad\per\meter} \qquad \mu_0 = \SI{1.25663706212(19)e-6}{\henry\per\meter}\]
The SI is now a truly rationalised system. Before its redefinition in 2019 it used to
be an unrationalised system in disguise, because \(\mu_0\) was defined as
exactly \(4\pi \cdot \SI{e-7}{\henry\per\meter}\).
We can make this explicit by defining a scaling constant
\(a = \SI{e7}{\ampere\squared\per\newton}\).
\[\alphaG = \frac{4 \pi c^2}{a} \qquad \alphaA = \frac{4 \pi}{a} \qquad \kappac = 1 \qquad \kappaf = 1 \qquad \gammaM = c^2 \qquad \gammaF = 1\]
%
\subsection{Electrostatic units -- ESU}
Electrostatic units are defined by taking Coulomb's constant to be unity.
\[\alphaG = 4 \pi \qquad \alphaA = \frac{4\pi}{c^2} \qquad \kappac = 1 \qquad \kappaf = 1 \qquad \gammaM = c^2 \qquad \gammaF = 1\]
This system is explicitly unrationalised.
%
\subsection{Electromagnetic units -- EMU}
Electromagnetic units are defined in order to make Ampère's force constant unity.
The name electromagnetic is not very apt, since Ampère's force law is only valid
in the magnetostatic limit.
\[\alphaG = 4 \pi c^2 \qquad \alphaA = 4\pi \qquad \kappac = 1 \qquad \kappaf = 1 \qquad \gammaM = c^2 \qquad \gammaF = 1\]
This system is explicitly unrationalised, but notice how it is the same as the old SI
units, except for the lack of the scaling factor \(a\).
The reason for this is that the SI adopted a rescaled version of EMU that were in
practical use among engineers.
%
\subsection{Gaussian units -- GU}
Gaussian units are defined as follows.
\[\alphaG = 4 \pi \qquad \alphaA = \frac{4\pi}{c} \qquad \kappac = 1 \qquad \kappaf = c \qquad \gammaM = c \qquad \gammaF = c\]
This system is explicitly unrationalised.
%
\subsection{Lorentz-Heaviside units -- LHU}
Lorentz-Heaviside units are obtained by rationalising Gaussian units.
\[\alphaG = 1 \qquad \alphaA = \frac{1}{c} \qquad \kappac = 1 \qquad \kappaf = c \qquad \gammaM = c \qquad \gammaF = c\]
Obviously, this system is rationalised.
%
\subsection{Symetrised Gaussian units -- SGU}
Gaussian units can be made fully symmetrical, with \(\alphaG = \alphaA = 4 \pi\),
by taking \(\kappac = c\). This also has the advantage of making current
density and charge density have the same units, like the electric and magnetic
field in the standard Gaussian system.
\[\alphaG = 4 \pi \qquad \alphaA = 4\pi \qquad \kappac = c \qquad \kappaf = c \qquad \gammaM = c \qquad \gammaF = c\]
This system is explicitly unrationalised.
%
\subsection{Symmetrised Lorentz-Heaviside units -- SLHU}
Lorentz-Heaviside units can also be made fully symmetrical, with \(\alphaG = \alphaA = 1\).
\[\alphaG = 1 \qquad \alphaA = 1 \qquad \kappac = c \qquad \kappaf = c \qquad \gammaM = c \qquad \gammaF = c\]
This system is rationalised.
%
%
