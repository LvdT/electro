\section{Maxwell's equations for macroscopic fields}
%
Maxwell's equations depend upon all of the charges and currents present in a physical
system; if we study the behaviour of the electric and magnetic field given by
Maxwell's equation in some medium, we have contributions coming from the nuclei
and electrons of each atom in the medium.
We obtain two fields \(\vefi_\mathrm{loc}\) and \(\vbfi_\mathrm{loc}\) are called the
\textbf{microscopic} or \textbf{local fields} because they explicitly depend upon
these microscopic phenomena, which are generally not easy to treat.\\[1em]
We can simplify the problem by averaging the local fields over a volume \(\Delta \Omega\),
that must be chosen so that the average is statistically significant, but at the
same time it must be small enough to still allow us to treat the system as
a continuum. Doing so we obtain the \textbf{macroscopic fields} \(\vefi\) and \(\vbfi\).
\[\vefi\of{\vec{r},t} = \frac{1}{\Delta \Omega} \iiint_{\Delta \Omega} \vefi_\mathrm{loc}\of{\vec{r}+\vec{\xi},t}\de^3\xi\]
\[\vbfi\of{\vec{r},t} = \frac{1}{\Delta \Omega} \iiint_{\Delta \Omega} \vbfi_\mathrm{loc}\of{\vec{r}+\vec{\xi},t}\de^3\xi\]
Despite this approximation, several contributions still find their way into the
source terms in Maxwell's equations.
The total charge and current densities can be split into a free and a bound term.
\[\rho = \rho_\mathrm{f} + \rho_\mathrm{b} \qquad \vec{j} = \vec{j}_\mathrm{f} + \vec{j}_\mathrm{b}\]
Bound charges or currents are phenomena that generally happen as a result of
the interaction of a physical system with the electric and magnetic field, such
as polarisation or magnetisation.
The former produces both a charge and a current density, while the latter only
produces a current density.
\[\rho_\mathrm{b} = \rho_\mathrm{p} \qquad \vec{j}_\mathrm{b} = \vec{j}_\mathrm{p} + \vec{j}_\mathrm{m}\]
The polarisation terms can be represented by a \textbf{polarisation field} \(\vpfi\).
\[\rho_\mathrm{p} = - \dvg{\vpfi} \qquad \vec{j}_\mathrm{p} = \frac{1}{\kappac} \pderiv{\vpfi}{t}\]
The magnetisation term can be represented by a \textbf{magnetisation field} \(\vmfi\).
\[\vec{j}_\mathrm{m} = \frac{\kappaf}{\kappac} \crl{\vmfi}\]
The polarisation and magnetisation fields are defined in such a way that the
corresponding densities each respect a continuity equation.
In the case of the magnetisation current, no charge density is necessary, since the
divergence of a curl is zero.\\[1em]
We rewrite Maxwell's equations, explicitly separating the free and bound terms.
\begin{center}
  \begin{tabular}{cccl}
    \(\mathrm{M}_\mathrm{I}\) & \(\dvg{\vefi}\) & \(=\) & \(\alphaG \of{\rho_\mathrm{f} + \rho_\mathrm{p}}\) \\[1em]
    \(\mathrm{M}_\mathrm{II}\) & \(\dvg{\vbfi}\) & \(=\) & \(0\) \\[1em]
    \(\mathrm{M}_\mathrm{III}\) & \(\crl{\vefi}\) & \(=\) & \(\displaystyle - \frac{1}{\gammaF} \pderiv{\vbfi}{t}\) \\[1em]
    \(\mathrm{M}_\mathrm{IV}\) & \(\crl{\vbfi}\) & \(=\) & \(\displaystyle \alphaA \of{\vec{j}_\mathrm{f} + \vec{j}_\mathrm{p} + \vec{j}_\mathrm{m}} + \frac{1}{\gammaM} \pderiv{\vefi}{t}\) \\
  \end{tabular}
\end{center}
We focus on Gauss's law for electricity.
\begin{gather*}
  \dvg{\vefi} = \alphaG \of{\rho_\mathrm{f} + \rho_\mathrm{p}} \\
  \dvg{\vefi} - \alphaG \, \rho_\mathrm{p} = \alphaG \, \rho_\mathrm{f} \\
  \dvg{\vefi} + \alphaG \, \dvg{\vpfi} = \alphaG \, \rho_\mathrm{f} \\
  \dvg{\of{\vefi + \alphaG \, \vpfi}} = \alphaG \, \rho_\mathrm{f}
\end{gather*}
We have two choices on how to define the \textbf{electric displacement field} \(\vdfi\).
\begin{equation}
  \vdfi_1 = \vefi + \alphaG \, \vpfi \qquad \vdfi_2 = \frac{\vefi}{\alphaG} + \vpfi
\end{equation}
They produce, respectively, the following equations.
\begin{equation}
  \dvg{\vdfi_1} = \alphaG \, \rho_\mathrm{f} \qquad \dvg{\vdfi_2} = \rho_\mathrm{f}
\end{equation}
We now take Ampère--Maxwell's law.
\begin{gather*}
  \crl{\vbfi} = \alphaA \of{\vec{j}_\mathrm{f} + \vec{j}_\mathrm{p} + \vec{j}_\mathrm{m}} + \frac{1}{\gammaM} \pderiv{\vefi}{t} \\
  \crl{\vbfi} - \alphaA \vec{j}_\mathrm{m} = \alphaA \vec{j}_\mathrm{f} + \frac{1}{\gammaM} \pderiv{\vefi}{t} + \alphaA \vec{j}_\mathrm{p} \\
  \crl{\vbfi} - \frac{\kappaf}{\kappac} \alphaA \crl{\vmfi} = \alphaA \vec{j}_\mathrm{f} + \frac{1}{\gammaM} \pderiv{\vefi}{t} + \frac{\alphaA}{\kappac} \pderiv{\vpfi}{t} \\
  \crl{\of{\vbfi - \frac{\kappaf}{\kappac} \alphaA \vmfi}} = \alphaA \vec{j}_\mathrm{f} + \frac{1}{\gammaM} \pderiv{\vefi}{t} + \frac{\alphaG}{\gammaM} \pderiv{\vpfi}{t} \\
  \crl{\of{\vbfi - \frac{\kappaf}{\kappac} \alphaA \vmfi}} = \alphaA \vec{j}_\mathrm{f} + \frac{1}{\gammaM} \pderiv{}{t} \of{\vefi + \alphaG \pderiv{\vpfi}{t}}
\end{gather*}
We have two choices on how to define the \textbf{magnetising field} \(\vhfi\).
\begin{equation}
  \vhfi_1 = \vbfi - \frac{\kappaf}{\kappac} \alphaA \vmfi \qquad \vhfi_2 = \frac{\kappac}{\kappaf} \frac{\vbfi}{\alphaA} - \vmfi
\end{equation}
They produce, respectively, the following equations.
\begin{equation}
  \crl{\vhfi_1} = \alphaA \, \vec{j}_\mathrm{f} + \frac{1}{\gammaM} \pderiv{\vdfi_1}{t} \qquad \crl{\vhfi_2} = \frac{\kappac}{\kappaf} \, \vec{j}_\mathrm{f} + \frac{1}{\kappaf} \pderiv{\vdfi_2}{t}
\end{equation}
%
%
\subsection{Auxiliary fields}
%
We have defined two \textbf{auxiliary fields} to aid in our description of
electromagnetic phenomena in matter, these are \(\vdfi\) and \(\vhfi\).
We can take two different approaches in the way we define these fields, one is
aimed at maintaining the appearance of Maxwell's equations substantially unchanged,
while the other is aimed at getting rid of the coupling constants with the sources
by including them in the definition of the auxiliary field.
For the sake of brevity, we define the following constants.
\[\lambda_\mathrm{e} = \alphaG \qquad \lambda_\mathrm{m} = \alphaA \frac{\kappaf}{\kappac}\]
We call the fields in the first approach \textbf{non-rescaled auxiliary fields}
because in a vacuum (absence of polarisation/magnetisation) we have \(\vdfi_1 = \vefi\)
and \(\vhfi_1 = \vbfi\).
\[\vdfi_1 = \vefi + \lambda_\mathrm{e} \, \vpfi \qquad \vhfi_1 = \vbfi - \lambda_\mathrm{m} \, \vmfi\]
In contrast to this, we call the fields in the first approach \textbf{rescaled auxiliary fields}
because in a vacuum we have \(\vdfi_2 = \frac{1}{\lambda_\mathrm{e}} \vefi\)
and \(\vhfi_2 = \frac{1}{\lambda_\mathrm{m}} \vbfi\).
\[\vdfi_2 = \frac{1}{\lambda_\mathrm{e}} \vefi + \vpfi \qquad \vhfi_2 = \frac{1}{\lambda_\mathrm{m}} \vbfi - \vmfi\]
Both approaches are equally legitimate, and each provides a set of Maxwell's
equations in which only the free charges and currents appear.
\begin{center}
  \begin{tabular}{c|ccl|ccl}
    \toprule
    & \multicolumn{3}{c|}{\textbf{Non-rescaled}} & \multicolumn{3}{c}{\textbf{Rescaled}} \\
    \midrule
    \(\mathrm{M}_\mathrm{I}\) & \(\dvg{\vdfi_1}\) & \(=\) & \(\alphaG \rho_\mathrm{f}\) & \(\dvg{\vdfi_2}\) & \(=\) & \(\rho_\mathrm{f}\) \\[1em]
    \(\mathrm{M}_\mathrm{II}\) & \(\dvg{\vbfi}\) & \(=\) & \(0\) & \(\dvg{\vbfi}\) & \(=\) & \(0\) \\[1em]
    \(\mathrm{M}_\mathrm{III}\) & \(\crl{\vefi}\) & \(=\) & \(\displaystyle - \frac{1}{\gammaF} \pderiv{\vbfi}{t}\) & \(\crl{\vefi}\) & \(=\) & \(\displaystyle - \frac{1}{\kappaf} \pderiv{\vbfi}{t}\) \\[1em]
    \(\mathrm{M}_\mathrm{IV}\) & \(\crl{\vhfi_1}\) & \(=\) & \(\displaystyle \alphaA \vec{j}_\mathrm{f} + \frac{1}{\gammaM} \pderiv{\vdfi_1}{t}\) & \(\crl{\vhfi_2}\) & \(=\) & \(\displaystyle \frac{\kappac}{\kappaf} \, \vec{j}_\mathrm{f} + \frac{1}{\kappaf} \pderiv{\vdfi_2}{t}\) \\
    \bottomrule
  \end{tabular}
\end{center}
Equations \(\mathrm{M}_\mathrm{II}\) and \(\mathrm{M}_\mathrm{III}\) do not change,
however in the rescaling approach we have written \(\kappaf\) instead
of \(\gammaF\) to make it apparent that the equations do not depend upon
physical coupling constants, but only upon the constants used in the definition of the
magnetic field and of current (\(\kappac\) and \(\kappaf\)).\\[1em]
It can be convenient to summarise the two approaches into a single expression
involving constants \(\kappa_\mathrm{e}\) and \(\kappa_\mathrm{m}\) as follows.
\[\vdfi = \frac{1}{\kappa_\mathrm{e}} \of{\vefi + \lambda_\mathrm{e} \vpfi} \qquad \vhfi = \frac{1}{\kappa_\mathrm{m}} \of{\vbfi - \lambda_\mathrm{m} \vmfi}\]
Choosing \(\kappa_\mathrm{e} = \kappa_\mathrm{m} = 1\) gives us the non-rescaled
fields, while we get the rescaled fields with \(\kappa_\mathrm{e} = \lambda_\mathrm{e}\)
and \(\kappa_\mathrm{m} = \lambda_\mathrm{m}\).
We once again remark that, while the definition of \(\lambda_\mathrm{e}\) and
\(\lambda_\mathrm{m}\) is simply a matter of convenience, and they depend solely
upon previously defined constants, the choice of \(\kappa_\mathrm{e}\) and \(\kappa_\mathrm{m}\)
is entirely independent of our previous conventions.\\[1em]
With this new convention Maxwell's equations take the following form.
\begin{center}
  \begin{tabular}{cccl}
    \(\mathrm{M}_\mathrm{I}\) & \(\dvg{\vdfi}\) & \(=\) & \(\dfrac{\alphaG}{\kappa_\mathrm{e}} \rho_\mathrm{f}\) \\[1em]
    \(\mathrm{M}_\mathrm{II}\) & \(\dvg{\vbfi}\) & \(=\) & \(0\) \\[1em]
    \(\mathrm{M}_\mathrm{III}\) & \(\crl{\vefi}\) & \(=\) & \(\displaystyle - \frac{1}{\gammaF} \pderiv{\vbfi}{t}\) \\[1em]
    \(\mathrm{M}_\mathrm{IV}\) & \(\crl{\vhfi}\) & \(=\) & \(\displaystyle \frac{\alphaA}{\kappa_\mathrm{m}} \vec{j}_\mathrm{f} + \frac{\kappa_\mathrm{e}}{\kappa_\mathrm{m}} \frac{1}{\gammaM} \pderiv{\vdfi}{t}\) \\
  \end{tabular}
\end{center}
%
%
