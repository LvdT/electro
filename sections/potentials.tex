\section{Potentials}
%
From \(\mathrm{M}_\mathrm{II}\) we can say that the magnetic field is solenoidal,
which means that it can be written as the curl of a \textbf{vector potential} \(\vafi\).
\[\vbfi = \crl{\vafi}\]
We substitute this expression for \(\vbfi\) into \(\mathrm{M}_\mathrm{III}\).
\[\crl{\vefi} = - \frac{1}{\gammaF} \pderiv{}{t} \crl{\vafi} \qquad \crl{\of{\vefi + \frac{1}{\kappaf} \pderiv{\vafi}{t}}} = 0\]
This shows that the combination of the electric field with the time derivative of
the vector potential is irrotational, meaning that it admits a \textbf{scalar potential} \(\scp\).
\[\vefi = - \grd{\scp} - \frac{1}{\kappaf} \pderiv{\vafi}{t}\]
Notice how the fields depend only upon the derivatives of the potentials, due to
this if we are given a pair of potentials \(\scp\) and \(\vafi\) we can define
a new pair \(\scp'\) and \(\vafi'\) producing the same fields. The easiest example
of this would be simply adding a constant to each potential, but the most general
case involves adding a field \(\Lambda\of{\vec{r},t}\) as follows.
\[\vafi'\of{\vec{r},t} = \vafi\of{\vec{r},t} + \grd\Lambda\of{\vec{r},t}
\qquad \scp'\of{\vec{r},t} = \scp\of{\vec{r},t} + \frac{1}{\kappaf} \pderiv{\Lambda}{t}\]
It is important to remark that the transformation must be performed on both fields
at the same time. This phenomenon is known as \textbf{gauge invariance}.
The ambiguity in the definition of the fields can be solved, or at least assuaged,
by requiring an additional condition that the potentials must satisfy, this is
called \textbf{gauge fixing}.\\[1em]
The most used gauge choices in physics are the \textbf{Coulomb gauge}
and the \textbf{Lorenz gauge}.
\[\text{Coulomb:  } \dvg{\vafi} = 0 \qquad\qquad \text{Lorenz:  } \frac{\kappaf}{c^2} \pderiv{\scp}{t} + \dvg{\vafi} = 0\]
Substituting the fields expressed in terms of the potential into Maxwell's equations
gives us equations for the potentials themselves, two of the equations, namely
\(\mathrm{M}_\mathrm{II}\) and \(\mathrm{M}_\mathrm{III}\), are already satisfied by
virtue of how we defined the potentials. We start by considering \(\mathrm{M}_\mathrm{I}\),
from which we obtain the following expression.
\[\dvg{\vefi} = \alphaG \, \rho \qquad - \lp \phi - \frac{1}{\kappaf} \pderiv{}{t} \dvg{\vafi} = \alphaG \, \rho\]
We now consider \(\mathrm{M}_\mathrm{IV}\), which requires some further manipulation.
\[\crl{\vbfi} = \alphaA \, \vec{j} + \frac{1}{\gammaM} \pderiv{\vefi}{t}
\qquad \crl{\of{\crl{\vafi}}} = \alphaA \, \vec{j} - \frac{1}{\gammaM} \pderiv{}{t} \grd{\scp} - \frac{1}{\gammaM\kappaf} \pderiv[2]{\vafi}{t}\]
We recall that \(\crl{\of{\crl{\vec{a}}}} = \grd{\of{\dvg{\vec{a}}}} - \lp{\vec{a}}\)
and that \(\gammaM\kappaf = \gammaM\gammaF = c^2\).
\[\grd{\of{\dvg{\vafi}}} - \lp{\vafi} = \alphaA \, \vec{j} - \frac{1}{\gammaM} \pderiv{}{t} \grd{\scp} - \frac{1}{c^2} \pderiv[2]{\vafi}{t}\]
\[\frac{1}{c^2} \pderiv[2]{\vafi}{t} - \lp{\vafi} + \grd{\of{\dvg{\vafi} + \frac{\kappaf}{c^2} \pderiv{\scp}{t} }} = \alphaA \, \vec{j}\]
We thus have a system of coupled differential equations, which is not trivial to
solve. With the correct gauge choice, though, the equations become simpler.
%
%
\subsection{Static potentials}
%
\(\mathrm{M}_\mathrm{I}\)
seems to suggest the Coulomb gauge, producing Poisson's equation.
\[\lp{\scp} = - \alphaG \, \rho\]
In this gauge \(\mathrm{M}_\mathrm{IV}\) is not particularly simple. We obtain
d'Alembert's equation with a source term depending on \(\scp\) (which, however,
we can find indepedently of \(\vafi\) from the previous equation).
\[\dal{\vafi} = \alphaA \, \vec{j} - \frac{\kappaf}{c^2} \pderiv{}{t} \grd{\scp}\]
Things become a lot simpler in the static case, that is when the system is independent
of time and therefore all time derivatives vanish. We obtain Poisson's equation.
\[\lp{\vafi} = - \alphaA \, \vec{j}\]
Using the Green function for the Laplacian operator we find the \textbf{static potentials}.
\[
  \scp\of{\vec{r}} = \frac{\alphaG}{4 \pi} \iiint_\Omega \frac{\rho\of{\vec{\xi}}}{\nrm{\vec{r}-\vec{\xi}}} \de^3\xi
  \qquad
  \vafi\of{\vec{r}} = \frac{\alphaA}{4 \pi} \iiint_\Omega \frac{\vec{j}\of{\vec{\xi}}}{\nrm{\vec{r}-\vec{\xi}}} \de^3\xi
\]
%
%
\subsection{Retarded potentials}
%
A different way to simplify \(\mathrm{M}_\mathrm{IV}\) is to use the Lorenz gauge,
so that the whole term within the gradient vanishes. Doing so we obtain d'Alembert's equation.
\[\dal{\vafi} = \alphaA \, \vec{j}\]
In this gauge the two equations are fully decoupled, indeed we obtain d'Alembert's
equation for the scalar potential as well.
\[\dal{\scp} = \alphaG \, \rho\]
Using the Green function method we find the \textbf{retarded potentials}.
\[
  \scp\of{\vec{r},t} = \frac{\alphaG}{4 \pi} \iiint_\Omega \frac{\rho\of{\vec{\xi},t_\mathrm{r}}}{\nrm{\vec{r}-\vec{\xi}}} \de^3\xi
  \qquad
  \vafi\of{\vec{r},t} = \frac{\alphaA}{4 \pi} \iiint_\Omega \frac{\vec{j}\of{\vec{\xi},t_\mathrm{r}}}{\nrm{\vec{r}-\vec{\xi}}} \de^3\xi
\]
The \textbf{retarded time} \(t_\mathrm{r}\) used in the expression accounts exactly for the
time it takes for light to travel from the point \(\vec{r}\) to the point \(\vec{\xi}\)
and is defined as follows.
\[t_\mathrm{r} = t - \frac{c}{\nrm{\vec{r}-\vec{\xi}}}\]
So the potentials in the Lorenz gauge properly satisfy causality.
Another advantage of the Lorenz gauge is that it can be written in a manifestly
covariant manner, making it very useful for relativistic purposes.
%
%
