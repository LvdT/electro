\section{Conduction}
%
Another interesting response function is the one linking current density to the
electric field, called \textbf{conductivity} \(\sigma\).
\begin{equation}
\vec{j} = \sigma \vefi
\end{equation}
Its reciprocal is called \textbf{resistivity} and is unfortunately denoted with
the letter \(\rho\), risking confusion with the charge density.
Resistivity is widely used in engineering applications, but in solid state physics
it is more common to work with conductivity, possibly because the electric field
causing the current density is the more direct implication.
%
%
\subsection{Ohm’s Laws}
%
We can consider a simple system consisting of a conductor of length \(\ell\),
cross section \(a\) and uniform conductivity \(\sigma\) to which a uniform electric
field \(\vefi\of{t}\) is applied.
The current density through the conductor will have the following expression,
where \(f\of{\vec{x}}\) is a function that is unity within the conductor an zero outside.
\[\vec{j}\of{\vec{x},t} = \sigma f\of{\vec{x}}\vefi\of{t} \]
We can compute the current from the definition of current density, for brevity we
stop writing the time dependence explicitly.
\[i = \iint_{\Sigma} \vec{j}\of{\vec{\xi}} \cdot \vec{n} \, \de^2\xi
= \sigma \vefi \cdot \vec{n} \iint_\Sigma f\of{\vec{\xi}} \, \de^2\xi\]
From the way we defined \(f\), its integral is the area of the cross section.
\[i = \sigma a \vefi \cdot \vec{n}\]
We integrate again along the path \(\gamma\) of the conductor.
\[\int_\gamma i \de\lambda = \sigma a \int_\gamma \vefi \cdot \vec{n} \, \de\lambda\]
The normal to the cross section \(\vec{n}\) is the tangent to the path of the conductor,
allowing us to recognise the expression for the electromotive force.
\[i \ell = \sigma a \emf\]
This leads us to Ohm’s laws expressed in terms of the \textbf{conductance} \(G\).
\[i = G \emf \qquad G = \sigma \frac{a}{\ell}\]
%
%
\subsection{Drude model}
%
A very simple model for conduction involves considering a gas of charged particles
with mass \(m\), charge \(q\) and number density \(n\) subject to scattering with
a relaxation time of \(\tau\).
The motion of these particles is described by the following differential equation.
\[\D{\vec{p}}{t} = q \vefi - \frac{1}{\tau} \vec{p}\]
This equation can be easily solved in Fourier space.
\[- \icmp \omega \vec{p} = q \vefi - \frac{1}{\tau} \vec{p}
\qquad
\vec{p} = \frac{q \tau}{1 - \icmp \omega \tau} \vefi\]
We now recall equation \eqref{eq::current-density-moving} and observe that charge
density is simply \(nq\).
\[\vec{j} = \frac{\rho}{\kappac} \vec{v} = \frac{nq}{\kappac} \frac{\vec{p}}{m}
= \frac{n q^2 \tau}{\kappac m} \frac{1}{1 - \icmp \omega \tau} \vefi\]
We denote the steady state (\(\omega \to 0\)) conductivity as \(\sigma_0\),
obtaining the following expression for conductivity as a function of angular frequency.
\begin{equation}
\sigma(\omega) = \frac{\sigma_0}{1 - \icmp \omega \tau} \qquad \sigma_0 = \frac{n q^2 \tau}{\kappac m}
\end{equation}
