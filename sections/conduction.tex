\section{Conduction}
%
Another interesting response function is the one linking current density to the
electric field, called \textbf{conductivity} \(\sigma\).
\begin{equation}
\vec{j} = \sigma \vefi
\end{equation}
Conduction is due to the motion of charged particles called \textbf{(charge) carriers}.
In the presence of multiple types of charge carriers we can express the total
conductivity as a sum of several contributions.
\[\sigma = \sum_n \sigma_n\]
A different view of this phenomenon can be given by studying the \textbf{mobility}
\(\mu\) (a symbol unfortunately shared with permeability), which links the
\emph{average} or \emph{drift} velocity to the electric field instead.
\[\vec{v} = \mu \vefi\]
The main difference in this approach is that, in general, each carrier will have
a different mobility within the same material.
In other words, while conductivity is a property of the material, mobility is a
property of a particle inside that material.\\[1em]
If we consider a single type of carrier, with charge \(q\) and number density \(n\of{\vec{r},t}\),
we can recall equation \eqref{eq::current-density-moving} and observe that
\(\rho\of{\vec{r},t} = q n\of{\vec{r},t}\) in order to write the current density
in terms of velocity.
\[\vec{j} = \frac{\rho}{\kappac} \vec{v} = \frac{nq}{\kappac} \vec{v}\]
This gives an expression linking mobility and conductivity for a specific type of carrier.
\[\sigma = \frac{nq}{\kappac} \mu\]
The reciprocal of conductivity is called the \textbf{resistivity} \(\rho\) (this
time causing confusion with the symbol used for charge density).
%
%
\subsection{Conductivity in Maxwell’s equations}
%
In writing Maxwell’s equations for the auxiliary fields, we have left \emph{free}
charge and current densities as source terms.
These can, however, be further split into a conduction and an external term.
\[\rho_\mathrm{f} = \rho_\mathrm{c} + \rho_\mathrm{e} \qquad \vec{j}_\mathrm{f} = \vec{j}_\mathrm{c} + \vec{j}_\mathrm{e}\]
The former is induced by the presence of the electric field, therefore the current
density must depend upon the electric field through the conductivity.
\[\vec{j}_\mathrm{c} = \sigma \vefi\]
The latter represents instead the currents and charges not belonging to the material.\\[1em]
We shall work on the Fourier-transformed version of Maxwell’s equations, taking
\(\mathrm{M}_\mathrm{I}\).
\[\icmp \vec{k} \cdot \vdfi = \frac{\alphaG}{\kappae} \rho_\mathrm{c} + \frac{\alphaG}{\kappae} \rho_\mathrm{e}\]
We can express \(\rho_\mathrm{c}\) in terms of \(\vec{j}_\mathrm{c}\), and therefore
\(\vefi\) by using the Fourier-transformed version of the continuity equation.
\[\vec{k} \cdot \vec{j}_\mathrm{c} - \frac{\omega}{\kappac} \rho_\mathrm{c} = 0
\qquad \rho_\mathrm{c} = \frac{\kappac}{\omega} \vec{k} \cdot \vec{j}_\mathrm{c}
= \frac{\kappac}{\omega} \vec{k} \cdot \of{\sigma \vefi}\]
We substitute into \(\mathrm{M}_\mathrm{I}\), along with \(\vdfi = \epsilon \vefi\).
\[\icmp \vec{k} \cdot \of{\epsilon \vefi} = \frac{\kappac}{\kappae} \frac{\alphaG}{\omega} \vec{k} \cdot \of{\sigma \vefi} + \frac{\alphaG}{\kappae} \rho_\mathrm{e} \]
\[\icmp \vec{k} \cdot \of{\epsilon \vefi + \icmp \frac{\kappac}{\kappae} \frac{\alphaG}{\omega} \sigma \vefi} = \frac{\alphaG}{\kappae} \rho_\mathrm{e}\]
This suggests that we define a \textbf{complex permittivity} (or austerity, going
with our terminology) \(\ecmp\) that combines both terms in the left hand into one.
\[\ecmp\of{\vec{k},\omega} = \epsilon\of{\vec{k},\omega} + \icmp \frac{\kappac}{\kappae} \frac{\alphaG}{\omega} \sigma\of{\vec{k},\omega}
\qquad \icmp \vec{k} \cdot \of{\ecmp \vefi} = \frac{\alphaG}{\kappae} \rho_\mathrm{e}\]
In this way the only remaining source term is the externally applied charge density.
The same substitution is possible in \(\mathrm{M}_\mathrm{IV}\) as well.
\begin{align*}
\icmp \vec{k} \times \vhfi &= \frac{\alphaA}{\kappam} \vec{j}_\mathrm{c} + \frac{\alphaA}{\kappam} \vec{j}_\mathrm{e} - \frac{\kappae}{\kappam} \frac{\icmp \omega}{\gammaM} \vdfi\\
\icmp \vec{k} \times \vhfi &= \frac{\alphaA}{\kappam} \sigma \vefi + \frac{\alphaA}{\kappam} \vec{j}_\mathrm{e} - \frac{\kappae}{\kappam} \frac{\icmp \omega}{\gammaM} \epsilon \vefi\\
\icmp \vec{k} \times \vhfi &= \frac{\alphaA}{\kappam} \vec{j}_\mathrm{e} - \frac{\kappae}{\kappam} \frac{\icmp \omega}{\gammaM} \of{\epsilon \vefi + \icmp \frac{\gammaM}{\kappae} \frac{\alphaA}{\omega} \sigma \vefi}
\end{align*}
Observing that \(\gammaM \alphaA = \kappac \alphaG\) we arrive at the same expression as before.
\begin{align*}
\icmp \vec{k} \times \vhfi &= \frac{\alphaA}{\kappam} \vec{j}_\mathrm{e} - \frac{\kappae}{\kappam} \frac{\icmp \omega}{\gammaM} \of{\epsilon \vefi + \icmp \frac{\kappac}{\kappae} \frac{\alphaG}{\omega} \sigma \vefi}\\
\icmp \vec{k} \times \vhfi &= \frac{\alphaA}{\kappam} \vec{j}_\mathrm{e} - \frac{\kappae}{\kappam} \frac{\icmp \omega}{\gammaM} \ecmp \vefi
\end{align*}
By defining a new electric displacement field \(\vddfi\of{\vec{k},\omega} = \ecmp\of{\vec{k},\omega} \vefi\of{\vec{k},\omega}\)
it becomes possible to write a set Maxwell’s equations where the only source terms
are the externally applied charge and current densities.
\begin{center}
  \begin{tabular}{cccl}
    \(\mathrm{M}_\mathrm{I}\) & \(\dvg{\vddfi}\) & \(=\) & \(\dfrac{\alphaG}{\kappae} \rho_\mathrm{e}\) \\[1em]
    \(\mathrm{M}_\mathrm{II}\) & \(\dvg{\vbfi}\) & \(=\) & \(0\) \\[1em]
    \(\mathrm{M}_\mathrm{III}\) & \(\crl{\vefi}\) & \(=\) & \(\displaystyle - \frac{1}{\gammaF} \pderiv{\vbfi}{t}\) \\[1em]
    \(\mathrm{M}_\mathrm{IV}\) & \(\crl{\vhfi}\) & \(=\) & \(\displaystyle \frac{\alphaA}{\kappam} \vec{j}_\mathrm{e} + \frac{\kappae}{\kappam} \frac{1}{\gammaM} \pderiv{\vddfi}{t}\) \\
  \end{tabular}
\end{center}
It is very important to keep in mind that in \(\vddfi\of{\vec{r},t}\) is not just
the product, but rather the convolution of \(\ecmp\of{\vec{r},t}\) and \(\vefi\of{\vec{r},t}\).
%
%
\subsection{Circuit Laws}
%
We consider a system consisting of a conductor with uniform conductivity \(\sigma\)
described by a curve \(\gamma\) parametised by \(\lambda\).
The curve has length \(\ell = \int_\gamma \de\lambda\).
The conductor has a uniform cross section which can be represented by a function
\(f\of{\vec{\xi}}\), where \(\vec{\xi}\) is the radial coordinate in the normal plane,
which is unity within the conductor an zero outside.
The area of the cross section is \(a = \iint_{\mathbb{R}^2} f\of{\vec{\xi}} \de^2 \xi\).\\[1em]
An electric field \(\vefi\of{\lambda,t}\) is applied to the curve, we consider the
approximation in which the thickness of the conductor is small enough that we can
disregard changes in the field along the radial coordinate, which we have represented
by a dependence on \(\lambda\) alone.
The current density through the conductor will have the following expression.
\[\vec{j}\of{\vec{r},t} = \sigma f\of{\vec{\xi}}\vefi\of{\lambda,t} \]
We can compute the current at a point \(\lambda\) along the curve by integrating
on a surface \(\Sigma\) in the normal plane that is wide enough to include all points
where \(f\of{\vec{\xi}} \neq 0\).
We observe that the normal vector to such a surface is simply the tangent vector
to the curve and therefore we denote it as \(\vec{t}(\lambda)\).
\begin{align*}
i\of{\lambda,t} &= \iint_{\Sigma} \vec{j}\of{\vec{r},t} \cdot \vec{t}\of{\lambda} \, \de^2 \xi
= \sigma \, \vefi\of{\lambda,t} \cdot \vec{t}\of{\lambda} \iint_\Sigma f\of{\vec{\xi}} \, \de^2 \xi \\
&= \sigma \, a \, \vefi\of{\lambda,t} \cdot \vec{t}\of{\lambda}
\end{align*}
We now compute the average current \(i\of{t}\) across the conductor.
\[i\of{t} = \frac{1}{\ell} \int_\gamma i\of{\lambda,t} \de\lambda = \sigma \frac{a}{\ell} \int_\gamma \vefi\of{\lambda,t} \cdot \vec{t}\of{\lambda} \de\lambda\]
This allows us to recognise the expression for the electromotive force (technically
only for the induced electromotive force, but given the absence of a magnetic
field the motional electromotive force would be zero).
\[i\of{t} = \sigma \frac{a}{\ell} \emf\of{t}\]
This leads us to \textbf{Ohm’s law} and \textbf{Pouillet’s law} in terms of the \textbf{conductance} \(G\).
\[i = G \emf \qquad G = \sigma \frac{a}{\ell}\]
The laws are more commonly written using the reciprocal quantity, the
\textbf{resistance} \(R\).
\[\emf = R i \qquad R = \rho \frac{\ell}{a}\]
%
%
\subsection{Drude model}
%
A very simple model for conduction involves considering a gas of charged particles
with mass \(m\), charge \(q\) and number density \(n\) subject to scattering with
a relaxation time of \(\tau\).
The motion of these particles is described by the following differential equation,
where \(\vec{p}\) is the average momentum of a particle.
\[\D{\vec{p}}{t} = q \vefi - \frac{1}{\tau} \vec{p}\]
This equation can be easily solved in Fourier space.
\[- \icmp \omega \vec{p} = q \vefi - \frac{1}{\tau} \vec{p}
\qquad
\vec{p} = \frac{q \tau}{1 - \icmp \omega \tau} \vefi\]
Bringing this in terms of velocity allows us to find the mobility of the carriers.
\[\vec{v} = \frac{q\tau}{m} \frac{1}{1 - \icmp \omega \tau} \vefi\]
We denote the steady state (\(\omega \to 0\)) mobility as \(\mu_0\),
obtaining the following expression for mobility as a function of angular frequency.
\[\mu(\omega) = \frac{\mu_0}{1 - \icmp \omega \tau} \qquad \mu_0 = \frac{q \tau}{m}\]
This, in turn, allows us to write the conductivity of our gas.
\begin{equation}
\sigma(\omega) = \frac{\sigma_0}{1 - \icmp \omega \tau} \qquad \sigma_0 = \frac{n \tau q^2}{\kappac m}
\end{equation}
