\section{Conduction}
%
Another interesting response function is the one linking current density to the
electric field, called \textbf{conductivity} \(\sigma\).
\begin{equation}
\vec{j} = \sigma \vefi
\end{equation}
Its reciprocal is called \textbf{resistivity} and is unfortunately denoted with
the letter \(\rho\), risking confusion with the charge density.
Resistivity is widely used in engineering applications, but in solid state physics
it is more common to work with conductivity, possibly because the electric field
causing the current density is the more direct implication.
%
\newpage
%
\subsection{Ohm’s Laws}
%
We consider a system consisting of a conductor with uniform conductivity \(\sigma\)
described by a curve \(\gamma\) parametised by \(\lambda\).
The curve has length \(\ell = \int_\gamma \de\lambda\).
The conductor has a uniform cross section which can be represented by a function
\(f\of{\vec{r}}\), where \(\vec{r}\) is the radial coordinate in the normal plane,
which is unity within the conductor an zero outside.
The area of the cross section is \(a = \iint_{\mathbb{R}^2} f\of{\vec{r}} \de^2 r\).\\[1em]
An electric field \(\vefi\of{\lambda,t}\) is applied to the curve, we consider the
approximation in which the thickness of the conductor is small enough that we can
disregard changes in the field along the radial coordinate, which we have represented
by a dependence on \(\lambda\) alone.
The current density through the conductor will have the following expression.
\[\vec{j}\of{\vec{x},t} = \sigma f\of{\vec{r}}\vefi\of{\lambda,t} \]
We can compute the current at a point \(\lambda\) along the curve by integrating
on a surface \(\Sigma\) in the normal plane that is wide enough to include all points
where \(f\of{\vec{r}} \neq 0\).
We observe that the normal vector to such a surface is simply the tangent vector
to the curve and therefore we denote it as \(\vec{t}(\lambda)\).
\begin{align*}
i\of{\lambda,t} &= \iint_{\Sigma} \vec{j}\of{\vec{x}} \cdot \vec{t}\of{\lambda} \, \de^2 r
= \sigma \vefi\of{\lambda,t} \cdot \vec{t}\of{\lambda} \iint_\Sigma f\of{\vec{r}} \, \de^2 r \\
&= \sigma a \vefi\of{\lambda,t} \cdot \vec{t}\of{\lambda}
\end{align*}
We now compute the average current \(i\of{t}\) across the conductor.
\[i\of{t} = \frac{1}{\ell} \int_\gamma i\of{\lambda,t} \de\lambda = \sigma \frac{a}{\ell} \int_\gamma \vefi\of{\lambda,t} \cdot \vec{t}\of{\lambda} \de\lambda\]
This allows us to recognise the expression for the electromotive force (technically
only for the induced electromotive force, but given the absence of a magnetic
field the motional electromotive force would be zero).
\[i\of{t} = \sigma \frac{a}{\ell} \emf\of{t}\]
This leads us to \textbf{Ohm’s laws} expressed in terms of the \textbf{conductance} \(G\).
\[i = G \emf \qquad G = \sigma \frac{a}{\ell}\]
The laws are more commonly written using the reciprocal quantity, the
\textbf{resistance} \(R\).
\[\emf = R i \qquad R = \rho \frac{\ell}{a}\]
%
%
\subsection{Drude model}
%
A very simple model for conduction involves considering a gas of charged particles
with mass \(m\), charge \(q\) and number density \(n\) subject to scattering with
a relaxation time of \(\tau\).
The motion of these particles is described by the following differential equation.
\[\D{\vec{p}}{t} = q \vefi - \frac{1}{\tau} \vec{p}\]
This equation can be easily solved in Fourier space.
\[- \icmp \omega \vec{p} = q \vefi - \frac{1}{\tau} \vec{p}
\qquad
\vec{p} = \frac{q \tau}{1 - \icmp \omega \tau} \vefi\]
We now recall equation \eqref{eq::current-density-moving} and observe that charge
density is simply \(nq\).
\[\vec{j} = \frac{\rho}{\kappac} \vec{v} = \frac{nq}{\kappac} \frac{\vec{p}}{m}
= \frac{n q^2 \tau}{\kappac m} \frac{1}{1 - \icmp \omega \tau} \vefi\]
We denote the steady state (\(\omega \to 0\)) conductivity as \(\sigma_0\),
obtaining the following expression for conductivity as a function of angular frequency.
\begin{equation}
\sigma(\omega) = \frac{\sigma_0}{1 - \icmp \omega \tau} \qquad \sigma_0 = \frac{n q^2 \tau}{\kappac m}
\end{equation}
