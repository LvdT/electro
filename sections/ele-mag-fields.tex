\section{Electric and magnetic fields}
%
The force acting on a small \emph{test charge} \(q\) as a result of electromagnetic
phenomena can be split into two contributions, an electric force which is independent
of the charge's motion and a magnetic term which also depends on its velocity.
\[\vec{F}\of{\vec{r},t} = \vec{F}_\mathrm{e}\of{\vec{r},t} + \vec{F}_\mathrm{m}\of{\vec{v},\vec{r},t}\]
Experimentally, both terms are observed to be linear in \(q\).\\[1em]
The \textbf{electric field} \(\vefi\) is defined in terms of the electric force
and is obtained by factoring out charge, since we have just observed that
the force depends linearly upon it.
\[\vec{F}_\mathrm{e}\of{\vec{r},t} = q \, \vefi\of{\vec{r},t}\]
The \textbf{magnetic field} \(\vbfi\) behaves in a slightly more complicated manner.
The magnetic force is observed to always be perpendicular to the particle's velocity, which
suggests that a cross product is involved in the definition.
Moreover, we need to introduce a scaling constant \(\kappaf\), which will
be necessary to give us the correct value for the speed of light as an electromagnetic
wave.
\[\vec{F}_\mathrm{m}\of{\vec{v},\vec{r},t} = \frac{q\vec{v} \times \vbfi\of{\vec{r},t}}{\kappaf}\]
The sum of these two contributions is called the \textbf{Lorentz force}.
\begin{equation}
  \vec{F}\of{\vec{r},t} = q \of{\vefi\of{\vec{r},t} + \frac{\vec{v} \times \vbfi\of{\vec{r},t}}{\kappaf}}
\end{equation}
The Lorentz force can also be expressed in terms of charge and current densities,
in order to do so it is convenient to define the \textbf{force density} \(\vec{f}\),
a local property defined by the amount of force per unit volume.
\[\vec{F}(t) = \iiint_\Omega \vec{f}\of{\vec{\xi},t} \de^3 \xi\]
We start with the electric force, we can turn the expression above into a volume
integral by using the definition of the Dirac delta.
\[\vec{F}_\mathrm{e}\of{\vec{r},t} = \iiint_\Omega q \, \delta\of{\vec{r}-\vec{\xi}} \vefi\of{\vec{\xi},t}\de^3 \xi\]
The expression \(q \, \delta\of{\vec{r}-\vec{\xi}}\) is the charge density of a
point charge \(q\) located at \(\vec{r}\), this leads us to an expression for the
electric force density.
\[\vec{F}_\mathrm{e}\of{t} = \iiint_\Omega \rho\of{\vec{\xi},t} \vefi\of{\vec{\xi},t} \de^3 \xi
\qquad \vec{f}_\mathrm{e}\of{\vec{r},t} = \rho\of{\vec{r},t} \vefi\of{\vec{r},t}\]
We can proceed in a similar manner with the magnetic force, we immediately identify
the expression for the charge density.
\[\vec{F}_\mathrm{m}\of{t} = \iiint_\Omega \frac{\rho\of{\vec{\xi},t}\vec{v} \times \vbfi\of{\vec{\xi},t}}{\kappaf} \de^3 \xi\]
We now refer to equation \eqref{eq::current-density-moving} and replace \(\rho\of{\vec{\xi},t}\vec{v}\)
with \(\kappac \, \vec{j}\of{\vec{\xi},t}\), obtaining an expression for the
magnetic force density.
\[\vec{F}_\mathrm{m}\of{t} = \iiint_\Omega \frac{\kappac}{\kappaf} \vec{j}\of{\vec{\xi},t} \times \vbfi\of{\vec{\xi,t}} \de^3 \xi
\qquad \vec{f}_\mathrm{m}\of{\vec{r},t} = \frac{\kappac}{\kappaf} \, \vec{j}\of{\vec{r},t} \times \vbfi\of{\vec{r},t}\]
%
Taking both contributions together gives us the Lorentz force density.
\[\vec{f}\of{\vec{r},t} = \rho\of{\vec{r},t} \vefi\of{\vec{r},t} + \frac{\kappac}{\kappaf} \, \vec{j}\of{\vec{r},t} \times \vbfi\of{\vec{r},t}\]
%
\subsection{Electromotive force}
%
We consider a closed loop built out of an electrical conductor, delimiting a surface
\(\Sigma\). We define the \textbf{electromotive force} as the work per unit
of charge performed by the electromagnetic force on a charge going around the whole loop,
which is to say, the \emph{circulation} of the Lorentz force per unit charge.
\[\emf\of{t} = \frac{1}{q} \oint_{\partial\Sigma\of{t}} \vec{F}\of{\xi,t} \cdot \vec{t}\of{\xi,t} \de\xi\]
By substituting the expression for the Lorentz force we obtain the following.
\[\emf\of{t} = \oint_{\partial\Sigma\of{t}} \of{\vefi\of{\xi,t} + \frac{\vec{v}\of{\xi,t} \times \vbfi\of{\xi,t}}{\kappaf}} \cdot \vec{t}\of{\xi,t} \de\xi\]
%
The electromotive force is therefore the sum of two contributions, the first is
due to the electric field and is called \textbf{induced electromotive force},
the second is due to motion in a magnetic field and is called \textbf{motional
electromotive force}.
%
\[\emf = \emf_\mathrm{ind} + \emf_\mathrm{mot}\]
\[\emf_\mathrm{ind}\of{t} = \oint_{\partial\Sigma\of{t}} \vefi\of{\xi,t} \cdot \vec{t}\of{\xi,t} \de\xi \qquad
\emf_\mathrm{mot}\of{t} = \oint_{\partial\Sigma\of{t}} \frac{\vec{v}\of{\xi,t} \times \vbfi\of{\xi,t}}{\kappaf} \cdot \vec{t}\of{\xi,t} \de\xi\]
%
Experimentally, it is known that the electromotive force on a circuit depends upon
the time derivative of the magnetic flux through the surface delimited by the circuit;
this relationship is known as the \textbf{Faraday--Neumann--Lenz law}.
\[\emf = - k_\mathrm{F} \D{\Phi_\vbfi}{t}\]
%
\subsection{Joule--Lenz law}
%
Consider a charge distribution \(\rho\of{\vec{r},t}\) together with a velocity
field \(\vec{v}\of{\vec{r},t}\) in a region where both an electric and magnetic fields
exist.
The power density of such a system can be computed as the dot product between the
force density and the velocity field.
\[\dot{u} = \vec{f} \cdot \vec{v} = \rho \vefi \cdot \vec{v} + \frac{\kappac}{\kappaf} \of{\vec{v} \times \vbfi} \cdot \vec{v} = \vefi \cdot \of{\rho \, \vec{v}}\]
The second term vanishes because we're taking the dot product of two perpendicular
vectors. We recognise the expression for the current density.
\[\dot{u} = \kappac \, \vefi \cdot \vec{j} \qquad \frac{1}{\kappac} \D{u}{t} = \vefi \cdot \vec{j}\]
This is known as the \textbf{Joule--Lenz law}.
%
