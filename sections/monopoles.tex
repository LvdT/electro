\section{Magnetic monopoles}
%
Looking at Maxwell's equations, there appears to be a fundamental lack of symmetry
between the electric and magnetic fields.
This lack of symmetry is due to the fact that, as of now, no experiment has ever
observed particles which carry such a thing as \textbf{magnetic charge}, that is
to say some particle which is capable of generating a magnetic field while standing still.
All magnetic phenomena are related to the motion of \emph{electric} charges, whether
this motion is macroscopic (as is the case with current-carrying wires) or microscopic
(as is the case with bar magnets).\\[1em]
Despite the absence of evidence, there have been theoretical arguments for the existence of such particles,
and a correction to Maxwell's equations is in order to account for their presence.
From here on we denote electric charges and currents  with the subscript \(\mathrm{e}\)
and magnetic charges and currents with the subscript \(\mathrm{m}\).\\[1em]
After introducing the concept of magnetic charge \(Q_\mathrm{m}\), the concept of
magnetic current quickly follows. While we could, in principle, take a different
proportionality constant between the current and the time derivative of charge,
this would not be particularly useful, we thus use \(\kappac\) as we
did for electric charge.
\[I_\mathrm{m} = - \frac{1}{\kappac} \pderiv{Q_\mathrm{m}}{t}\]
%
%
\subsection{Continuity equation}
%
Naturally, we can define the corresponding densities.
\[Q_\mathrm{m} = \iiint_\Omega \rho_\mathrm{m}\of{\vec{\xi},t} \de^3\xi
\qquad I_\mathrm{m} = \iint_\Sigma \vec{j}_\mathrm{m}\of{\vec{\xi},t} \cdot \vec{n} \de^2\xi\]
The densities must satisfy a continuity equation, the proof of which is identical
to the one given for the electric case, and shall not be repeated.
\begin{equation}\label{eq::continuity-magnetic}
	\frac{1}{\kappac} \pderiv{\rho_\mathrm{m}}{t} + \dvg{\vec{j}_\mathrm{m}} = 0
\end{equation}
Due to the existence of a continuity equation, we also have the following relation
with the velocity field, once again the proof is identical to that done previously.
\begin{equation}
\vec{j}_\mathrm{m}\of{\vec{r},t} = \frac{\vec{v}\of{\vec{r},t} \, \rho_\mathrm{m}\of{\vec{r},t}}{\kappac}
\end{equation}
%
%
\subsection{Maxwell’s equations}
%
To correct Maxwell's equations we introduce two new coupling constants \(\betaG\)
and \(\betaA\) for Gauss's law and Faraday--Maxwell's law, respectively.
\begin{center}
  \begin{tabular}{ccccl}
    \(\mathrm{M}_\mathrm{I}\) & \textbf{Gauss's law (for \(\vefi\))} & \(\dvg{\vefi}\) & \(=\) & \(\alphaG \, \rho_\mathrm{e}\) \\[1em]
    \(\mathrm{M}_\mathrm{II}\) & \textbf{Gauss's law (for \(\vbfi\))} & \(\dvg{\vbfi}\) & \(=\) & \(\betaG \, \rho_\mathrm{m}\) \\[1em]
    \(\mathrm{M}_\mathrm{III}\) & \textbf{Faraday--Maxwell's law} & \(\crl{\vefi}\) & \(=\) & \(\displaystyle - \betaA \, \vec{j}_\mathrm{m} - \frac{1}{\gammaF} \pderiv{\vbfi}{t}\) \\[1em]
    \(\mathrm{M}_\mathrm{IV}\) & \textbf{Ampère--Maxwell's law} & \(\crl{\vbfi}\) & \(=\) & \(\displaystyle \alphaA \, \vec{j}_\mathrm{e} + \frac{1}{\gammaM} \pderiv{\vefi}{t}\) \\
  \end{tabular}
\end{center}
We can then determine what conditions must apply to \(\beta_\mathrm{G}\) and \(\beta_\mathrm{A}\) in order
for these to be consistent with the continuity equation we just obtained.
We begin by taking the divergence of \(\mathrm{M}_\mathrm{III}\), recalling that
the divergence of a curl must be zero.
\begin{align*}
  \dvg{\of{\crl{\vefi}}} &= - \betaA \dvg{\vec{j}_\mathrm{m}} - \frac{1}{\gammaF} \dvg{\pderiv{\vbfi}{t}} \\
  0 &= - \betaA \dvg{\vec{j}_\mathrm{m}} - \frac{1}{\gammaF} \pderiv{}{t} \dvg{\vbfi} \\
  0 &= - \betaA \dvg{\vec{j}_\mathrm{m}} - \frac{\betaG}{\gammaF} \pderiv{\rho_\mathrm{m}}{t}
\end{align*}
We obtain something that resembles a continuity equation.
\[\frac{\betaG}{\betaA\gammaF} \pderiv{\rho_\mathrm{m}}{t} + \dvg{\vec{j}_\mathrm{m}} = 0\]
Comparing it to equation \eqref{eq::continuity-magnetic} we obtain a relation between the constants.
\[\frac{\betaA\gammaF}{\betaG} = \kappac\]
Having previously determined that \(\gammaF = \kappaf\), we can rearrange this as follows.
\[\betaA = \frac{\kappac}{\kappaf} \betaG\]
Thus in any system where \(\kappac = \kappaf\) it is necessary that also \(\betaA = \betaG\).
%
%

\subsection{Lorentz force}
%
We must now find the correct expression for the Lorentz force of a magnetically
charged particle. To do so we attempt to find an analogue of the Faraday--Neumann--Lenz
law that relates the flux of the electric field to the work done per unit charge.
As before, we consider a closed loop delimiting a surface \(\Sigma\) and we take
the time derivative of the flux of the electric field through this surface.
\[\D{\Phi_\vefi}{t} = \D{}{t} \iint_{\Sigma\of{t}} \vefi\of{\vec{\xi},t} \cdot \vec{n}\of{\vec{\xi},t} \de^2\xi\]
We apply the Reynolds transport theorem in order to compute this derivative.
\[\D{\Phi_\vefi}{t} = \iint_{\Sigma\of{t}} \of{\pderiv{\vefi}{t} + \of{\dvg{\vefi}} \vec{v}} \cdot \vec{n} \, \de^2\xi
- \oint_{\partial\Sigma(t)} \of{\vec{v} \times \vefi} \cdot \vec{t}\,  \de\xi\]
This time the field is not solenoidal, using Gauss's law for the electric field and the Ampère--Maxwell law, we get the
following expression.
\[\D{\Phi_\vefi}{t} = \iint_{\Sigma\of{t}} \of{\gammaM \crl{\vbfi} - \alphaA \gammaM \vec{j}_\mathrm{e} +
\alphaG \, \rho_\mathrm{e} \, \vec{v}} \cdot \vec{n} \, \de^2\xi
- \oint_{\partial\Sigma(t)} \of{\vec{v} \times \vefi} \cdot \vec{t}\,  \de\xi\]
\[\D{\Phi_\vefi}{t} = \iint_{\Sigma\of{t}} \of{- \alphaA \gammaM \vec{j}_\mathrm{e} +
\alphaG \, \rho_\mathrm{e} \, \vec{v}} \cdot \vec{n} \, \de^2\xi
+ \oint_{\partial\Sigma(t)} \of{\gammaM \, \vbfi - \vec{v} \times \vefi} \cdot \vec{t}\,  \de\xi\]
We recall that \(\alphaA \gammaM = \alphaG \kappac\),
giving us the following expression.
\[\D{\Phi_\vefi}{t} = \alphaG \iint_{\Sigma\of{t}} \of{ \rho_\mathrm{e} \, \vec{v} - \kappac \, \vec{j}_\mathrm{e}} \cdot \vec{n} \, \de^2\xi + \gammaM \oint_{\partial\Sigma(t)} \of{ \vbfi - \frac{\vec{v} \times \vefi}{\gammaM}} \cdot \vec{t} \, \de\xi\]
Then we notice that the first integral vanishes, because \(\rho_\mathrm{e} \, \vec{v} = \kappac \, \vec{j}_\mathrm{e}\).
\[\D{\Phi_\vefi}{t} = \gammaM \oint_{\partial\Sigma(t)} \of{ \vbfi - \frac{\vec{v} \times \vefi}{\gammaM}} \cdot \vec{t} \, \de\xi\]
We have managed to find an expression similar to the Faraday--Neumann--Lenz law,
but some care must be taken in naming the right hand side of this equation. We shall
call this \textbf{magnetomotive force} \(\emf_\mathrm{m}\), but we remark that this term is already
in use for a distinct concept in magnetic circuits.
\begin{equation}
  \D{\Phi_\vefi}{t} = \gammaM \, \emf_\mathrm{m}
\end{equation}
We notice there is a difference in sign with the Faraday--Neumann--Lenz law
for the electromotive force, this is in line with the difference in sign between the
Faraday--Maxwell and the Ampère--Maxwell laws. By analogy with the electromotive force,
we can reasonably expect the Lorentz force for a moving magnetically charge particle to have the following expression.
\begin{equation*}
  \vec{F}\of{\vec{r},t} = q_\mathrm{m} \of{\vbfi\of{\vec{r},t} - \frac{\vec{v} \times \vefi\of{\vec{r},t}}{\gammaM}}
\end{equation*}
Naturally, if a particle has both an electric and magnetic charge, then the Lorentz force must
include both contributions.
\begin{equation}
  \vec{F} = q_\mathrm{e} \of{\vefi + \frac{\vec{v} \times \vbfi}{\gammaF}} + q_\mathrm{m} \of{\vbfi - \frac{\vec{v} \times \vefi}{\gammaM}}
\end{equation}
%
%
\subsection{Electromotive force}
%
