\section{Maxwell's equations}
%
While the Lorentz force gives us a way to compute the way a distribution of charges
and currents interacts with the electromagnetic field, we still need a way to
determine how the fields themselves are generated by a distribution of charges.
This is achieved by \textbf{Maxwell's equations}, which can be stated either in
integral or differential form.
For the sake of simplicity, we choose to use the differential form.
\begin{center}
  \begin{tabular}{ccccl}
    \(\mathrm{M}_\mathrm{I}\) & \textbf{Gauss's law (for \(\vefi\))} & \(\dvg{\vefi}\) & \(=\) & \(\alphaG \, \rho\) \\[1em]
    \(\mathrm{M}_\mathrm{II}\) & \textbf{Gauss's law (for \(\vbfi\))} & \(\dvg{\vbfi}\) & \(=\) & \(0\) \\[1em]
    \(\mathrm{M}_\mathrm{III}\) & \textbf{Faraday--Maxwell's law} & \(\crl{\vefi}\) & \(=\) & \(\displaystyle - \frac{1}{\gammaF} \pderiv{\vbfi}{t}\) \\[1em]
    \(\mathrm{M}_\mathrm{IV}\) & \textbf{Ampère--Maxwell's law} & \(\crl{\vbfi}\) & \(=\) & \(\displaystyle \alphaA \, \vec{j} + \frac{1}{\gammaM} \pderiv{\vefi}{t}\) \\
  \end{tabular}
\end{center}
We have used two different types of constants, \(\alphaG\)
and \(\alphaA\) that couple the fields to the charges and currents respectively,
and \(\gammaF\) and \(\gammaM\) that couple
the fields to each other.
%
\subsection{Continuity equation}
%
We can verify that the continuity equation is consistent with Maxwell's equation,
under a proper choice of constants.
We begin by taking the divergence of \(\mathrm{M}_\mathrm{IV}\), recalling that
the divergence of a curl must be zero.
\begin{align*}
  \dvg{\of{\crl{\vbfi}}} &= \alphaA \dvg{\vec{j}} + \frac{1}{\gammaM} \dvg{\pderiv{\vefi}{t}} \\
  0 &= \alphaA \dvg{\vec{j}} + \frac{1}{\gammaM} \pderiv{}{t} \dvg{\vefi} \\
  0 &= \alphaA \dvg{\vec{j}} + \frac{\alphaG}{\gammaM} \pderiv{\rho}{t}
\end{align*}
We obtain something that resembles a continuity equation.
\[\frac{\alphaG}{\alphaA\gammaM} \pderiv{\rho}{t} + \dvg{\vec{j}} = 0\]
Comparing it to equation \eqref{eq::continuity} we obtain a relation between the constants.
\[\frac{\alphaA\gammaM}{\alphaG} = \kappac\]
%
\subsection{Propagation of light}
%
We consider Maxwell's equation in the absence of charges or currents.
\begin{center}
  \begin{tabular}{cccl}
    \(\mathrm{M}_\mathrm{I}\) & \(\dvg{\vefi}\) & \(=\) & \(0\) \\[1em]
    \(\mathrm{M}_\mathrm{II}\) & \(\dvg{\vbfi}\) & \(=\) & \(0\) \\[1em]
    \(\mathrm{M}_\mathrm{III}\) & \(\crl{\vefi}\) & \(=\) & \(\displaystyle - \frac{1}{\gammaF} \pderiv{\vbfi}{t}\) \\[1em]
    \(\mathrm{M}_\mathrm{IV}\) & \(\crl{\vbfi}\) & \(=\) & \(\displaystyle \frac{1}{\gammaM} \pderiv{\vefi}{t}\) \\
  \end{tabular}
\end{center}
We take the curl of \(\mathrm{M}_\mathrm{III}\), use the identity \(\crl{\vec{a}} = \grd{\of{\dvg{\vec{a}}}} - \lp{\vec{a}}\)
and substitute \(\mathrm{M}_\mathrm{I}\) and \(\mathrm{M}_\mathrm{IV}\) where the
corresponding terms appear.
\begin{align*}
  \crl{\of{\crl{\vefi}}} &= - \frac{1}{\gammaF} \pderiv{}{t} \crl{\vbfi} \\
  \grd{\of{\dvg{\vefi}}} - \lp{\vefi} &= - \frac{1}{\gammaF} \pderiv{}{t} \crl{\vbfi} \\
  - \lp{\vefi} &= - \frac{1}{\gammaF\gammaM} \pderiv[2]{\vefi}{t}
\end{align*}
We obtain d'Alembert's equation for the electric field.
\[\frac{1}{\gammaF\gammaM} \pderiv[2]{\vefi}{t} - \lp{\vefi} = 0\]
We know for a fact that the propagation speed of an electromagnetic wave is \(c\);
in terms of our constants, this means that the following condition must hold.
\[\gammaF\gammaM = c^2\]
The process can be repeated by taking the curl of \(\mathrm{M}_\mathrm{IV}\), and
proceeding in a similar manner, but this won't give us any new condition.
The fact that electric and magnetic waves travel at the same speed is intrinsic
in the structure of Maxwell's equations.
\begin{align*}
  \crl{\of{\crl{\vbfi}}} &= \frac{1}{\gammaM} \pderiv{}{t} \crl{\vefi} \\
  \grd{\of{\dvg{\vbfi}}} - \lp{\vbfi} &= \frac{1}{\gammaM} \pderiv{}{t} \crl{\vefi} \\
  - \lp{\vbfi} &= - \frac{1}{\gammaF\gammaM} \pderiv[2]{\vbfi}{t}
\end{align*}
As mentioned, we obtain exactly the same equation we had for the electric field.
\[\frac{1}{\gammaF\gammaM} \pderiv[2]{\vbfi}{t} - \lp{\vbfi} = 0\]
%
%
\subsection{Faraday--Neumann--Lenz law}
%
Let us consider a closed loop built out of an electrical conductor, delimiting a
surface \(\Sigma\) and let us take the time derivative of the flux of the magnetic
field through this surface. The shape and position of the loop can change with time.
\[\D{\Phi_\vbfi}{t} = \D{}{t} \iint_{\Sigma\of{t}} \vbfi\of{\vec{\xi},t} \cdot \vec{n}\of{\vec{\xi},t} \de^2\xi\]
%
The rate of change depends both on the change in magnetic field and in the geometry
of the loop. We can resort to a form of the Reynolds transport theorem in
order to compute this derivative.
\[\D{\Phi_\vbfi}{t} = \iint_{\Sigma\of{t}} \of{\pderiv{\vbfi}{t} + \of{\dvg{\vbfi}}\vec{v}\of{\vec{\xi},t}} \cdot \vec{n}\of{\vec{\xi},t} \de^2\xi
- \oint_{\partial\Sigma(t)} \of{\vec{v}\of{\vec{\xi},t} \times \vbfi\of{\vec{\xi},t}} \cdot \vec{t}\of{\vec{\xi},t} \de\xi\]
We now use Gauss's law for magnetism to remove the term involving the divergence
of the magnetic field, which is zero.
\[\D{\Phi_\vbfi}{t} = \iint_{\Sigma\of{t}} \pderiv{\vbfi}{t} \cdot \vec{n}\of{\vec{\xi},t} \de^2\xi
- \oint_{\partial\Sigma(t)} \of{\vec{v}\of{\vec{\xi},t} \times \vbfi\of{\vec{\xi},t}} \cdot \vec{t}\of{\vec{\xi},t} \de\xi\]
We then use Faraday--Maxwell's law to replace the first term
\begin{align*}
\D{\Phi_\vbfi}{t} &= - \gammaF \iint_{\Sigma\of{t}} \crl{\vefi} \cdot \vec{n}\of{\vec{\xi},t} \de^2\xi
- \oint_{\partial\Sigma(t)} \of{\vec{v}\of{\vec{\xi},t} \times \vbfi\of{\vec{\xi},t}} \cdot \vec{t}\of{\vec{\xi},t} \de\xi \\
\D{\Phi_\vbfi}{t} &= - \gammaF \oint_{\partial\Sigma\of{t}} \vefi\of{\vec{\xi},t} \cdot \vec{t}\of{\vec{\xi},t} \de\xi
- \oint_{\partial\Sigma(t)} \of{\vec{v}\of{\vec{\xi},t} \times \vbfi\of{\vec{\xi},t}} \cdot \vec{t}\of{\vec{\xi},t} \de\xi
\end{align*}
Finally, we recognise the expressions for the induced and motional electromotive
forces.
\[\D{\Phi_\vbfi}{t} = - \gammaF \emf_\mathrm{ind} - \kappaf \, \emf_\mathrm{mot}\]
In order for the Faraday--Neumann--Lenz law to be valid, we need to be able to factor
this expression, so we obtain a third and last condition for our constants, along
with an expression giving us the value of Faraday's constant in terms of such constants.
\[\gammaF = \kappaf = k_\mathrm{F}\]
%
%
\subsection{Adding it all up}
%
We started out with six constants: \(\kappac, \kappaf, \alphaG, \alphaA, \gammaF\) and \(\gammaM\),
but we observed that requiring the consistency of Maxwell's equations with the
continuity equation for electric charges, with the fact that electromagnetic
waves propagate with a speed of \(c\) and with the Faraday--Neumann--Lenz law has
given us three equations that relate these constants to each other.
\begin{equation}
  \alphaG = \frac{c^2}{\kappaf \kappac} \alphaA \qquad \gammaM = \frac{c^2}{\kappaf} \qquad \gammaF = \kappaf
\end{equation}
We are left with only three constants which we can freely choose.
We rearrange the above equations in order to give us a system where we are free
to choose the couplings between the fields and the charges, in addition to the
constant \(\kappac\) that defines current.
\begin{equation}
  \kappaf = \gammaF = \frac{c^2}{\kappac} \frac{\alphaA}{\alphaG}
  \qquad \gammaM = \frac{\alphaG}{\alphaA} \kappac
\end{equation}
In such a system Maxwell's equations take the following form.
\begin{center}
  \begin{tabular}{ccccl}
    \(\mathrm{M}_\mathrm{I}\) & \textbf{Gauss's law (for \(\vefi\))} & \(\dvg{\vefi}\) & \(=\) & \(\alphaG \, \rho\) \\[1em]
    \(\mathrm{M}_\mathrm{II}\) & \textbf{Gauss's law (for \(\vbfi\))} & \(\dvg{\vbfi}\) & \(=\) & \(0\) \\[1em]
    \(\mathrm{M}_\mathrm{III}\) & \textbf{Faraday--Maxwell's law} & \(\crl{\vefi}\) & \(=\) & \(\displaystyle - \frac{1}{\kappaf} \pderiv{\vbfi}{t}\) \\[1em]
    \(\mathrm{M}_\mathrm{IV}\) & \textbf{Ampère--Maxwell's law} & \(\crl{\vbfi}\) & \(=\) & \(\displaystyle \alphaA \, \vec{j} + \frac{\kappaf}{c^2} \pderiv{\vefi}{t}\) \\
  \end{tabular}
\end{center}
%
%
