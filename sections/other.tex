
\section{Other laws}
\textbf{Coulomb's law}
\[\vec{F}_{12} = k_\mathrm{C} \iiint_{\Omega_1} \iiint_{\Omega_2} \frac{\rho_1\of{\vec{\xi_1}} \rho_2\of{\vec{\xi_2}}}{\nrm{\vec{r}_1 - \vec{r}_2}^3} \of{\vec{r}_1 - \vec{r}_2} \de^3\xi_1 \de^3\xi_2\]
\[\vefi(\vec{r}) = k_\mathrm{C} \iiint_\Omega \frac{\rho\of{\vec{\xi}}}{\nrm{\vec{r} - \vec{\xi}}^3} \of{\vec{r} - \vec{\xi}} \de^3\xi\]
\textbf{Ampère's law}
\[\vec{F}_{12} = k_\mathrm{A} \iiint_{\Omega_1} \iiint_{\Omega_2} \frac{\vec{j}_1\of{\vec{\xi_1}} \times \of{\vec{j}_2\of{\vec{\xi_2}} \times \of{\vec{\xi}_1 - \vec{\xi}_2}}}{\nrm{\vec{\xi}_1 - \vec{\xi}_2}^3} \de^3\xi_1 \de^3\xi_2\]
\textbf{Biot-Savart's law}
\[\vbfi(\vec{r}) = k_\mathrm{S} \iiint_\Omega \frac{\vec{j}\of{\vec{\xi}} \times \of{\vec{r} - \vec{\xi}}}{\nrm{\vec{r} - \vec{\xi}}^3} \de^3\xi\]
%
%
\newpage
\section{Continuum mechanics}
Consider a body undergoing deformation, a reference configuration in which the
coordinates of the points are given by \(\vec{X}\) (Lagrangian or material picture) and the
current configuration in which they are given by \(\vec{x}\) (Eulerian or spatial picture).\\[1em]
A \textbf{motion} in the body is given by a function \(\vec{\chi}\) mapping the coordinates
in the reference configuration to those in the current configuration.
\[\vec{x} = \vec{\chi}\of{\vec{X},t}\]
We take this function to be invertible; we denote its inverse by \(\vec{\xi}\).
\[\vec{X} = \vec{\xi}\of{\vec{x},t}\]
We moreover take \(\vec{\chi}\) to be continuously differentiable and name its
gradient (a rank 2 tensor) the \textbf{material deformation gradient} \(\vec{F}\).
\[\vec{F} = \pderiv{\vec{x}}{\vec{X}} = \Grd{\vec{x}}\]
In the same way, we name the gradient of \(\vec{\xi}\) the \textbf{spatial
deformation gradient} \(\vec{H}\).
\[\vec{H} = \pderiv{\vec{X}}{\vec{x}} = \grd{\vec{X}}\]
These two tensors are such that the following relations on the differentials are true.
\[\de\vec{x} = \vec{F} \cdot \de\vec{X} \qquad \de\vec{X} = \vec{H} \cdot \de\vec{x}\]
The spatial deformation gradient is the inverse of the material deformation gradient,
depending on the order in which they are contracted, we obtain either the Lagrangian
or Eulerian identity tensor.
\[\vec{F} \cdot \vec{H} = \vec{I}_\mathrm{E} \qquad \vec{H} \cdot \vec{F} = \vec{I}_\mathrm{L}\]
Another useful property of the material deformation gradient is that its determinant,
usually denoted by \(\mathcal{J}\) is a measure of the change of a volume element
(trivial, since \(\vec{F}\) is the Jacobian of the coordinate change).
\[\mathcal{J} = \det \vec{F} \qquad \de^3x = \mathcal{J} \de^3X\]
A useful relation to know is the derivative of \(\mathcal{J}\) with regards to \(\vec{F}\),
which is just a restatement of the formula for the derivative of a determinant.
\[\pderiv{\mathcal{J}}{\vec{F}} = \mathcal{J} \, \tr{\vec{H}}\]
Less trivial is \textbf{Nanson's formula}, used to change the oriented area elements,
here \(\vec{n}\) is the normal vector in the current configuration and \(\vec{N}\)
is the normal vector in the reference configuration.
\[\vec{n} \, \de^2x = \mathcal{J} \tr{\vec{H}} \cdot \vec{N} \, \de^2X\]
We now take the time derivative of the material deformation gradient.
\[\dot{\vec{F}} = \D{}{t} \pderiv{\vec{x}}{\vec{X}} = \pderiv{}{\vec{X}} \D{\vec{x}}{t} = \pderiv{\vec{V}}{\vec{X}} = \Grd{\vec{V}}\]
We have introduced the \textbf{velocity} \(\vec{V}\of{\vec{X},t}\) as the time derivative of the
position in the current configuration.
The time derivative of the material deformation gradient is thus the material velocity gradient.
It is often more useful to express things in terms of a spatial gradient,
in which case we denote the velocity by \(\vec{v}\of{\vec{x},t}\) because we consider it in
the Eulerian picture, rather than in the Lagrangian.
\[\dot{\vec{F}} = \pderiv{\vec{v}}{\vec{x}} \cdot \pderiv{\vec{x}}{\vec{X}} = \vec{L} \cdot \vec{F}\]
This led us to defining the \textbf{spatial velocity gradient} \(\vec{L}\).
\[\vec{L} = \pderiv{\vec{v}}{\vec{x}} = \grd{\vec{v}}\]
We can use this result to compute the time derivative of the spatial deformation gradient, we
do this by invoking the formula for the derivative of matrix inverse.
\[\dot{\vec{H}} = - \vec{H} \cdot \dot{\vec{F}} \cdot \vec{H} = - \vec{H} \cdot \vec{L} \cdot \vec{F} \cdot \vec{H} = - \vec{H} \cdot \vec{L}\]
And we can compute the time derivative of the determinant \(\mathcal{J}\) as well.
\[\dot{\mathcal{J}} = \pderiv{\mathcal{J}}{\vec{F}} : \dot{\vec{F}} = \mathcal{J} \tr{\vec{H}} : \vec{L} \cdot \vec{F} = \mathcal{J} \trace{\vec{L}} = \mathcal{J} \, \dvg{\vec{v}}\]
Where we have used the property that \(\vec{A}:\vec{B} = \trace\of{\tr{\vec{A}}\cdot\vec{B}}\)
and the fact that the trace is conserved under transformations, so \(\trace\of{\vec{H}\cdot\vec{L}\cdot{\vec{F}}} = \trace{\vec{L}}\).
%
%
\subsection{Proof of the Reynolds Transport Theorem}
%
Consider the flux of some vector field \(\vec{a}\of{\vec{x},t}\) through a surface
\(\Sigma\of{t}\), and take its time derivative, which will depend both on the intrinsic
change in the vector field, and in the change of shape of the surface.
\[\D{\Phi_\vec{a}}{t} = \D{}{t} \iint_{\Sigma\of{t}} \vec{a}\of{\vec{x},t} \cdot \vec{n}\of{\vec{x},t} \de^2x\]
We use Nanson's formula to bring the integral to the reference configuration, where
the domain of intgration is independent of time, we denote by \(\vec{A}\) the material
representation of the vector field: \(\vec{A}\of{\vec{X},t} = \vec{a}\of{\vec{x}\of{\vec{X},t},t}\).
\[\D{\Phi_\vec{a}}{t} = \D{}{t} \iint_{\Sigma_0} \vec{A} \cdot \mathcal{J} \tr{\vec{H}} \cdot \vec{N}\,\de^2X\]
We can now bring the derivative inside the integral, giving us the following.
\[\D{\Phi_\vec{a}}{t} = \iint_{\Sigma_0} \of{\dot{\vec{A}} \cdot \mathcal{J} \tr{\vec{H}} +
    \vec{A} \cdot \dot{\mathcal{J}} \tr{\vec{H}} +
    \vec{A} \cdot \mathcal{J} \tr{\dot{\vec{H}}}} \cdot \vec{N}\,\de^2X
\]
Substituting the relevant quantities we find the following.
\begin{align*}
    \D{\Phi_\vec{a}}{t} &= \iint_{\Sigma_0} \of{\dot{\vec{A}} \cdot \mathcal{J} \tr{\vec{H}} +
    \vec{A} \cdot \of{\mathcal{J} \trace \vec{L}} \tr{\vec{H}} -
    \vec{A} \cdot \mathcal{J} \of{\tr{\vec{L}} \cdot \tr{\vec{H}}}} \cdot \vec{N}\,\de^2X \\
    &= \iint_{\Sigma_0} \of{\dot{\vec{A}} + \vec{A} \trace \vec{L} - \vec{A} \cdot \tr{\vec{L}}} \cdot \mathcal{J} \tr{\vec{H}} \cdot \vec{N}\,\de^2X
\end{align*}
We can now apply Nanson's formula backwards to bring the integral back to the
current configuration.
\[\D{\Phi_\vec{a}}{t}= \iint_{\Sigma\of{t}} \of{\dot{\vec{a}} + \vec{a} \trace \vec{L} - \vec{a} \cdot \tr{\vec{L}}} \cdot \vec{n}\,\de^2x\]
We expand \(\dot{\vec{a}}\) as the material derivative and replace \(\vec{L}\) with the
relevant expressions in terms of \(\vec{v}\).
\[\D{\Phi_\vec{a}}{t}= \iint_{\Sigma\of{t}} \of{\pderiv{\vec{a}}{t} + \vec{v} \cdot \grd{\vec{a}} + \vec{a} \of{\dvg{\vec{v}}} - \vec{a} \cdot \grd{\vec{v}}} \cdot \vec{n}\,\de^2x\]
We now recall the following vector identity for the curl of a cross product.
\[\crl{\of{\vec{a}\times\vec{v}}} = \vec{a} \of{\dvg{\vec{v}}} - \vec{v} \of{\dvg{\vec{a}}} + \vec{v} \cdot \grd{\vec{a}} - \vec{a} \cdot \grd{\vec{v}}\]
We notice that most of the terms appear in our expression, we thus obtain one of
the many possible expressions of the transport theorem for the flux of a vector field.
\[\D{\Phi_\vec{a}}{t}= \iint_{\Sigma\of{t}} \of{\pderiv{\vec{a}}{t} + \vec{v} \of{\dvg{\vec{a}}} + \crl{\of{\vec{a}\times\vec{v}}}} \cdot \vec{n}\,\de^2x\]
Using the curl theorem we can also bring the last term in terms of a circulation.
\[\D{\Phi_\vec{a}}{t}= \iint_{\Sigma\of{t}} \of{\pderiv{\vec{a}}{t} + \vec{v} \of{\dvg{\vec{a}}}} \cdot \vec{n}\,\de^2x
  + \oint_{\partial\Sigma\of{t}} \of{\vec{a}\times\vec{v}} \cdot \vec{t} \, \de x\]
This form is especially useful for solenoidal fields, where \(\dvg{\vec{a}}\) disappears.
\[\D{\Phi_\vec{a}}{t}= \iint_{\Sigma\of{t}} \pderiv{\vec{a}}{t} \cdot \vec{n}\,\de^2x
  + \oint_{\partial\Sigma\of{t}} \of{\vec{a}\times\vec{v}} \cdot \vec{t} \, \de x\]
%
%
\newpage
\section{Fourier Transforms}
%
For the sake of generality we define the Fourier transform as follows, depending
on parameters \(\mathcal{A}\), \(\mathcal{B}\) and \(a\).
\begin{equation}
  \begin{matrix*}[l]
  f\of{\omega} &=& \displaystyle \mathcal{A} \int_{\mathbb{R}} f\of{t} e^{-\icmp a \omega t} \de t \\[1em]
  f\of{t} &=& \displaystyle \mathcal{B} \int_{\mathbb{R}} f\of{\omega} e^{\icmp a \omega t} \de \omega
  \end{matrix*}
\end{equation}
We can find a relation between these parameters by checking that the inverse transform
gives the correct result.
\begin{align*}
f\of{t} &= \mathcal{B} \int \of{\mathcal{A} \int f\of{\tau} e^{-\icmp a \omega \tau} \de \tau} e^{\icmp a \omega t} \de \omega \\
f\of{t} &= \mathcal{A}\mathcal{B} \iint f\of{\tau} e^{-\icmp a \omega \of{t - \tau}} \de\tau \de\omega
\end{align*}
We proceed by recognising the integral expression of Dirac’s delta.
\[f\of{t} = \mathcal{A}\mathcal{B} \frac{2\pi}{\abs{a}} \iint f\of{\tau} \delta\of{t - \tau} \de\tau = \mathcal{A}\mathcal{B} \frac{2\pi}{\abs{a}} f\of{t}\]
From here we obtain the following relation.
\begin{equation}
\mathcal{A}\mathcal{B} = \frac{\abs{a}}{2\pi}
\end{equation}
In a similar vein we can extend the definition of Fourier transform to a function
of multiple (more specifically \(n\)) variables.
\[
  \begin{matrix*}[l]
  f\of{\offf{\omega_m}} &=& \displaystyle \mathcal{A}^n \int_{\mathbb{R}^n} f\of{\offf{t_m}} \prod_{j=1}^n e^{-\icmp a \omega_j t_j} \de t_j \\[1em]
  f\of{\offf{t_m}} &=& \displaystyle \mathcal{B}^n \int_{\mathbb{R}^n} f\of{\offf{\omega_m}} \prod_{j=1}^n e^{\icmp a \omega_j t_j} \de \omega_j
  \end{matrix*}
\]
%
%
\subsection{Generalised Convolution Theorem}
%
Consider a function \(h\) of one variable, defined as the convolution of two
functions of \(n\) variables \(f\) and \(g\).
\[h\of{t} = \int\!\!\!\dots\!\!\!\int f\of{\offf{t-\tau_m}} g\of{\offf{\tau_m}} \prod_{j=1}^n \de\tau_j\]
We analyse its Fourier transform.
\[h\of{\omega} = \mathcal{A} \int\!\!\!\dots\!\!\!\int e^{- \icmp a \omega t} f\of{\offf{t-\tau_m}} g\of{\offf{\tau_m}} \prod_{j=1}^n \de\tau_j \de t\]
We can decouple the various variables by introducing a number of Dirac deltas.
\[h\of{\omega} = \mathcal{A} \int\!\!\!\dots\!\!\!\int e^{- \icmp a \omega t} f\of{\offf{\theta_m-\tau_m}} g\of{\offf{\tau_m}} \prod_{j=1}^n \delta\of{\theta_j - t} \de\tau_j \de \theta_j \de t\]
Then we perform a change of variables \(\vartheta_j = \theta_j - \tau_j\).
\[h\of{\omega} = \mathcal{A} \int\!\!\!\dots\!\!\!\int e^{- \icmp a \omega t} f\of{\offf{\vartheta_m}} g\of{\offf{\tau_m}} \prod_{j=1}^n \delta\of{\vartheta_j + \tau_j - t} \de\tau_j \de \vartheta_j \de t\]
We replace the deltas with their integral expression.
\[h\of{\omega} = \mathcal{A}^{n+1} \mathcal{B}^n \int\!\!\!\dots\!\!\!\int e^{- \icmp a \omega t} f\of{\offf{\vartheta_m}} g\of{\offf{\tau_m}}
\prod_{j=1}^n e^{-\icmp a \omega_j \of{\vartheta_j + \tau_j - t}} \de\tau_j \de \vartheta_j \de \omega_j \de t\]
For brevity we denote \(\Omega = \sum \omega_j\).
\[h\of{\omega} = \mathcal{A}^{n+1} \mathcal{B}^n \int\!\!\!\dots\!\!\!\int e^{- \icmp a \of{\omega - \Omega} t}
\prod_{j=1}^n f\of{\offf{\vartheta_m}}e^{-\icmp a \omega_j \vartheta_j} g\of{\offf{\tau_m}}e^{-\icmp a \omega_j \tau_j} \de\tau_j \de \vartheta_j \de \omega_j \de t\]
Solving the integrals in \(\de\tau_j\) and \(\de\vartheta_j\) we obtain the Fourier
transforms of \(f\) and \(g\).
\begin{align*}
h\of{\omega} &= \mathcal{A}^{n+1} \mathcal{B}^n \int\!\!\!\dots\!\!\!\int e^{- \icmp a \of{\omega - \Omega} t}
\prod_{j=1}^n \frac{f\of{\offf{\omega_m}}}{\mathcal{A}^n} \frac{g\of{\offf{\omega_m}}}{\mathcal{A}^n} \de \omega_j \de t\\
h\of{\omega} &= \mathcal{A}^{1-n} \mathcal{B}^n \int\!\!\!\dots\!\!\!\int e^{- \icmp a \of{\omega - \Omega} t}
f\of{\offf{\omega_m}} g\of{\offf{\omega_m}} \prod_{j=1}^n \de \omega_j \de t
\end{align*}
Integrating over \(\de t\) gives us a delta instead.
\begin{align*}
h\of{\omega} &= \mathcal{A}^{1-n} \mathcal{B}^n \int\!\!\!\dots\!\!\!\int \frac{\delta\of{\omega - \Omega}}{\mathcal{A}\mathcal{B}}
f\of{\offf{\omega_m}} g\of{\offf{\omega_m}} \prod_{j=1}^n \de \omega_j\\
h\of{\omega} &= \mathcal{A}^{-n} \mathcal{B}^{n-1} \int\!\!\!\dots\!\!\!\int \delta\of{\omega - \Omega}
f\of{\offf{\omega_m}} g\of{\offf{\omega_m}} \prod_{j=1}^n \de \omega_j
\end{align*}
By rearranging we obtain the final form of the theorem.
\begin{equation}
h\of{\omega} = \frac{1}{\mathcal{B}} \of{\frac{\mathcal{B}}{\mathcal{A}}}^{n} \int\!\!\!\dots\!\!\!\int f\of{\offf{\omega_m}} g\of{\offf{\omega_m}} \delta\parens[\Big]{\omega - \sum_{j=1}^n \omega_j} \prod_{j=1}^n \de \omega_j
\end{equation}
It is easy to see that for \(n=1\) we obtain the usual convolution theorem.
\[h\of{\omega} = \frac{1}{\mathcal{B}} \frac{\mathcal{B}}{\mathcal{A}} \int f\of{\omega_1} g\of{\omega_1} \delta\of{\omega - \omega_1} \de \omega_1 = \frac{f\of{\omega}g\of{\omega}}{\mathcal{A}}\]
