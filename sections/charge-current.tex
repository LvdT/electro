\section{Charge and current}
%
We take the concept of \textbf{electric charge} \(Q\) as a primitive.
All particles have a certain electric charge, which can be positive, negative or zero,
this value determines how the particle couples with electromagnetic interactions.
A fundamental law of physics is that total electric charge is conserved.
The amount of charge contained within some region of space can still change due
to particles moving in or out of the region itself, so that in general \(Q\) can
be taken to be a function of time.\\[1em]
The force acting between two point charges \(Q_1\) and \(Q_2\) at a distance \(d\)
from each other is observed experimentally to follow \textbf{Coulomb's law}.
\[F = k_\mathrm{C} \frac{\Abs{Q_1}\Abs{Q_2}}{d^2}\]
The choice of units for charge is implicit in the value given to \(k_\mathrm{C}\);
two fundamentally distinct approaches are possible: in the first electrical units
arise naturally from mechanical units and \(k_\mathrm{C}\) is just a number, in the second
electrical phenomena are linked to a new \emph{base} quantity and \(k_\mathrm{C}\)
is a constant with its own units.\\[1em]
We follow by defining the \textbf{electric current} \(i\) as the rate of change of
the electric charge contained in a region of space, scaled by a constant \(\kappac\).
The negative sign is used so that a positive current means that charge is leaving
the region.
\begin{equation}\label{eq::current-definition}
  i = - \frac{1}{\kappac} \pderiv{Q}{t}
\end{equation}
While \(\kappac\) is often taken to be unity, its presence is useful
for several reasons.
It allows us to choose the units of current independently of the units of charge,
moreover, with the choice \(\kappac = c\), it gives additional symmetry
to Maxwell's equations and makes the relativistic treatment of electromagnetism
appear more natural.\\[1em]
The force (per unit length) acting between two straight wires at a distance \(d\)
from each other, each carrying a \emph{constant} current (respectively \(i_1\)
and \(i_2\)) is observed experimentally to follow \textbf{Ampère's force law}.
\[\frac{F}{L} = 2 \, k_\mathrm{A} \frac{\Abs{i_1}\Abs{i_2}}{d}\]
Once again, the choice of units for current is implicit in the value given to
\(k_\mathrm{A}\) and we can either take it to be a number, or give it units.
We could, in principle, define two distinct base quantities, one for charge and
one for current, and choose the appropriate units for \(\kappac\),
although this is not very convenient in practice.
%
%
\subsection{Densities}
%
It is often useful to work with \textbf{charge density} \(\rho\), rather than
total electric charge.
We define \(\rho\) as a function of position and time, such that its volume integral
over some region \(\Omega\) gives us exactly the charge contained in the region
at the given time.
\begin{equation}\label{eq::charge-density}
  Q\of{t} = \iiint_\Omega \rho\of{\vec{\xi},t} \de^3\xi
\end{equation}
Since total electric charge must be conserved, this implies that in the presence
of current, some motion of charges across the boundary of the region is taking place.
There exists some some quantity, which we shall name the \textbf{current density}
vector \(\vec{j}\), of which the total electric current is the flux.
\begin{equation}\label{eq::current-flux}
  i(t) = \iint_{\Sigma} \vec{j}\of{\vec{\xi},t} \cdot \vec{n} \, \de^2\xi
\end{equation}
Two special cases are the charge density of a point charge \(Q\) located at \(\vec{r}_0\)
\[\rho\of{\vec{r}} = Q \, \delta\of{\vec{r} - \vec{r}_0}\]
and the current density of a thin wire described by a curve \(\gamma\) and carrying a current \(i\).
\[\vec{j}\of{\vec{r}} = i \, \int_{\mathbb{R}} \delta\of{\vec{r} - \vec{\gamma}\of{s}} \vec{t}\of{s}\de s\]
%
%
\subsection{Continuity equation}
%
We now take equation and \eqref{eq::current-flux} specify it to the case where
\(\Sigma\) is the boundary of the region \(\Omega\) where the charge is contained,
we also take equation \eqref{eq::charge-density} and
substitute both in \eqref{eq::current-definition} and apply the divergence theorem.
\begin{align*}
  \oiint_{\partial \Omega} \vec{j}\of{\vec{\xi},t} \cdot \vec{n} \, \de^2\xi &= - \frac{1}{\kappac} \pderiv{}{t} \iiint_\Omega \rho\of{\vec{\xi},t} \de^3\xi \\
  \iiint_{\Omega} \dvg{\vec{j}} \, \de^3\xi &= - \frac{1}{\kappac} \iiint_\Omega \pderiv{\rho}{t} \de^3\xi
\end{align*}
\[\iiint_{\Omega} \of{\dvg{\vec{j}} + \frac{1}{\kappac} \pderiv{\rho}{t}} \de^3\xi = 0\]
Since this identity must hold regardless of the choice of \(\Omega\) it must be
that the integrand itself is zero. We get the \textbf{continuity equation}.
\begin{equation}\label{eq::continuity}
  \frac{1}{\kappac} \pderiv{\rho}{t} + \dvg{\vec{j}} = 0
\end{equation}
We can now relate the current density to the velocity field. Let us consider a
surface enclosing a region \(\Omega_0\) at time \(t_0\) and containing a charge \(Q\).
We define \(\Omega\of{t}\) as the region containing exactly the same particles as
\(\Omega_0\). If the motion of the particles is continuous, the motion of the surface
will also be continuous; we can therefore apply the Reynolds transport theorem
from continuum mechanics. We take the time derivative of the charge within the
region, which must be zero since it is constant.
\[\D{Q}{t} = \D{}{t} \iiint_{\Omega\of{t}} \rho\of{\vec{\xi},t} \de^3\xi = \iiint_{\Omega\of{t}} \of{\pderiv{\rho}{t} + \dvg{\of{\rho \, \vec{v}}}} \de^3\xi = 0\]
If we divide everything by \(\kappac\) we obtain an integrand that shows
a strong resemblance to the continuity equation.
\[\frac{1}{\kappac} \D{Q}{t} = \iiint_{\Omega\of{t}} \of{\frac{1}{\kappac} \pderiv{\rho}{t} + \dvg{\frac{\rho \, \vec{v}}{\kappac}}} \de^3\xi = 0\]
As this integral must be zero for any choice of the starting sufrace \(\Omega_0\),
it is the integrand itself that must vanish. For this to happen it must be true
that the term of which we are taking the divergence must be the current density.
\begin{equation}\label{eq::current-density-moving}
  \vec{j}\of{\vec{r},t} = \frac{\vec{v}\of{\vec{r},t} \, \rho\of{\vec{r},t}}{\kappac}
\end{equation}
%
%
