\section{Electromagnetic Radiation}
%
We have already shown that the solution to Maxwell's equations in the absence of
charges and currents gives us d'Alembert's equation.
\[\frac{1}{c^2} \pderiv[2]{\vefi}{t} - \lp{\vefi} = 0 \qquad \frac{1}{c^2} \pderiv[2]{\vbfi}{t} - \lp{\vbfi} = 0\]
We know that a possible set of solutions to d'Alembert's equation are plane waves, we
take \(\omega\) to be the pulsatance and \(\vec{k}\) to be the (angular) wave vector of the solution.
\[\vefi\of{\vec{r},t} = \vefi_0 \, \exp^{\iun \vec{k} \cdot \vec{r}} \exp^{-\iun \omega t} \qquad \vbfi\of{\vec{r},t} = \vbfi_0 \, \exp^{\iun \vec{k} \cdot \vec{r}} \exp^{-\iun \omega t}\]
%We compute the Laplacian and the second derivative of the electric field.
%\[\lp{\vefi} = - k^2 \vefi \qquad \pderiv[2]{\vefi}{t} = - \omega^2 \vefi\]
%
%
\subsection{Dispersion relation}
%
Substituting the plane waves into d'Alembert's equation gives us the \emph{dispersion relation}.
\[\frac{\omega^2}{c^2} \vefi - k^2 \vefi = 0 \qquad \omega = c \, k\]
%
%
\subsection{Orientation of the fields}
%
Substituting the plane waves into \(\mathrm{M}_\mathrm{I}\) and \(\mathrm{M}_\mathrm{II}\)
gives us the following conditions.
\[\vec{k} \cdot \vefi = 0 \qquad \vec{k} \cdot \vbfi = 0\]
This shows that the electric and magnetic fields are \emph{transverse}, i.e. they are perpendicular to the direction
of propagation of the wave.\\
Substituting instead into \(\mathrm{M}_\mathrm{III}\) and \(\mathrm{M}_\mathrm{IV}\)
gives us the following conditions.
\[\vec{k} \times \vefi = \frac{\omega}{\kappaf} \vbfi \qquad \vec{k} \times \vbfi = - \frac{\kappaf \, \omega}{c^2} \vefi\]
Through the dispersion relation, these can be rewritten as follows.
\[\vbfi = \frac{\kappaf}{c} \uvec{k} \times \vefi \qquad \vefi = - \frac{c}{\kappaf} \uvec{k} \times \vbfi\]
This shows that the electric and magnetic field are perpendicular to each other.
%Taking the norm of the second equation, recalling the identity for the norm of
%a cross product and using the dispersion relation, we find the relation between the
%magnitudes of the electric and magnetic fields in free space.
%\[\nrm{\vec{k} \times \vefi} = \sqrt{k^2 \efi^2 - \of{\vec{k}\cdot\vefi}^2} = k \efi\]
%\[k \efi = \frac{\omega}{\kappaf} \bfi \qquad \efi = \frac{c}{\kappaf} \bfi\]
%
%
\subsection{Impedance of free space}
%
We can also compute the Poynting vector of the wave.
\[\vpoy = \frac{\vefi \times \vbfi}{\alphaA} =
\frac{\kappaf}{c} \frac{1}{\alphaA} \vefi \times \of{\uvec{k} \times \vefi} =
\frac{\kappaf}{c} \frac{\efi^2}{\alphaA} \uvec{k}
\]
We take notice of the fact that it is in the same direction as the wave vector,
i.e. in the direction of wave propagation.
Taking just its norm we can write the following.
\[\poy = \frac{\efi^2}{Z_0} \qquad Z_0 = \frac{c}{\kappaf} \alphaA = \frac{\kappac}{c} \alphaG\]
Where \(Z_0\) is the \textbf{impedance of free space}.
Another convention is sometimes in use, where one naïvely extends the
impedance of free space found in the SI to another unit system. This does not, in
general, have the correct units to be called an impedance and doesn’t, of course, give
the correct ratio between the Poynting vector and the electric field. We shall denote
this by \(\widetilde{Z}_0\).
\[\widetilde{Z}_0 = \sqrt{\alphaG \, \alphaA} = \frac{c}{\sqrt{\kappac \kappaf}} \, \alphaA = \frac{\sqrt{\kappac \kappaf}}{c} \, \alphaG\]
We see that this definition is identical to \(Z_0\) only when \(\kappac = \kappaf\).
%
%
