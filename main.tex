\documentclass[12pt]{scrartcl}
\title{Electromagnetism \\and its units of measurement}
\author{I. Saltini}
\date{}
\setlength{\parindent}{0pt}

\usepackage{polyglossia}
\setmainlanguage[variant=british]{english}

%this avoids hypenation
\tolerance=1
\emergencystretch=\maxdimen
\hyphenpenalty=10000
\hbadness=10000

\usepackage{amssymb,amsmath,amsthm}
\usepackage{mathtools}
\usepackage{booktabs,multirow}
\usepackage{graphicx}
\usepackage{enumerate}

\usepackage{caption}
\usepackage{subcaption}
\usepackage[hidelinks,pdfusetitle]{hyperref}
\usepackage[figure]{hypcap}
\usepackage{float}

\usepackage{cool}
\Style{DSymb={\mathrm{d}},IntegrateDifferentialDSymb={\mathrm{d}}}

\usepackage[math-style=ISO,bold-style=ISO]{unicode-math}
\defaultfontfeatures{Ligatures=TeX,ExternalLocation=fonts/,Extension=.otf}
\setmathfont{XITSMath}
\setmathfont[range={\mathcal,\mathbfcal},StylisticSet=1]{XITSMath}
\defaultfontfeatures{
  Ligatures=TeX,
  ExternalLocation=fonts/,
  Extension=.otf,
  UprightFont=*R,
  ItalicFont=*I,
  BoldFont=*B,
  BoldItalicFont=*BI
}
\setmainfont{TGPagella}
\setsansfont{TGAdventor}
\setmonofont{TGCursor}

\usepackage[
  list-units=single,
  range-units=single,
  multi-part-units=single,
  exponent-product=\cdot,
  per-mode=fraction,
  math-ohm=\mathrm{\Omega},
  text-ohm={\ensuremath{\mathrm{\Omega}}},
  retain-explicit-plus=true
]{siunitx}
\DeclareSIUnit{\atp}{at.\percent}
\DeclareSIUnit{\magn}{\ensuremath{\times}}
\DeclareSIUnit{\rpm}{rpm}
\DeclareSIUnit{\nit}{nt}
\DeclareSIUnit{\talbot}{Tb}

\usepackage{xparse}

\NewDocumentCommand{\efi}{}{{\ensuremath{\mathcal{E}}}}
\NewDocumentCommand{\vefi}{}{{\ensuremath{\mathbfcal{E}}}}
\NewDocumentCommand{\bfi}{}{{\ensuremath{\mathcal{B}}}}
\NewDocumentCommand{\vbfi}{}{{\ensuremath{\mathbfcal{B}}}}
\NewDocumentCommand{\dfi}{}{{\ensuremath{\mathcal{D}}}}
\NewDocumentCommand{\vdfi}{}{{\ensuremath{\mathbfcal{D}}}}
\NewDocumentCommand{\hfi}{}{{\ensuremath{\mathcal{H}}}}
\NewDocumentCommand{\vhfi}{}{{\ensuremath{\mathbfcal{H}}}}
\NewDocumentCommand{\pfi}{}{{\ensuremath{\mathcal{P}}}}
\NewDocumentCommand{\vpfi}{}{{\ensuremath{\mathbfcal{P}}}}
\NewDocumentCommand{\mfi}{}{{\ensuremath{\mathcal{M}}}}
\NewDocumentCommand{\vmfi}{}{{\ensuremath{\mathbfcal{M}}}}
\NewDocumentCommand{\poy}{}{{\ensuremath{\mathcal{S}}}}
\NewDocumentCommand{\vpoy}{}{{\ensuremath{\mathbfcal{S}}}}
\NewDocumentCommand{\mxw}{}{{\ensuremath{\sigma}}}
\NewDocumentCommand{\vmxw}{}{{\ensuremath{\symbf{\sigma}}}}
\NewDocumentCommand{\afi}{}{{\ensuremath{\mathcal{A}}}}
\NewDocumentCommand{\vafi}{}{{\ensuremath{\mathbfcal{A}}}}
\NewDocumentCommand{\scp}{}{\ensuremath{\phi}}
\NewDocumentCommand{\frd}{}{{\ensuremath{\mathcal{F}}}}
\NewDocumentCommand{\vfrd}{}{{\ensuremath{\mathbfcal{F}}}}
\NewDocumentCommand{\frdg}{}{{\ensuremath{\mathcal{G}}}}
\NewDocumentCommand{\vfrdg}{}{{\ensuremath{\mathbfcal{G}}}}
\NewDocumentCommand{\emf}{}{{\ensuremath{\mathcal{V}}}}

\NewDocumentCommand{\de}{}{\ensuremath{\mathrm{d}}}
\NewDocumentCommand{\vde}{}{\ensuremath{\symbf{d}}}
\RenewDocumentCommand{\exp}{}{\mathrm{e}}
\NewDocumentCommand{\iun}{}{\ensuremath{\mathrm{i}}}

\NewDocumentCommand{\kappac}{}{\ensuremath{\kappa_\mathrm{c}}}
\NewDocumentCommand{\kappaf}{}{\ensuremath{\kappa_\mathrm{f}}}
\NewDocumentCommand{\alphaG}{}{\ensuremath{\alpha_\mathrm{G}}}
\NewDocumentCommand{\alphaA}{}{\ensuremath{\alpha_\mathrm{A}}}
\NewDocumentCommand{\gammaF}{}{\ensuremath{\gamma_\mathrm{F}}}
\NewDocumentCommand{\gammaM}{}{\ensuremath{\gamma_\mathrm{M}}}
\NewDocumentCommand{\betaG}{}{\ensuremath{\beta_\mathrm{G}}}
\NewDocumentCommand{\betaA}{}{\ensuremath{\beta_\mathrm{A}}}

\NewDocumentCommand{\abs}{m}{\left|{#1}\right|}
\NewDocumentCommand{\of}{m}{\left({#1}\right)}
\NewDocumentCommand{\brof}{m}{\left\{{#1}\right\}}
\RenewDocumentCommand{\vec}{m}{{\symbf{#1}}}
\NewDocumentCommand{\uvec}{m}{\symbf{u}_{#1}}
\NewDocumentCommand{\grd}{m}{\vec{\nabla} {#1}}
\NewDocumentCommand{\Grd}{m}{\mathop{\vec{\nabla}_0} {#1}}
\NewDocumentCommand{\lp}{m}{\nabla^2 {#1}}
\NewDocumentCommand{\dal}{m}{\Box^2 {#1}}
\NewDocumentCommand{\dvg}{m}{\vec{\nabla} \cdot {#1}}
\NewDocumentCommand{\crl}{m}{\vec{\nabla} \times {#1}}
\NewDocumentCommand{\nrm}{m}{\left\|{#1}\right\|}
\NewDocumentCommand{\tr}{m}{{#1}^\mathsf{T}}
\DeclareMathOperator{\trace}{tr}

% \usepackage{csquotes}
\usepackage[sorting=none,isbn=true,url=false,doi=false,backend=biber]{biblatex}

% \usepackage[usenames,dvipsnames]{xcolor}

\usepackage{tikz}
\usetikzlibrary{
  positioning,
  shapes,
  shadows,
  arrows,
  fit,
  decorations,
  patterns,
  mindmap
}

\usepackage{readarray}
\usepackage{ifthen}
\usepackage{pgfplots}
\usepackage{chemfig}

% \usepackage[most]{tcolorbox}
\newtcolorbox{examplebox}[1][0]{
  colback=MyPaleBlue,
  colframe=MyDarkBlue,
  colbacktitle=MyLiteBlue,enhanced,
  fonttitle=\bfseries,
  attach boxed title to top center={yshift=-2mm},
  title=#1
}


\begin{document}
\maketitle

\begin{abstract}
In the present document, we shall strive to provide a description of electromagnetism
that is independent of the system of units being used.
In order to do that, we shall introduce a number of constants, determine how these
constants relate to each other, and finally provide the values of the constants
for the most commonly used systems.
\end{abstract}
\section{Charge and current}
%
We take the concept of \textbf{electric charge} \(Q\) as a primitive.
All particles have a certain electric charge, which can be positive, negative or zero,
this value determines how the particle couples with electromagnetic interactions.
A fundamental law of physics is that total electric charge is conserved.
The amount of charge contained within some region of space can still change due
to particles moving in or out of the region itself, so that in general \(Q\) can
be taken to be a function of time.\\[1em]
The force acting between two point charges \(Q_1\) and \(Q_2\) at a distance \(d\)
from each other is observed experimentally to follow \textbf{Coulomb's law}.
\[F = k_\mathrm{C} \frac{\abs{Q_1}\abs{Q_2}}{d^2}\]
The choice of units for charge is implicit in the value given to \(k_\mathrm{C}\);
two fundamentally distinct approaches are possible: in the first electrical units
arise naturally from mechanical units and \(k_\mathrm{C}\) is just a number, in the second
electrical phenomena are linked to a new \emph{base} quantity and \(k_\mathrm{C}\)
is a constant with its own units.\\[1em]
We follow by defining the \textbf{electric current} \(i\) as the rate of change of
the electric charge contained in a region of space, scaled by a constant \(\kappac\).
The negative sign is used so that a positive current means that charge is leaving
the region.
\begin{equation}\label{eq::current-definition}
  i = - \frac{1}{\kappac} \pderiv{Q}{t}
\end{equation}
While \(\kappac\) is often taken to be unity, its presence is useful
for several reasons.
It allows us to choose the units of current independently of the units of charge,
moreover, with the choice \(\kappac = c\), it gives additional symmetry
to Maxwell's equations and makes the relativistic treatment of electromagnetism
appear more natural.\\[1em]
The force (per unit length) acting between two straight wires at a distance \(d\)
from each other, each carrying a \emph{constant} current (respectively \(i_1\)
and \(i_2\)) is observed experimentally to follow \textbf{Ampère's force law}.
\[\frac{F}{L} = 2 \, k_\mathrm{A} \frac{\abs{i_1}\abs{i_2}}{d}\]
Once again, the choice of units for current is implicit in the value given to
\(k_\mathrm{A}\) and we can either take it to be a number, or give it units.
We could, in principle, define two distinct base quantities, one for charge and
one for current, and choose the appropriate units for \(\kappac\),
although this is not very convenient in practice.
%
%
\subsection{Charge and current densitites}
%
It is often useful to work with \textbf{charge density} \(\rho\), rather than
total electric charge.
We define \(\rho\) as a function of position and time, such that its volume integral
over some region \(\Omega\) gives us exactly the charge contained in the region
at the given time.
\begin{equation}\label{eq::charge-density}
  Q\of{t} = \iiint_\Omega \rho\of{\vec{\xi},t} \de^3\xi
\end{equation}
Since total electric charge must be conserved, this implies that in the presence
of current, some motion of charges across the boundary of the region is taking place.
There exists some some quantity, which we shall name the \textbf{current density}
vector \(\vec{j}\), of which the total electric current is the flux.
\begin{equation}\label{eq::current-flux}
  i(t) = \iint_{\Sigma} \vec{j}\of{\vec{\xi},t} \cdot \vec{n} \, \de^2\xi
\end{equation}
Two special cases are the charge density of a point charge \(Q\) located at \(\vec{r}_0\)
\[\rho\of{\vec{r}} = Q \, \delta\of{\vec{r} - \vec{r}_0}\]
and the current density of a thin wire described by a curve \(\gamma\) and carrying a current \(i\).
\[\vec{j}\of{\vec{r}} = i \, \int_{\mathbb{R}} \delta\of{\vec{r} - \vec{\gamma}\of{s}} \vec{t}\of{s}\de s\]
%
%
\subsection{Continuity equation}
%
We now take equation and \eqref{eq::current-flux} specify it to the case where
\(\Sigma\) is the boundary of the region \(\Omega\) where the charge is contained,
we also take equation \eqref{eq::charge-density} and
substitute both in \eqref{eq::current-definition} and apply the divergence theorem.
\begin{align*}
  \oiint_{\partial \Omega} \vec{j}\of{\vec{\xi},t} \cdot \vec{n} \, \de^2\xi &= - \frac{1}{\kappac} \pderiv{}{t} \iiint_\Omega \rho\of{\vec{\xi},t} \de^3\xi \\
  \iiint_{\Omega} \dvg{\vec{j}} \, \de^3\xi &= - \frac{1}{\kappac} \iiint_\Omega \pderiv{\rho}{t} \de^3\xi
\end{align*}
\[\iiint_{\Omega} \of{\dvg{\vec{j}} + \frac{1}{\kappac} \pderiv{\rho}{t}} \de^3\xi = 0\]
Since this identity must hold regardless of the choice of \(\Omega\) it must be
that the integrand itself is zero. We get the \textbf{continuity equation}.
\begin{equation}\label{eq::continuity}
  \frac{1}{\kappac} \pderiv{\rho}{t} + \dvg{\vec{j}} = 0
\end{equation}
We can now relate the current density to the velocity field. Let us consider a
surface enclosing a region \(\Omega_0\) at time \(t_0\) and containing a charge \(Q\).
We define \(\Omega\of{t}\) as the region containing exactly the same particles as
\(\Omega_0\). If the motion of the particles is continuous, the motion of the surface
will also be continuous; we can therefore apply the Reynolds transport theorem
from continuum mechanics. We take the time derivative of the charge within the
region, which must be zero since it is constant.
\[\D{Q}{t} = \D{}{t} \iiint_{\Omega\of{t}} \rho\of{\vec{\xi},t} \de^3\xi = \iiint_{\Omega\of{t}} \of{\pderiv{\rho}{t} + \dvg{\of{\rho \, \vec{v}}}} \de^3\xi = 0\]
If we divide everything by \(\kappac\) we obtain an integrand that shows
a strong resemblance to the continuity equation.
\[\frac{1}{\kappac} \D{Q}{t} = \iiint_{\Omega\of{t}} \of{\frac{1}{\kappac} \pderiv{\rho}{t} + \dvg{\frac{\rho \, \vec{v}}{\kappac}}} \de^3\xi = 0\]
As this integral must be zero for any choice of the starting sufrace \(\Omega_0\),
it is the integrand itself that must vanish. For this to happen it must be true
that the term of which we are taking the divergence must be the current density.
\begin{equation}\label{eq::current-density-moving}
  \vec{j}\of{\vec{r},t} = \frac{\vec{v}\of{\vec{r},t} \, \rho\of{\vec{r},t}}{\kappac}
\end{equation}
%
%
\section{Electric and magnetic fields}
%
The force acting on a small \emph{test charge} \(q\) as a result of electromagnetic
phenomena can be split into two contributions, an electric force which is independent
of the charge's motion and a magnetic term which also depends on its velocity.
\[\vec{F}\of{\vec{r},t} = \vec{F}_\mathrm{e}\of{\vec{r},t} + \vec{F}_\mathrm{m}\of{\vec{v},\vec{r},t}\]
Experimentally, both terms are observed to be linear in \(q\).\\[1em]
The \textbf{electric field} \(\vefi\) is defined in terms of the electric force
and is obtained by factoring out charge, since we have just observed that
the force depends linearly upon it.
\[\vec{F}_\mathrm{e}\of{\vec{r},t} = q \, \vefi\of{\vec{r},t}\]
The \textbf{magnetic field} \(\vbfi\) behaves in a slightly more complicated manner.
The magnetic force is observed to always be perpendicular to the particle's velocity, which
suggests that a cross product is involved in the definition.
Moreover, we need to introduce a scaling constant \(\kappaf\), which will
be necessary to give us the correct value for the speed of light as an electromagnetic
wave.
\[\vec{F}_\mathrm{m}\of{\vec{v},\vec{r},t} = \frac{q\vec{v} \times \vbfi\of{\vec{r},t}}{\kappaf}\]
The sum of these two contributions is called the \textbf{Lorentz force}.
\begin{equation}
  \vec{F}\of{\vec{r},t} = q \of{\vefi\of{\vec{r},t} + \frac{\vec{v} \times \vbfi\of{\vec{r},t}}{\kappaf}}
\end{equation}
The Lorentz force can also be expressed in terms of charge and current densities,
in order to do so it is convenient to define the \textbf{force density} \(\vec{f}\),
a local property defined by the amount of force per unit volume.
\[\vec{F}(t) = \iiint_\Omega \vec{f}\of{\vec{\xi},t} \de^3 \xi\]
We start with the electric force, we can turn the expression above into a volume
integral by using the definition of the Dirac delta.
\[\vec{F}_\mathrm{e}\of{\vec{r},t} = \iiint_\Omega q \, \delta\of{\vec{r}-\vec{\xi}} \vefi\of{\vec{\xi},t}\de^3 \xi\]
The expression \(q \, \delta\of{\vec{r}-\vec{\xi}}\) is the charge density of a
point charge \(q\) located at \(\vec{r}\), this leads us to an expression for the
electric force density.
\[\vec{F}_\mathrm{e}\of{t} = \iiint_\Omega \rho\of{\vec{\xi},t} \vefi\of{\vec{\xi},t} \de^3 \xi
\qquad \vec{f}_\mathrm{e}\of{\vec{r},t} = \rho\of{\vec{r},t} \vefi\of{\vec{r},t}\]
We can proceed in a similar manner with the magnetic force, we immediately identify
the expression for the charge density.
\[\vec{F}_\mathrm{m}\of{t} = \iiint_\Omega \frac{\rho\of{\vec{\xi},t}\vec{v} \times \vbfi\of{\vec{\xi},t}}{\kappaf} \de^3 \xi\]
We now refer to equation \eqref{eq::current-density-moving} and replace \(\rho\of{\vec{\xi},t}\vec{v}\)
with \(\kappac \, \vec{j}\of{\vec{\xi},t}\), obtaining an expression for the
magnetic force density.
\[\vec{F}_\mathrm{m}\of{t} = \iiint_\Omega \frac{\kappac}{\kappaf} \vec{j}\of{\vec{\xi},t} \times \vbfi\of{\vec{\xi,t}} \de^3 \xi
\qquad \vec{f}_\mathrm{m}\of{\vec{r},t} = \frac{\kappac}{\kappaf} \, \vec{j}\of{\vec{r},t} \times \vbfi\of{\vec{r},t}\]
%
Taking both contributions together gives us the Lorentz force density.
\[\vec{f}\of{\vec{r},t} = \rho\of{\vec{r},t} \vefi\of{\vec{r},t} + \frac{\kappac}{\kappaf} \, \vec{j}\of{\vec{r},t} \times \vbfi\of{\vec{r},t}\]
%
%
\section{Electromotive force}
%
We consider a closed loop built out of an electrical conductor, delimiting a surface
\(\Sigma\). We define the \textbf{electromotive force} as the work per unit
of charge performed by the electromagnetic force on a charge going around the whole loop,
which is to say, the \emph{circulation} of the Lorentz force per unit charge.
\[\emf\of{t} = \frac{1}{q} \oint_{\partial\Sigma\of{t}} \vec{F}\of{\xi,t} \cdot \vec{t}\of{\xi,t} \de\xi\]
By substituting the expression for the Lorentz force we obtain the following.
\[\emf\of{t} = \oint_{\partial\Sigma\of{t}} \of{\vefi\of{\xi,t} + \frac{\vec{v}\of{\xi,t} \times \vbfi\of{\xi,t}}{\kappaf}} \cdot \vec{t}\of{\xi,t} \de\xi\]
%
The electromotive force is therefore the sum of two contributions, the first is
due to the electric field and is called \textbf{induced electromotive force},
the second is due to motion in a magnetic field and is called \textbf{motional
electromotive force}.
%
\[\emf = \emf_\mathrm{ind} + \emf_\mathrm{mot}\]
\[\emf_\mathrm{ind}\of{t} = \oint_{\partial\Sigma\of{t}} \vefi\of{\xi,t} \cdot \vec{t}\of{\xi,t} \de\xi \qquad
\emf_\mathrm{mot}\of{t} = \oint_{\partial\Sigma\of{t}} \frac{\vec{v}\of{\xi,t} \times \vbfi\of{\xi,t}}{\kappaf} \cdot \vec{t}\of{\xi,t} \de\xi\]
%
Experimentally, it is known that the electromotive force on a circuit depends upon
the time derivative of the magnetic flux through the surface delimited by the circuit;
this relationship is known as the \textbf{Faraday--Neumann--Lenz law}.
\[\emf = - k_\mathrm{F} \D{\Phi_\vbfi}{t}\]
%
%
\section{Joule--Lenz law}
%
Consider a charge distribution \(\rho\of{\vec{r},t}\) together with a velocity
field \(\vec{v}\of{\vec{r},t}\) in a region where both an electric and magnetic fields
exist.
The power density of such a system can be computed as the dot product between the
force density and the velocity field.
\[\dot{u} = \vec{f} \cdot \vec{v} = \rho \vefi \cdot \vec{v} + \frac{\kappac}{\kappaf} \of{\vec{v} \times \vbfi} \cdot \vec{v} = \vefi \cdot \of{\rho \, \vec{v}}\]
The second term vanishes because we're taking the dot product of two perpendicular
vectors. We recognise the expression for the current density.
\[\dot{u} = \kappac \, \vefi \cdot \vec{j} \qquad \frac{1}{\kappac} \D{u}{t} = \vefi \cdot \vec{j}\]
This is known as the \textbf{Joule--Lenz law}.
%
%
\section{Maxwell's equations}
%
While the Lorentz force gives us a way to compute the way a distribution of charges
and currents interacts with the electromagnetic field, we still need a way to
determine how the fields themselves are generated by a distribution of charges.
This is achieved by \textbf{Maxwell's equations}, which can be stated either in
integral or differential form.
For the sake of simplicity, we choose to use the differential form.
\begin{center}
  \begin{tabular}{ccccl}
    \(\mathrm{M}_\mathrm{I}\) & \textbf{Gauss's law (for \(\vefi\))} & \(\dvg{\vefi}\) & \(=\) & \(\alphaG \, \rho\) \\[1em]
    \(\mathrm{M}_\mathrm{II}\) & \textbf{Gauss's law (for \(\vbfi\))} & \(\dvg{\vbfi}\) & \(=\) & \(0\) \\[1em]
    \(\mathrm{M}_\mathrm{III}\) & \textbf{Faraday--Maxwell's law} & \(\crl{\vefi}\) & \(=\) & \(\displaystyle - \frac{1}{\gammaF} \pderiv{\vbfi}{t}\) \\[1em]
    \(\mathrm{M}_\mathrm{IV}\) & \textbf{Ampère--Maxwell's law} & \(\crl{\vbfi}\) & \(=\) & \(\displaystyle \alphaA \, \vec{j} + \frac{1}{\gammaM} \pderiv{\vefi}{t}\) \\
  \end{tabular}
\end{center}
We have used two different types of constants, \(\alphaG\)
and \(\alphaA\) that couple the fields to the charges and currents respectively,
and \(\gammaF\) and \(\gammaM\) that couple
the fields to each other.
%
\subsection{Continuity equation}
%
We can verify that the continuity equation is consistent with Maxwell's equation,
under a proper choice of constants.
We begin by taking the divergence of \(\mathrm{M}_\mathrm{IV}\), recalling that
the divergence of a curl must be zero.
\begin{align*}
  \dvg{\of{\crl{\vbfi}}} &= \alphaA \dvg{\vec{j}} + \frac{1}{\gammaM} \dvg{\pderiv{\vefi}{t}} \\
  0 &= \alphaA \dvg{\vec{j}} + \frac{1}{\gammaM} \pderiv{}{t} \dvg{\vefi} \\
  0 &= \alphaA \dvg{\vec{j}} + \frac{\alphaG}{\gammaM} \pderiv{\rho}{t}
\end{align*}
We obtain something that resembles a continuity equation.
\[\frac{\alphaG}{\alphaA\gammaM} \pderiv{\rho}{t} + \dvg{\vec{j}} = 0\]
Comparing it to equation \eqref{eq::continuity} we obtain a relation between the constants.
\[\frac{\alphaA\gammaM}{\alphaG} = \kappac\]
%
\subsection{Propagation of light}
%
We consider Maxwell's equation in the absence of charges or currents.
\begin{center}
  \begin{tabular}{cccl}
    \(\mathrm{M}_\mathrm{I}\) & \(\dvg{\vefi}\) & \(=\) & \(0\) \\[1em]
    \(\mathrm{M}_\mathrm{II}\) & \(\dvg{\vbfi}\) & \(=\) & \(0\) \\[1em]
    \(\mathrm{M}_\mathrm{III}\) & \(\crl{\vefi}\) & \(=\) & \(\displaystyle - \frac{1}{\gammaF} \pderiv{\vbfi}{t}\) \\[1em]
    \(\mathrm{M}_\mathrm{IV}\) & \(\crl{\vbfi}\) & \(=\) & \(\displaystyle \frac{1}{\gammaM} \pderiv{\vefi}{t}\) \\
  \end{tabular}
\end{center}
We take the curl of \(\mathrm{M}_\mathrm{III}\), use the identity \(\crl{\vec{a}} = \grd{\of{\dvg{\vec{a}}}} - \lp{\vec{a}}\)
and substitute \(\mathrm{M}_\mathrm{I}\) and \(\mathrm{M}_\mathrm{IV}\) where the
corresponding terms appear.
\begin{align*}
  \crl{\of{\crl{\vefi}}} &= - \frac{1}{\gammaF} \pderiv{}{t} \crl{\vbfi} \\
  \grd{\of{\dvg{\vefi}}} - \lp{\vefi} &= - \frac{1}{\gammaF} \pderiv{}{t} \crl{\vbfi} \\
  - \lp{\vefi} &= - \frac{1}{\gammaF\gammaM} \pderiv[2]{\vefi}{t}
\end{align*}
We obtain d'Alembert's equation for the electric field.
\[\frac{1}{\gammaF\gammaM} \pderiv[2]{\vefi}{t} - \lp{\vefi} = 0\]
We know for a fact that the propagation speed of an electromagnetic wave is \(c\);
in terms of our constants, this means that the following condition must hold.
\[\gammaF\gammaM = c^2\]
The process can be repeated by taking the curl of \(\mathrm{M}_\mathrm{IV}\), and
proceeding in a similar manner, but this won't give us any new condition.
The fact that electric and magnetic waves travel at the same speed is intrinsic
in the structure of Maxwell's equations.
\begin{align*}
  \crl{\of{\crl{\vbfi}}} &= \frac{1}{\gammaM} \pderiv{}{t} \crl{\vefi} \\
  \grd{\of{\dvg{\vbfi}}} - \lp{\vbfi} &= \frac{1}{\gammaM} \pderiv{}{t} \crl{\vefi} \\
  - \lp{\vbfi} &= - \frac{1}{\gammaF\gammaM} \pderiv[2]{\vbfi}{t}
\end{align*}
As mentioned, we obtain exactly the same equation we had for the electric field.
\[\frac{1}{\gammaF\gammaM} \pderiv[2]{\vbfi}{t} - \lp{\vbfi} = 0\]
%
%
\subsection{Faraday--Neumann--Lenz law}
%
Let us consider a closed loop built out of an electrical conductor, delimiting a
surface \(\Sigma\) and let us take the time derivative of the flux of the magnetic
field through this surface. The shape and position of the loop can change with time.
\[\D{\Phi_\vbfi}{t} = \D{}{t} \iint_{\Sigma\of{t}} \vbfi\of{\vec{\xi},t} \cdot \vec{n}\of{\vec{\xi},t} \de^2\xi\]
%
The rate of change depends both on the change in magnetic field and in the geometry
of the loop. We can resort to a form of the Reynolds transport theorem in
order to compute this derivative.
\[\D{\Phi_\vbfi}{t} = \iint_{\Sigma\of{t}} \of{\pderiv{\vbfi}{t} + \of{\dvg{\vbfi}}\vec{v}\of{\vec{\xi},t}} \cdot \vec{n}\of{\vec{\xi},t} \de^2\xi
- \oint_{\partial\Sigma(t)} \of{\vec{v}\of{\vec{\xi},t} \times \vbfi\of{\vec{\xi},t}} \cdot \vec{t}\of{\vec{\xi},t} \de\xi\]
We now use Gauss's law for magnetism to remove the term involving the divergence
of the magnetic field, which is zero.
\[\D{\Phi_\vbfi}{t} = \iint_{\Sigma\of{t}} \pderiv{\vbfi}{t} \cdot \vec{n}\of{\vec{\xi},t} \de^2\xi
- \oint_{\partial\Sigma(t)} \of{\vec{v}\of{\vec{\xi},t} \times \vbfi\of{\vec{\xi},t}} \cdot \vec{t}\of{\vec{\xi},t} \de\xi\]
We then use Faraday--Maxwell's law to replace the first term.
\[\D{\Phi_\vbfi}{t} = - \gammaF \oint_{\partial\Sigma\of{t}} \vefi\of{\vec{\xi},t} \cdot \vec{t}\of{\vec{\xi},t} \de\xi
- \oint_{\partial\Sigma(t)} \of{\vec{v}\of{\vec{\xi},t} \times \vbfi\of{\vec{\xi},t}} \cdot \vec{t}\of{\vec{\xi},t} \de\xi\]
Finally, we recognise the expressions for the induced and motional electromotive
forces.
\[\D{\Phi_\vbfi}{t} = - \gammaF \emf_\mathrm{ind} - \kappaf \, \emf_\mathrm{mot}\]
In order for the Faraday--Neumann--Lenz law to be valid, we need to be able to factor
this expression, so we obtain a third and last condition for our constants, along
with an expression giving us the value of Faraday's constant in terms of such constants.
\[\gammaF = \kappaf = k_\mathrm{F}\]
%
%
\subsection{Adding it all up}
%
We started out with six constants: \(\kappac, \kappaf, \alphaG, \alphaA, \gammaF\) and \(\gammaM\),
but we observed that requiring the consistency of Maxwell's equations with the
continuity equation for electric charges, with the fact that electromagnetic
waves propagate with a speed of \(c\) and with the Faraday--Neumann--Lenz law has
given us three equations that relate these constants to each other.
\begin{equation}
  \alphaG = \frac{c^2}{\kappaf \kappac} \alphaA \qquad \gammaM = \frac{c^2}{\kappaf} \qquad \gammaF = \kappaf
\end{equation}
We are left with only three constants which we can freely choose.
We rearrange the above equations in order to give us a system where we are free
to choose the couplings between the fields and the charges, in addition to the
constant \(\kappac\) that defines current.
\begin{equation}
  \kappaf = \gammaF = \frac{c^2}{\kappac} \frac{\alphaA}{\alphaG}
  \qquad \gammaM = \frac{\alphaG}{\alphaA} \kappac
\end{equation}
In such a system Maxwell's equations take the following form.
\begin{center}
  \begin{tabular}{ccccl}
    \(\mathrm{M}_\mathrm{I}\) & \textbf{Gauss's law (for \(\vefi\))} & \(\dvg{\vefi}\) & \(=\) & \(\alphaG \, \rho\) \\[1em]
    \(\mathrm{M}_\mathrm{II}\) & \textbf{Gauss's law (for \(\vbfi\))} & \(\dvg{\vbfi}\) & \(=\) & \(0\) \\[1em]
    \(\mathrm{M}_\mathrm{III}\) & \textbf{Faraday--Maxwell's law} & \(\crl{\vefi}\) & \(=\) & \(\displaystyle - \frac{1}{\kappaf} \pderiv{\vbfi}{t}\) \\[1em]
    \(\mathrm{M}_\mathrm{IV}\) & \textbf{Ampère--Maxwell's law} & \(\crl{\vbfi}\) & \(=\) & \(\displaystyle \alphaA \, \vec{j} + \frac{\kappaf}{c^2} \pderiv{\vefi}{t}\) \\
  \end{tabular}
\end{center}
%
%
\section{Coulomb's law}
%
We have already introduced Coulomb's law when we were discussing the possible
choices of units of charge, we now endeavour to find a connection between
Coulomb's constant \(k_\mathrm{C}\) and the constants we have used in Maxwell's
equations.\\[1em]
We consider an immobile point charge \(Q\) at the origin, for which \(\rho\of{\vec{r},t} = Q \, \delta\of{\vec{r}}\),
and integrate Gauss's law over a spherical volume \(\Omega\) with radius \(d\).
\[\iiint_\Omega \dvg{\vefi} \de^3\xi = \alphaG \iiint_\Omega Q \, \delta\of{\vec{\xi}} \de^3\xi
\qquad\Rightarrow\qquad
\oiint_{\partial\Omega} \vefi\of{\vec{\xi}} \cdot \vec{n}\of{\vec{\xi}} \de^2\xi = \alphaG \, Q\]
Since the system is spherically symmetrical we can assume that the electric field
shares such a property, this means that it must be directed radially and that its
magnitude only depensd upon the magnitude of \(\vec{\xi}\), giving us the following
expression for it.
\[\vefi\of{\vec{\xi}} = \efi_n\!\of{\xi} \, \vec{n}\of{\vec{\xi}}\]
We plug this into the previous expression and recall that the sphere has
radius \(d\), so that \(\xi\) is the same at all points
and we can extract \(\efi_n\) from the integral.
\[\oiint_{\partial\Omega} \efi_n\!\of{d} \de^2\xi = \efi_n\!\of{d} \oiint_{\partial\Omega} \de^2\xi = 4 \pi d^2 \, \efi_n\!\of{d}\]
Solving for the electric field gives us an inverse square law.
\[\efi_n\!\of{d} = \frac{\alphaG}{4\pi} \frac{Q}{d^2}\]
Assuming we have a second charge \(Q_2\) at a distance \(d\) from the first, which
we shall now denote by \(Q_1\), the resulting force in the radial direction can
be computed as the electric force produced by this field.
\[F_n\!\of{d} = Q_2 \, \efi_n\!\of{d} = \frac{\alphaG}{4\pi} \frac{Q_1\,Q_2}{d^2}\]
Taking the magnitude of this force gives us Coulomb's law and therefore Coulomb's
constant, which only depends upon \(\alphaG\).
\[F\of{d} = \frac{\alphaG}{4\pi} \frac{\abs{Q_1}\abs{Q_2}}{d^2} \qquad k_\mathrm{C} = \frac{\alphaG}{4\pi}\]
%
%
\section{Ampère's force law}
%
In a similar manner to what we did for Coulomb's law, we now attempt to find a connection
between Ampère's force constant \(k_\mathrm{A}\) and the constants we've been using
in Maxwell's equations.\\[1em]
We consider an infinitely long straight wire carrying a current \(I\), passing
through the origin and going in the direction \(z\). The wire can be parametrised
\(\vec{\gamma}\of{s} = \of{0,0,s}\), the tangent to which is simply \(\uvec{z}\).
The current density for such a system is as follows.
\[\vec{j}\of{\vec{r},t} = I \delta\of{x} \delta\of{y} \uvec{z} \int_{\mathbb{R}} \delta\of{z - s} \de s = I \delta\of{x} \delta\of{y} \uvec{z}\]
We take the flux of Ampère-Maxwell's law over a circle \(\Sigma\) with radius \(d\)
lying in the \(\of{x,y}\) plane and thus having the vector \(\uvec{z}\) as a normal.
\[\iint_\Sigma \of{\crl{\vbfi}} \cdot \uvec{z} \, \de^2\xi = \alphaA \iint_\Sigma \of{I \delta\of{x}\delta\of{y}} \uvec{z} \cdot \uvec{z} \, \de^2\xi
\qquad\Rightarrow\qquad
\oint_{\partial\Sigma} \vbfi \cdot \uvec{\phi} \, \de\xi = \alphaA I\]
This system has cylindrical symmetry, so we can expect the magnetic field to also
have such a property. Therefore it should be directed along the tangent to the circle
at each point and have a magnitude dependent only upon the distance from the origin.
\[\vbfi\of{\vec{\xi}} = \bfi_\phi\!\of{\xi} \, \uvec{\phi}\of{\vec{\xi}}\]
We plug this into the previous expression and recall that the circle has
radius \(d\), so that \(\xi\) is the same at all points
and we can extract \(\bfi_\phi\) from the integral.
\[\oint_{\partial\Sigma} \bfi_\phi\of{d} \de\xi = \bfi_\phi\of{d} \oint_{\partial\Sigma} \de\xi = 2 \pi d \, \bfi_\phi\of{d}\]
Solving for the magnetic field gives us the \textbf{Biot--Savart law}.
\[\bfi_\phi\of{d} = \frac{\alpha_A}{2 \pi} \frac{I}{d}\]
We now take a second wire carrying a current \(I_2\), parallel to the original wire
and located at a point \(\of{x_0,y_0}\) such that its distance from the first wire
is \(d\), that is to say that \({x_0}^2 + {y_0}^2 = d^2\).
The current density for this wire is \(\vec{j_2}\of{\vec{r},t} = I_2 \delta\of{x-x_0}\delta\of{y-y_0} \uvec{z}\).
We also denote the current on the first wire by \(I_1\).\\[1em]
We should expect the force acting between the two wires to be infinite, since they
are indefinitely long, to get around this we must consider the force per unit lenght,
which we can compute by considering the expression for the magnetic force density,
integrating it over a region of space spanning the whole \(\of{x,y}\) plane and
between \(0\) and \(L\) in the \(z\) direction, then dividing this result by \(L\).
\[\frac{\vec{F}}{L} = \frac{1}{L} \int_0^L \of{\iint_{\mathbb{R}^2} \vec{f}_\mathrm{m} \de x \, \de y} \de z
 = \frac{1}{L} \int_0^L \of{\iint_{\mathbb{R}^2} \frac{\kappac}{\kappaf} \vec{j}_2 \times \vbfi \, \de x \, \de y} \de z\]
Replacing the expressions relating to our case we find the following.
\[\frac{\vec{F}}{L} = \frac{\kappac}{\kappaf} \frac{\alpha_A}{2 \pi} \frac{1}{L} \int_0^L \de z
\iint_{\mathbb{R}^2} \frac{I_1 \, I_2 \of{\uvec{\phi} \times \uvec{z}}}{\sqrt{x^2 - y^2}} \delta\of{x-x_0}\delta\of{y-y_0}\de x \de y\]
Solving the integral on \(z\) simply gives us \(L\), then by using the property of the Dirac's delta,
replacing \(\sqrt{{x_0}^2 + {y_0}^2}\) with \(d\) and recalling that \(\uvec{\phi} \times \uvec{z} = - \uvec{r}\)
we obtain the following expression for the force per unit lenght.
\[\frac{\vec{F}}{L} = - \frac{\kappac}{\kappaf} \frac{\alpha_A}{2 \pi} \frac{I_1 \, I_2}{d} \, \uvec{r}\]
Taking its magnitude we find Ampère's force law and therefore find the relation between
Ampère's force constant and \(\alphaA\), \(\kappac\) and \(\kappaf\).
\[\frac{F}{L} = 2 \frac{\kappac}{\kappaf} \frac{\alpha_A}{4 \pi} \frac{\abs{I_1} \abs{I_2}}{d} \qquad k_\mathrm{A} = \frac{\kappac}{\kappaf} \frac{\alpha_A}{4 \pi}\]
We notice that, unlike Coulomb's constant, Ampère's force constant does not depend
solely upon the coupling constant in the relevant Maxwell's equation, but also
on the scaling constants for the currents and fields.
%
%
\section{Poynting's theorem}
%
We consider \(\mathrm{M}_\mathrm{III}\) and \(\mathrm{M}_\mathrm{IV}\) and take
their scalar product with, respectively, \(\vbfi\) and \(\vefi\).
\[\vbfi \cdot \of{\crl{\vefi}} = - \frac{\vbfi}{\gammaF} \cdot \pderiv{\vbfi}{t}
\qquad
\vefi \cdot \of{\crl{\vbfi}} = \alphaA \, \vefi \cdot \vec{j} + \frac{\vefi}{\gammaM} \cdot \pderiv{\vefi}{t}\]
We then subtract the second equation from the first.
\[\vefi \cdot \of{\crl{\vbfi}} - \vbfi \cdot \of{\crl{\vefi}} = \alphaA \, \vefi \cdot \vec{j} + \frac{\vefi}{\gammaM} \cdot \pderiv{\vefi}{t} + \frac{\vbfi}{\gammaF} \cdot \pderiv{\vbfi}{t}\]
The left hand side of this equation is the divergence of \(\vefi \times \vbfi\).
\[- \dvg{\of{\vefi \times \vbfi}} = \alphaA \, \vefi \cdot \vec{j} + \frac{\vefi}{\gammaM} \cdot \pderiv{\vefi}{t} + \frac{\vbfi}{\gammaF} \cdot \pderiv{\vbfi}{t}\]
We now isolate the term containing \(\vefi \cdot \vec{j}\).
\begin{align*}
- \vefi \cdot \vec{j} &= \dvg{\of{\frac{\vefi \times \vbfi}{\alphaA}}} +
\frac{\vefi}{\alphaA \, \gammaM} \cdot \pderiv{\vefi}{t} + \frac{\vbfi}{\alphaA \, \gammaF} \cdot \pderiv{\vbfi}{t}\\
- \vefi \cdot \vec{j} &= \dvg{\of{\frac{\vefi \times \vbfi}{\alphaA}}} +
\frac{\vefi}{\alphaG \, \kappac} \cdot \pderiv{\vefi}{t} + \frac{\vbfi}{\alphaA \, \kappaf} \cdot \pderiv{\vbfi}{t}
\end{align*}
We then use the identity for the derivative of a product to rewrite the second and
third terms in the right hand side.
\begin{align*}
- \vefi \cdot \vec{j} &= \dvg{\of{\frac{\vefi \times \vbfi}{\alphaA}}} +
\frac{1}{2} \pderiv{}{t} \of{\frac{\vefi \cdot \vefi}{\kappac\alphaG} + \frac{\vbfi \cdot \vbfi}{\kappaf\alphaA}}\\
- \vefi \cdot \vec{j} &= \dvg{\of{\frac{\vefi \times \vbfi}{\alphaA}}} +
\frac{1}{2\kappac} \pderiv{}{t} \of{\frac{\vefi \cdot \vefi}{\alphaG} + \frac{\kappac}{\kappaf}\frac{\vbfi \cdot \vbfi}{\alphaA}}
\end{align*}
The left hand side is related to the power density dissipated due to the Joule--Lenz
effect; this suggests that the term in the derivative is an \textbf{energy density},
and the vector of which we are taking the divergence is its flux, called the
\textbf{Poynting vector} \(\vpoy\).
\[\vpoy = \frac{\vefi \times \vbfi}{\alphaA} \qquad u = \frac{1}{2} \of{\frac{\vefi \cdot \vefi}{\alphaG} + \frac{\kappac}{\kappaf}\frac{\vbfi \cdot \vbfi}{\alphaA}}\]
In these terms, Poynting's theorem states the following.
\[\frac{1}{\kappac} \D{u}{t} = \frac{1}{\kappac} \pderiv{u}{t} + \dvg{\vpoy}\]
%
%
\section{Electromagnetic Radiation}
%
We have already shown that the solution to Maxwell's equations in the absence of
charges and currents gives us d'Alembert's equation.
\[\frac{1}{c^2} \pderiv[2]{\vefi}{t} - \lp{\vefi} = 0 \qquad \frac{1}{c^2} \pderiv[2]{\vbfi}{t} - \lp{\vbfi} = 0\]
We know that a possible set of solutions to d'Alembert's equation are plane waves, we
take \(\omega\) to be the pulsatance and \(\vec{k}\) to be the (angular) wave vector of the solution.
\[\vefi\of{\vec{r},t} = \vefi_0 \, \exp^{\iun \vec{k} \cdot \vec{r}} \exp^{-\iun \omega t} \qquad \vbfi\of{\vec{r},t} = \vbfi_0 \, \exp^{\iun \vec{k} \cdot \vec{r}} \exp^{-\iun \omega t}\]
%We compute the Laplacian and the second derivative of the electric field.
%\[\lp{\vefi} = - k^2 \vefi \qquad \pderiv[2]{\vefi}{t} = - \omega^2 \vefi\]
%
%
\subsection{Dispersion relation}
%
Substituting the plane waves into d'Alembert's equation gives us the \emph{dispersion relation}.
\[\frac{\omega^2}{c^2} \vefi - k^2 \vefi = 0 \qquad \omega = c \, k\]
%
%
\subsection{Orientation of the fields}
%
Substituting the plane waves into \(\mathrm{M}_\mathrm{I}\) and \(\mathrm{M}_\mathrm{II}\)
gives us the following conditions.
\[\vec{k} \cdot \vefi = 0 \qquad \vec{k} \cdot \vbfi = 0\]
This shows that the electric and magnetic fields are \emph{transverse}, i.e. they are perpendicular to the direction
of propagation of the wave.\\
Substituting instead into \(\mathrm{M}_\mathrm{III}\) and \(\mathrm{M}_\mathrm{IV}\)
gives us the following conditions.
\[\vec{k} \times \vefi = \frac{\omega}{\kappaf} \vbfi \qquad \vec{k} \times \vbfi = - \frac{\kappaf \, \omega}{c^2} \vefi\]
Through the dispersion relation, these can be rewritten as follows.
\[\vbfi = \frac{\kappaf}{c} \uvec{k} \times \vefi \qquad \vefi = - \frac{c}{\kappaf} \uvec{k} \times \vbfi\]
This shows that the electric and magnetic field are perpendicular to each other.
%Taking the norm of the second equation, recalling the identity for the norm of
%a cross product and using the dispersion relation, we find the relation between the
%magnitudes of the electric and magnetic fields in free space.
%\[\nrm{\vec{k} \times \vefi} = \sqrt{k^2 \efi^2 - \of{\vec{k}\cdot\vefi}^2} = k \efi\]
%\[k \efi = \frac{\omega}{\kappaf} \bfi \qquad \efi = \frac{c}{\kappaf} \bfi\]
%
%
\subsection{Impedance of free space}
%
We can also compute the Poynting vector of the wave.
\[\vpoy = \frac{\vefi \times \vbfi}{\alphaA} =
\frac{\kappaf}{c} \frac{1}{\alphaA} \vefi \times \of{\uvec{k} \times \vefi} =
\frac{\kappaf}{c} \frac{\efi^2}{\alphaA} \uvec{k}
\]
We take notice of the fact that it is in the same direction as the wave vector,
i.e. in the direction of wave propagation.
Taking just its norm we can write the following.
\[\poy = \frac{\efi^2}{Z_0} \qquad Z_0 = \frac{c}{\kappaf} \alphaA = \frac{\kappac}{c} \alphaG\]
Where \(Z_0\) is the \textbf{impedance of free space}.
Another convention is sometimes in use, where one naïvely extends the
impedance of free space found in the SI to another unit system. This does not, in
general, have the correct units to be called an impedance and doesn’t, of course, give
the correct ratio between the Poynting vector and the electric field. We shall denote
this by \(\widetilde{Z}_0\).
\[\widetilde{Z}_0 = \sqrt{\alphaG \, \alphaA} = \frac{c}{\sqrt{\kappac \kappaf}} \, \alphaA = \frac{\sqrt{\kappac \kappaf}}{c} \, \alphaG\]
We see that this definition is identical to \(Z_0\) only when \(\kappac = \kappaf\).
%
%
\section{Maxwell's stress tensor}
%
We start with the expression of the Lorentz force density and replace the charge
and current densities with expressions resulting from Maxwell's equations.
\[\vec{f} = \rho \vefi + \frac{\kappac}{\kappaf} \vec{j} \times \vbfi = \frac{\dvg{\vefi}}{\alphaG}\vefi + \frac{\kappac}{\kappaf} \of{\frac{\crl\vbfi}{\alphaA} \times \vbfi
  - \frac{1}{\alphaA \gammaM} \pderiv{\vefi}{t} \times \vbfi}\]
We can then replace the last term with its expression in terms of \(\vefi \times \vbfi\).
\[\vec{f} = \frac{\dvg{\vefi}}{\alphaG}\vefi + \frac{\kappac}{\kappaf} \of{\frac{\crl\vbfi}{\alphaA} \times \vbfi
  - \frac{1}{\alphaA \gammaM} \pderiv{}{t} \of{\vefi \times \vbfi} + \frac{1}{\alphaA\gammaM} \vefi \times \pderiv{\vbfi}{t}}\]
Then we can use Faraday--Maxwell's equation to replace the derivative of \(\vbfi\).
\[\vec{f} = \frac{\dvg{\vefi}}{\alphaG}\vefi + \frac{\kappac}{\kappaf} \of{\frac{\crl\vbfi}{\alphaA} \times \vbfi
  - \frac{1}{\alphaA \gammaM} \pderiv{}{t} \of{\vefi \times \vbfi} - \frac{\gammaF}{\alphaA\gammaM} \vefi \times \of{\crl{\vefi}}}\]
\[\vec{f} = \frac{\dvg{\vefi}}{\alphaG}\vefi + \frac{\kappac}{\kappaf} \of{- \frac{\vbfi \times \of{\crl{\vbfi}}}{\alphaA}
  - \frac{1}{\gammaM} \pderiv{}{t} \of{\frac{\vefi \times \vbfi}{\alphaA}} - \frac{\kappaf}{\kappac} \frac{1}{\alphaG} \vefi \times \of{\crl{\vefi}}}\]
\[\vec{f} + \frac{\kappac}{\gammaM \kappaf} \pderiv{}{t} \of{\frac{\vefi \times \vbfi}{\alphaA}} = \frac{\vefi \of{\dvg{\vefi}} - \vefi \times \of{\crl{\vefi}}}{\alphaG}
  - \frac{\kappac}{\kappaf} \frac{\vbfi \times \of{\crl{\vbfi}}}{\alphaA}\]
We notice that a term is missing to achieve symmetry, this term would contain the divergence
of \(\vbfi\), which is zero, so it can be freely added to the expression.
\[\vec{f} + \frac{\kappac}{c^2} \pderiv{}{t} \of{\frac{\vefi \times \vbfi}{\alphaA}} = \frac{\vefi \of{\dvg{\vefi}} - \vefi \times \of{\crl{\vefi}}}{\alphaG}
  + \frac{\kappac}{\kappaf} \frac{\vbfi \of{\dvg{\vbfi}} - \vbfi \times \of{\crl{\vbfi}}}{\alphaA}\]
We recognise the Poynting vector and use \(\vec{a} \times \of{\crl{\vec{a}}} = \frac{1}{2} \grd{\of{\vec{a}\cdot\vec{a}}} - \of{\vec{a} \cdot \grd{}}\vec{a}\).
\[\vec{f} + \frac{\kappac}{c^2} \pderiv{\vpoy}{t} =
  \frac{\vefi \of{\dvg{\vefi}} + \of{\vefi \cdot \grd{}} \vefi - \frac{1}{2} \grd{\of{\vefi\cdot\vefi}}}{\alphaG}
  + \frac{\kappac}{\kappaf} \frac{\vbfi \of{\dvg{\vbfi}} + \of{\vbfi \cdot \grd{}} \vbfi - \frac{1}{2} \grd{\of{\vbfi\cdot\vbfi}}}{\alphaA}\]
We use the identity \(\dvg{\of{\vec{a} \otimes \vec{a}}} = \vec{a} \of{\dvg{\vec{a}}} + \of{\vec{a} \cdot \grd{}}\vec{a}\).
\[\vec{f} + \frac{\kappac}{c^2} \pderiv{\vpoy}{t} =
  \frac{\dvg{\of{\vefi\otimes\vefi}} - \frac{1}{2} \grd{\of{\vefi\cdot\vefi}}}{\alphaG}
  + \frac{\kappac}{\kappaf} \frac{\dvg{\of{\vbfi\otimes\vbfi}} - \frac{1}{2} \grd{\of{\vbfi\cdot\vbfi}}}{\alphaA}\]
By recalling that we can write the gradient of a scalar function as the divergence
of the same function multiplied by the identity tensor \(\vec{1}\), we can write the right hand
side of this expression as the divergence of a tensor called the \textbf{Maxwell stress tensor}.
\[\vmxw = \of{\frac{\vefi\otimes\vefi}{\alphaG} + \frac{\kappac}{\kappaf} \frac{\vbfi\otimes\vbfi}{\alphaA}}
  - \frac{\vec{1}}{2} \of{\frac{\vefi \cdot \vefi}{\alphaG} + \frac{\kappac}{\kappaf}\frac{\vbfi \cdot \vbfi}{\alphaA}}\]
We obtain the following expression, which expresses conservation of momentum (recall
that force is the time derivative of momentum).
\[\vec{f} + \frac{\kappac}{c^2} \pderiv{\vpoy}{t} = \dvg{\vmxw}\]
Looking at the expression for \(\vmxw\), we notice that it can be written as a tensor
minus half its trace. Moreover, the trace part is exactly the energy density for
the electromagnetic field which we found in Poynting's theorem, for this reason
we denote this tensor by \(\vec{u}\) and define it by including the \(\frac{1}{2}\)
factor, so the trace is exactly the energy density.
\[\vec{u} = \frac{1}{2} \of{\frac{\vefi\otimes\vefi}{\alphaG} + \frac{\kappac}{\kappaf} \frac{\vbfi\otimes\vbfi}{\alphaA}}
\qquad \trace{\vec{u}} = u\]
In terms of this new tensor, the Maxwell stress tensor can be written as follows,
making it evident that its trace is, once again, the energy density.
\[\vmxw = 2 \, \vec{u} - \of{\trace{\vec{u}}} \vec{1} \qquad \trace{\vmxw} = u\]
%
%
\section{Potentials}
%
From \(\mathrm{M}_\mathrm{II}\) we can say that the magnetic field is solenoidal,
which means that it can be written as the curl of a \textbf{vector potential} \(\vafi\).
\[\vbfi = \crl{\vafi}\]
We substitute this expression for \(\vbfi\) into \(\mathrm{M}_\mathrm{III}\).
\[\crl{\vefi} = - \frac{1}{\gammaF} \pderiv{}{t} \crl{\vafi} \qquad \crl{\of{\vefi + \frac{1}{\kappaf} \pderiv{\vafi}{t}}} = 0\]
This shows that the combination of the electric field with the time derivative of
the vector potential is irrotational, meaning that it admits a \textbf{scalar potential} \(\scp\).
\[\vefi = - \grd{\scp} - \frac{1}{\kappaf} \pderiv{\vafi}{t}\]
Notice how the fields depend only upon the derivatives of the potentials, due to
this if we are given a pair of potentials \(\scp\) and \(\vafi\) we can define
a new pair \(\scp'\) and \(\vafi'\) producing the same fields. The easiest example
of this would be simply adding a constant to each potential, but the most general
case involves adding a field \(\Lambda\of{\vec{r},t}\) as follows.
\[\vafi'\of{\vec{r},t} = \vafi\of{\vec{r},t} + \grd\Lambda\of{\vec{r},t}
\qquad \scp'\of{\vec{r},t} = \scp\of{\vec{r},t} + \frac{1}{\kappaf} \pderiv{\Lambda}{t}\]
It is important to remark that the transformation must be performed on both fields
at the same time. This phenomenon is known as \textbf{gauge invariance}.
The ambiguity in the definition of the fields can be solved, or at least assuaged,
by requiring an additional condition that the potentials must satisfy, this is
called \textbf{gauge fixing}.\\[1em]
The most used gauge choices in physics are the \textbf{Coulomb gauge}
and the \textbf{Lorenz gauge}.
\[\text{Coulomb:  } \dvg{\vafi} = 0 \qquad\qquad \text{Lorenz:  } \frac{\kappac}{c^2} \pderiv{\scp}{t} + \dvg{\vafi} = 0\]
Substituting the fields expressed in terms of the potential into Maxwell's equations
gives us equations for the potentials themselves, two of the equations, namely
\(\mathrm{M}_\mathrm{II}\) and \(\mathrm{M}_\mathrm{III}\), are already satisfied by
virtue of how we defined the potentials. We start by considering \(\mathrm{M}_\mathrm{I}\),
from which we obtain the following expression.
\[\dvg{\vefi} = \alphaG \, \rho \qquad - \lp \phi - \frac{1}{\kappaf} \pderiv{}{t} \dvg{\vafi} = \alphaG \, \rho\]
We now consider \(\mathrm{M}_\mathrm{IV}\), which requires some further manipulation.
\[\crl{\vbfi} = \alphaA \, \vec{j} + \frac{1}{\gammaM} \pderiv{\vefi}{t}
\qquad \crl{\of{\crl{\vafi}}} = \alphaA \, \vec{j} - \frac{1}{\gammaM} \pderiv{}{t} \grd{\scp} - \frac{1}{\gammaM\kappaf} \pderiv[2]{\vafi}{t}\]
We recall that \(\crl{\of{\crl{\vec{a}}}} = \grd{\of{\dvg{\vec{a}}}} - \lp{\vec{a}}\)
and that \(\gammaM\kappaf = \gammaM\gammaF = c^2\).
\[\grd{\of{\dvg{\vafi}}} - \lp{\vafi} = \alphaA \, \vec{j} - \frac{1}{\gammaM} \pderiv{}{t} \grd{\scp} - \frac{1}{c^2} \pderiv[2]{\vafi}{t}\]
\[\frac{1}{c^2} \pderiv[2]{\vafi}{t} - \lp{\vafi} + \grd{\of{\dvg{\vafi} + \frac{\kappaf}{c^2} \pderiv{\scp}{t} }} = \alphaA \, \vec{j}\]
We thus have a system of coupled differential equations, which is not trivial to
solve. With the correct gauge choice, though, the equations become simpler.
%
%
\subsection{Static potentials}
%
\(\mathrm{M}_\mathrm{I}\)
seems to suggest the Coulomb gauge, producing Poisson's equation.
\[\lp{\scp} = - \alphaG \, \rho\]
In this gauge \(\mathrm{M}_\mathrm{IV}\) is not particularly simple. We obtain
d'Alembert's equation with a source term depending on \(\scp\) (which, however,
we can find indepedently of \(\vafi\) from the previous equation).
\[\dal{\vafi} = \alphaA \, \vec{j} - \frac{\kappaf}{c^2} \pderiv{}{t} \grd{\scp}\]
Things become a lot simpler in the static case, that is when the system is independent
of time and therefore all time derivatives vanish. We obtain Poisson's equation.
\[\lp{\vafi} = - \alphaA \, \vec{j}\]
Using the Green function for the Laplacian operator we find the \textbf{static potentials}.
\[
  \scp\of{\vec{r}} = \frac{\alphaG}{4 \pi} \iiint_\Omega \frac{\rho\of{\vec{\xi}}}{\nrm{\vec{r}-\vec{\xi}}} \de^3\xi
  \qquad
  \vafi\of{\vec{r}} = \frac{\alphaA}{4 \pi} \iiint_\Omega \frac{\vec{j}\of{\vec{\xi}}}{\nrm{\vec{r}-\vec{\xi}}} \de^3\xi
\]
%
%
\subsection{Retarded potentials}
%
A different way to simplify \(\mathrm{M}_\mathrm{IV}\) is to use the Lorenz gauge,
so that the whole term within the gradient vanishes. Doing so we obtain d'Alembert's equation.
\[\dal{\vafi} = \alphaA \, \vec{j}\]
In this gauge the two equations are fully decoupled, indeed we obtain d'Alembert's
equation for the scalar potential as well.
\[\dal{\scp} = \alphaG \, \rho\]
Using the Green function method we find the \textbf{retarded potentials}.
\[
  \scp\of{\vec{r},t} = \frac{\alphaG}{4 \pi} \iiint_\Omega \frac{\rho\of{\vec{\xi},t_\mathrm{r}}}{\nrm{\vec{r}-\vec{\xi}}} \de^3\xi
  \qquad
  \vafi\of{\vec{r},t} = \frac{\alphaA}{4 \pi} \iiint_\Omega \frac{\vec{j}\of{\vec{\xi},t_\mathrm{r}}}{\nrm{\vec{r}-\vec{\xi}}} \de^3\xi
\]
The \textbf{retarded time} \(t_\mathrm{r}\) used in the expression accounts exactly for the
time it takes for light to travel from the point \(\vec{r}\) to the point \(\vec{\xi}\)
and is defined as follows.
\[t_\mathrm{r} = t - \frac{c}{\nrm{\vec{r}-\vec{\xi}}}\]
So the potentials in the Lorenz gauge properly satisfy causality.
Another advantage of the Lorenz gauge is that it can be written in a manifestly
covariant manner, making it very useful for relativistic purposes.
%
%
\section{Maxwell's equations for macroscopic fields}
%
Maxwell's equations depend upon all of the charges and currents present in a physical
system; if we study the behaviour of the electric and magnetic field given by
Maxwell's equation in some medium, we have contributions coming from the nuclei
and electrons of each atom in the medium.
We obtain two fields \(\vefi_\mathrm{loc}\) and \(\vbfi_\mathrm{loc}\) are called the
\textbf{microscopic} or \textbf{local fields} because they explicitly depend upon
these microscopic phenomena, which are generally not easy to treat.\\[1em]
We can simplify the problem by averaging the local fields over a volume \(\Delta \Omega\),
that must be chosen so that the average is statistically significant, but at the
same time it must be small enough to still allow us to treat the system as
a continuum. Doing so we obtain the \textbf{macroscopic fields} \(\vefi\) and \(\vbfi\).
\[\vefi\of{\vec{r},t} = \frac{1}{\Delta \Omega} \iiint_{\Delta \Omega} \vefi_\mathrm{loc}\of{\vec{r}+\vec{\xi},t}\de^3\xi\]
\[\vbfi\of{\vec{r},t} = \frac{1}{\Delta \Omega} \iiint_{\Delta \Omega} \vbfi_\mathrm{loc}\of{\vec{r}+\vec{\xi},t}\de^3\xi\]
Despite this approximation, several contributions still find their way into the
source terms in Maxwell's equations.
The total charge and current densities can be split into a free and a bound term.
\[\rho = \rho_\mathrm{f} + \rho_\mathrm{b} \qquad \vec{j} = \vec{j}_\mathrm{f} + \vec{j}_\mathrm{b}\]
Bound charges or currents are phenomena that generally happen as a result of
the interaction of a physical system with the electric and magnetic field, such
as polarisation or magnetisation.
The former produces both a charge and a current density, while the latter only
produces a current density.
\[\rho_\mathrm{b} = \rho_\mathrm{p} \qquad \vec{j}_\mathrm{b} = \vec{j}_\mathrm{p} + \vec{j}_\mathrm{m}\]
The polarisation terms can be represented by a \textbf{polarisation field} \(\vpfi\).
\[\rho_\mathrm{p} = - \dvg{\vpfi} \qquad \vec{j}_\mathrm{p} = \frac{1}{\kappac} \pderiv{\vpfi}{t}\]
The magnetisation term can be represented by a \textbf{magnetisation field} \(\vmfi\).
\[\vec{j}_\mathrm{m} = \frac{\kappaf}{\kappac} \crl{\vmfi}\]
The polarisation and magnetisation fields are defined in such a way that the
corresponding densities each respect a continuity equation.
In the case of the magnetisation current, no charge density is necessary, since the
divergence of a curl is zero.\\[1em]
We rewrite Maxwell's equations, explicitly separating the free and bound terms.
\begin{center}
  \begin{tabular}{cccl}
    \(\mathrm{M}_\mathrm{I}\) & \(\dvg{\vefi}\) & \(=\) & \(\alphaG \of{\rho_\mathrm{f} + \rho_\mathrm{p}}\) \\[1em]
    \(\mathrm{M}_\mathrm{II}\) & \(\dvg{\vbfi}\) & \(=\) & \(0\) \\[1em]
    \(\mathrm{M}_\mathrm{III}\) & \(\crl{\vefi}\) & \(=\) & \(\displaystyle - \frac{1}{\gammaF} \pderiv{\vbfi}{t}\) \\[1em]
    \(\mathrm{M}_\mathrm{IV}\) & \(\crl{\vbfi}\) & \(=\) & \(\displaystyle \alphaA \of{\vec{j}_\mathrm{f} + \vec{j}_\mathrm{p} + \vec{j}_\mathrm{m}} + \frac{1}{\gammaM} \pderiv{\vefi}{t}\) \\
  \end{tabular}
\end{center}
We focus on Gauss's law for electricity.
\begin{gather*}
  \dvg{\vefi} = \alphaG \of{\rho_\mathrm{f} + \rho_\mathrm{p}} \\
  \dvg{\vefi} - \alphaG \, \rho_\mathrm{p} = \alphaG \, \rho_\mathrm{f} \\
  \dvg{\vefi} + \alphaG \, \dvg{\vpfi} = \alphaG \, \rho_\mathrm{f} \\
  \dvg{\of{\vefi + \alphaG \, \vpfi}} = \alphaG \, \rho_\mathrm{f}
\end{gather*}
We have two choices on how to define the \textbf{electric displacement field} \(\vdfi\).
\begin{equation}
  \vdfi_1 = \vefi + \alphaG \, \vpfi \qquad \vdfi_2 = \frac{\vefi}{\alphaG} + \vpfi
\end{equation}
They produce, respectively, the following equations.
\begin{equation}
  \dvg{\vdfi_1} = \alphaG \, \rho_\mathrm{f} \qquad \dvg{\vdfi_2} = \rho_\mathrm{f}
\end{equation}
We now take Ampère--Maxwell's law.
\begin{gather*}
  \crl{\vbfi} = \alphaA \of{\vec{j}_\mathrm{f} + \vec{j}_\mathrm{p} + \vec{j}_\mathrm{m}} + \frac{1}{\gammaM} \pderiv{\vefi}{t} \\
  \crl{\vbfi} - \alphaA \vec{j}_\mathrm{m} = \alphaA \vec{j}_\mathrm{f} + \frac{1}{\gammaM} \pderiv{\vefi}{t} + \alphaA \vec{j}_\mathrm{p} \\
  \crl{\vbfi} - \frac{\kappaf}{\kappac} \alphaA \crl{\vmfi} = \alphaA \vec{j}_\mathrm{f} + \frac{1}{\gammaM} \pderiv{\vefi}{t} + \frac{\alphaA}{\kappac} \pderiv{\vpfi}{t} \\
  \crl{\of{\vbfi - \frac{\kappaf}{\kappac} \alphaA \vmfi}} = \alphaA \vec{j}_\mathrm{f} + \frac{1}{\gammaM} \pderiv{\vefi}{t} + \frac{\alphaG}{\gammaM} \pderiv{\vpfi}{t} \\
  \crl{\of{\vbfi - \frac{\kappaf}{\kappac} \alphaA \vmfi}} = \alphaA \vec{j}_\mathrm{f} + \frac{1}{\gammaM} \pderiv{}{t} \of{\vefi + \alphaG \pderiv{\vpfi}{t}}
\end{gather*}
We have two choices on how to define the \textbf{magnetising field} \(\vhfi\).
\begin{equation}
  \vhfi_1 = \vbfi - \frac{\kappaf}{\kappac} \alphaA \vmfi \qquad \vhfi_2 = \frac{\kappac}{\kappaf} \frac{\vbfi}{\alphaA} - \vmfi
\end{equation}
They produce, respectively, the following equations.
\begin{equation}
  \crl{\vhfi_1} = \alphaA \, \vec{j}_\mathrm{f} + \frac{1}{\gammaM} \pderiv{\vdfi_1}{t} \qquad \crl{\vhfi_2} = \frac{\kappac}{\kappaf} \, \vec{j}_\mathrm{f} + \frac{1}{\kappaf} \pderiv{\vdfi_2}{t}
\end{equation}
%
%
\subsection{Auxiliary fields}
%
We have defined two \textbf{auxiliary fields} to aid in our description of
electromagnetic phenomena in matter, these are \(\vdfi\) and \(\vhfi\).
We can take two different approaches in the way we define these fields, one is
aimed at maintaining the appearance of Maxwell's equations substantially unchanged,
while the other is aimed at getting rid of the coupling constants with the sources
by including them in the definition of the auxiliary field.
For the sake of brevity, we define the following constants.
\[\lambda_\mathrm{e} = \alphaG \qquad \lambda_\mathrm{m} = \alphaA \frac{\kappaf}{\kappac}\]
We call the fields in the first approach \textbf{non-rescaled auxiliary fields}
because in a vacuum (absence of polarisation/magnetisation) we have \(\vdfi_1 = \vefi\)
and \(\vhfi_1 = \vbfi\).
\[\vdfi_1 = \vefi + \lambda_\mathrm{e} \, \vpfi \qquad \vhfi_1 = \vbfi - \lambda_\mathrm{m} \, \vmfi\]
In contrast to this, we call the fields in the first approach \textbf{rescaled auxiliary fields}
because in a vacuum we have \(\vdfi_2 = \frac{1}{\lambda_\mathrm{e}} \vefi\)
and \(\vhfi_2 = \frac{1}{\lambda_\mathrm{m}} \vbfi\).
\[\vdfi_2 = \frac{1}{\lambda_\mathrm{e}} \vefi + \vpfi \qquad \vhfi_2 = \frac{1}{\lambda_\mathrm{m}} \vbfi - \vmfi\]
Both approaches are equally legitimate, and each provides a set of Maxwell's
equations in which only the free charges and currents appear.
\begin{center}
  \begin{tabular}{c|ccl|ccl}
    \toprule
    & \multicolumn{3}{c|}{\textbf{Non-rescaled}} & \multicolumn{3}{c}{\textbf{Rescaled}} \\
    \midrule
    \(\mathrm{M}_\mathrm{I}\) & \(\dvg{\vdfi_1}\) & \(=\) & \(\alphaG \rho_\mathrm{f}\) & \(\dvg{\vdfi_2}\) & \(=\) & \(\rho_\mathrm{f}\) \\[1em]
    \(\mathrm{M}_\mathrm{II}\) & \(\dvg{\vbfi}\) & \(=\) & \(0\) & \(\dvg{\vbfi}\) & \(=\) & \(0\) \\[1em]
    \(\mathrm{M}_\mathrm{III}\) & \(\crl{\vefi}\) & \(=\) & \(\displaystyle - \frac{1}{\gammaF} \pderiv{\vbfi}{t}\) & \(\crl{\vefi}\) & \(=\) & \(\displaystyle - \frac{1}{\kappaf} \pderiv{\vbfi}{t}\) \\[1em]
    \(\mathrm{M}_\mathrm{IV}\) & \(\crl{\vhfi_1}\) & \(=\) & \(\displaystyle \alphaA \vec{j}_\mathrm{f} + \frac{1}{\gammaM} \pderiv{\vdfi_1}{t}\) & \(\crl{\vhfi_2}\) & \(=\) & \(\displaystyle \frac{\kappac}{\kappaf} \, \vec{j}_\mathrm{f} + \frac{1}{\kappaf} \pderiv{\vdfi_2}{t}\) \\
    \bottomrule
  \end{tabular}
\end{center}
Equations \(\mathrm{M}_\mathrm{II}\) and \(\mathrm{M}_\mathrm{III}\) do not change,
however in the rescaling approach we have written \(\kappaf\) instead
of \(\gammaF\) to make it apparent that the equations do not depend upon
physical coupling constants, but only upon the constants used in the definition of the
magnetic field and of current (\(\kappac\) and \(\kappaf\)).\\[1em]
It can be convenient to summarise the two approaches into a single expression
involving constants \(\kappa_\mathrm{e}\) and \(\kappa_\mathrm{m}\) as follows.
\[\vdfi = \frac{1}{\kappa_\mathrm{e}} \of{\vefi + \lambda_\mathrm{e} \vpfi} \qquad \vhfi = \frac{1}{\kappa_\mathrm{m}} \of{\vbfi - \lambda_\mathrm{m} \vmfi}\]
Choosing \(\kappa_\mathrm{e} = \kappa_\mathrm{m} = 1\) gives us the non-rescaled
fields, while we get the rescaled fields with \(\kappa_\mathrm{e} = \lambda_\mathrm{e}\)
and \(\kappa_\mathrm{m} = \lambda_\mathrm{m}\).
We once again remark that, while the definition of \(\lambda_\mathrm{e}\) and
\(\lambda_\mathrm{m}\) is simply a matter of convenience, and they depend solely
upon previously defined constants, the choice of \(\kappa_\mathrm{e}\) and \(\kappa_\mathrm{m}\)
is entirely independent of our previous conventions.\\[1em]
With this new convention Maxwell's equations take the following form.
\begin{center}
  \begin{tabular}{cccl}
    \(\mathrm{M}_\mathrm{I}\) & \(\dvg{\vdfi}\) & \(=\) & \(\dfrac{\alphaG}{\kappa_\mathrm{e}} \rho_\mathrm{f}\) \\[1em]
    \(\mathrm{M}_\mathrm{II}\) & \(\dvg{\vbfi}\) & \(=\) & \(0\) \\[1em]
    \(\mathrm{M}_\mathrm{III}\) & \(\crl{\vefi}\) & \(=\) & \(\displaystyle - \frac{1}{\gammaF} \pderiv{\vbfi}{t}\) \\[1em]
    \(\mathrm{M}_\mathrm{IV}\) & \(\crl{\vhfi}\) & \(=\) & \(\displaystyle \frac{\alphaA}{\kappa_\mathrm{m}} \vec{j}_\mathrm{f} + \frac{\kappa_\mathrm{e}}{\kappa_\mathrm{m}} \frac{1}{\gammaM} \pderiv{\vdfi}{t}\) \\
  \end{tabular}
\end{center}
%
%
\section{Response functions}
%
We have mentioned that the polarisation and magnetisation vectors are induced by
the presence of, respectively, an electric or magnetic field.
We consider, in more general terms, how to write an induced response \(\mathcal{Y}\of{\vec{r},t}\)
in terms of a stimulus \(\mathcal{X}\of{\vec{r},t}\); both fields can be either scalar,
vectors or tensors of higher orders.
The most general case is given by the \textbf{Volterra series}, giving a nonlinear
response that depends upon the value of the stimulus at all points in space and
at all instants in time.
\[\mathcal{Y}\of{\vec{r},t} = \sum_{n=0}^\infty \int\!\!\!\dots\!\!\!\int \varrho_n\of{\vec{r},\brof{\vec{\xi}_m},t,\brof{\tau_m}} \prod_{j=1}^n \mathcal{X}\of{\vec{\xi}_j,\tau_j} \de^3\xi_j\de\tau_j\]
The functions \(\varrho_n\) are the \textbf{Volterra kernels}.
In this formula and the following, we shall use the convention that the index \(m\)
varies between \(1\) and \(n\).
To clarify what this means we write the arguments of the first three kernels explicitly.
\begin{align*}
  \varrho_0 &= \varrho_0\of{\vec{r},t} \\
  \varrho_1 &= \varrho_1\of{\vec{r},\vec{\xi}_1,t,\tau_1} \\
  \varrho_2 &= \varrho_2\of{\vec{r},\vec{\xi}_1,\vec{\xi}_2,t,\tau_1,\tau_2}
\end{align*}
An important assumption in physics is that physical laws must be time-invariant,
i.e. there is no such thing as \emph{absolute time}.
Therefore, the kernels can only depend on differences between their
temporal arguments, and not just the argument itself.
\[\mathcal{Y}\of{\vec{r},t} = \sum_{n=0}^\infty \int\!\!\!\dots\!\!\!\int \varrho_n\of{\vec{r},\brof{\vec{\xi}_m},\brof{t-\tau_m}} \prod_{j=1}^n \mathcal{X}\of{\vec{\xi}_j,\tau_j} \de^3\xi_j\de\tau_j\]
In addition, physical systems must obey causality; this further restricts the
choice of kernels to functions that vanish if, for any of their arguments, \(\tau_m > t\)
so that future values of the stimulus cannot influence the response.
This can be enforced through the use of the Heaviside step function \(\Theta\).
\[\varrho_n\of{\vec{r},\brof{\vec{\xi}_m},\brof{t-\tau_m}} = \varrho_n\of{\vec{r},\brof{\vec{\xi}_m},\brof{t-\tau_m}} \prod_{j=1}^n \Theta\of{t-\tau_j}\]
The spatial part of the function cannot in general be simplified.
Some physical systems, for example crystalline materials, are only invariant
under finite translations.
While these symmetries do influence the functional form of the kernel,
they don't do so in a general enough manner to be treated here, but they do prevent
us from having a function that only depends on distances, rather than absolute positions.\\[1em]
In systems that are symmetric under infinitesimal translations, such as fluids
or amorphous materials, we can write the Volterra series in the \textbf{semilocal approximation}.
\[\mathcal{Y}\of{\vec{r},t} = \sum_{n=0}^\infty \int\!\!\!\dots\!\!\!\int \varrho_n\of{\brof{\vec{r}-\vec{\xi}_m},\brof{t-\tau_m}} \prod_{j=1}^n \mathcal{X}\of{\vec{\xi}_j,\tau_j} \de^3\xi_j\de\tau_j\]
An even stronger approximation is the \textbf{local approximation}, where the
response is assumed to only be due to the value of the stimulus at the same
point in space.
That is to say, the spatial part of the kernel is a Dirac delta.
\[\varrho_n\of{\brof{\vec{r}-\vec{\xi}_m},\brof{t-\tau_m}} = \varrho_n\of{\brof{t-\tau_m}} \prod_{j=1}^n \delta\of{\vec{r}-\vec{\xi}_j}\]
In terms of the Volterra series, this gives us the following expression.
\[\mathcal{Y}\of{\vec{r},t} = \sum_{n=0}^\infty \int\!\!\!\dots\!\!\!\int \varrho_n\of{\brof{t-\tau_m}} \prod_{j=1}^n \mathcal{X}\of{\vec{r},\tau_j} \de\tau_j\]
Finally, we have the case of a \textbf{non-dispersive medium}, where both the
spatial and temporal part of the kernel are Dirac deltas.
\[\varrho_n\of{\brof{\vec{r}-\vec{\xi}_m},\brof{t-\tau_m}} = \varrho_n \prod_{j=1}^n \delta\of{\vec{r}-\vec{\xi}_j}\delta\of{t-\tau_j}\]
In this last case, the Volterra series simply becomes a Taylor series.
\[\mathcal{Y}\of{\vec{r},t} = \sum_{n=0}^\infty \varrho_n \, \mathcal{X}^n\of{\vec{r},t}\]
%
%
\subsection{Response functions for electromagnetic fields}
%
This allows us to define the \textbf{susceptibility} \(\chi\) of the material.
Depending on which field we choose to work with, we obtain two different definitions
of susceptibility, we denote the susceptibility to the physical field as \(\chi'\) and
the susceptibility to the auxiliary field as \(\chi''\).
\begin{equation}
  \vpfi = \frac{\chi_\mathrm{e}'}{\kappa_\mathrm{e}} \vefi = \chi_\mathrm{e}'' \vdfi
  \qquad
  \vmfi = \frac{\chi_\mathrm{m}'}{\kappa_\mathrm{m}} \vbfi = \chi_\mathrm{m}'' \vhfi
\end{equation}
While it might be tempting to say that the two definitions provide the same value,
since in a vacuum we have that \(\vefi = \kappa_\mathrm{e} \, \vdfi\) and
\(\vbfi = \kappa_\mathrm{m} \, \vhfi\), that is only true in a vacuum where, in
any case, the susceptibilities vanish.\\[1em]
We now substitute these relations into the definitions of the auxiliary fields in
order to find a response function binding the physical to the auxiliary field.
We start by using the susceptibility to the physical fields.
\[\vdfi = \frac{1}{\kappa_\mathrm{e}} \of{\vefi + \lambda_\mathrm{e} \vpfi}
  = \frac{1}{\kappa_\mathrm{e}} \of{\vefi + \frac{\lambda_\mathrm{e}}{\kappa_\mathrm{e}} \chi_\mathrm{e}' \vefi}
  = \frac{1}{\kappa_\mathrm{e}} \of{1 + \frac{\lambda_\mathrm{e}}{\kappa_\mathrm{e}} \chi_\mathrm{e}'} \vefi
  = \frac{\epsilon_\mathrm{r}}{\kappa_\mathrm{e}} \, \vefi = \epsilon \, \vefi
\]
\[\vhfi = \frac{1}{\kappa_\mathrm{m}} \of{\vbfi + \lambda_\mathrm{m} \vmfi}
  = \frac{1}{\kappa_\mathrm{m}} \of{\vbfi + \frac{\lambda_\mathrm{m}}{\kappa_\mathrm{m}} \chi_\mathrm{m}' \vbfi}
  = \frac{1}{\kappa_\mathrm{m}} \of{1 + \frac{\lambda_\mathrm{m}}{\kappa_\mathrm{m}} \chi_\mathrm{m}'} \vbfi
  = \frac{\nu_\mathrm{r}}{\kappa_\mathrm{m}} \, \vbfi = \nu \, \vbfi
\]
Unfortunately \(\epsilon\) usually named \textbf{permittivity} despite being
a measure of how strongly the material is able to resist the electric field by
polarizing. In order to get around this contradiction, we shall be naming this
quantity \textbf{austerity}. On the other hand \(\nu\) is more aptly named \textbf{reluctivity}.\\[1em]
We now switch to using the susceptibility to the auxiliary fields.
\[\vefi = \kappa_\mathrm{e} \vdfi - \lambda_\mathrm{e} \vpfi
  = \kappa_\mathrm{e} \vdfi - \lambda_\mathrm{e} \chi_\mathrm{e}'' \vdfi
  = \kappa_\mathrm{e} \of{1 - \frac{\lambda_\mathrm{e}}{\kappa_\mathrm{e}} \chi_\mathrm{e}''} \vdfi
  = \kappa_\mathrm{e} \, \eta_\mathrm{r} \vdfi = \eta \, \vdfi
\]
\[\vbfi = \kappa_\mathrm{m} \vhfi - \lambda_\mathrm{m} \vmfi
  = \kappa_\mathrm{m} \vhfi - \lambda_\mathrm{m} \chi_\mathrm{m}'' \vhfi
  = \kappa_\mathrm{m} \of{1 - \frac{\lambda_\mathrm{m}}{\kappa_\mathrm{m}} \chi_\mathrm{m}''} \vhfi
  = \kappa_\mathrm{m} \, \mu_\mathrm{r} \vhfi = \mu \, \vhfi
\]
The quantity \(\eta\) is rarely used and does not have a name, we shall refer to
it as \textbf{leniency}. The much more commonly used \(\mu\) is named \textbf{permeability}.\\[1em]
All of the quantities we have defined have both an \textbf{absolute} value (\(\epsilon\), \(\nu\),
\(\eta\), \(\mu\)) and a \textbf{relative} (\(\epsilon_\mathrm{r}\), \(\nu_\mathrm{r}\), \(\eta_\mathrm{r}\), \(\mu_\mathrm{r}\))
value, although if we are working with the non-rescaled fields the two values are the same
since \(\kappa_\mathrm{e} = \kappa_\mathrm{m} = 1\).\\[1em]
We can prove that \(\epsilon\) and \(\eta\) are mutually inverse and the same is
true for \(\nu\) and \(\mu\).
\[\vefi = \eta \, \vdfi = \eta \epsilon \, \vefi \qquad \eta \epsilon = 1\]
\[\vbfi = \mu \, \vhfi = \mu \nu \, \vbfi \qquad \mu \nu = 1\]
The same is true for the relative values. By writing this explicitly, for example
in the case of \(\epsilon\) and \(\eta\) we can gain further insight in the matter.
\[\eta = \frac{1}{\epsilon} \qquad\Rightarrow\qquad 1 - \frac{\lambda_\mathrm{e}}{\kappa_\mathrm{e}} \chi_\mathrm{e}'' = \frac{1}{1 + \frac{\lambda_\mathrm{e}}{\kappa_\mathrm{e}} \chi_\mathrm{e}'}
  = 1 - \frac{\lambda_\mathrm{e}}{\kappa_\mathrm{e}} \chi_\mathrm{e}' + \omicron\of{\chi_\mathrm{e}'}\]
This shows that, for small susceptibilities, \(\chi_\mathrm{e}' \approx \chi_\mathrm{e}''\).\\[1em]
All the previous considerations can be generalised to dispersive media if they are
written in terms of Fourier components, rather than treating them as constants (at
least up to the semi-local approximation). In the case of anisotropic media the
response functions become tensors.\\[1em]
Due to a long-standing debate on whether the \(\bfi\)-field or the \(\hfi\)-field
is the most fundamental, possibly spurred by the fact that magnetic phenomena are often
weak in matter, there is an unfortunate difference in which quantity is most commonly
used as a response function. While for the electric field the choice falls on
\(\epsilon\) (for some reason there has never been any doubt that \(\vefi\) is the
most fundamental), called the \textbf{dielectric constant}, for the magnetic field
it is customary to use \(\mu\), called the \textbf{diamagnetic constant}.
The reason for this is that it is much easier to measure the \(\hfi\)-field, leading
to a preference for this field in the engineering community.
In physical terms, however, the \(\bfi\)-field is clearly the most fundamental
for several reasons, including the fact that it appears in the Lorentz force and
that its source term is the total current, rather than just the free current.
%
%
\section{Systems}
%
A system of units is said to be \textbf{rationalised} when no factors of \(4 \pi\)
appear in Maxwell's equations.
This is useful because the absence of such factors implies that equations relating
to specific charge distributions will end up having factors related to the symmetry
of the system, such as \(4 \pi\) for spherical symmetry or \(2 \pi\) for cylindrical
symmetry.
\subsection{The International System -- SI}
The International System of units makes the following choice for the couplings.
\[\alphaG = \frac{1}{\epsilon_0} \qquad \alphaA = \mu_0 \qquad \kappac = 1 \qquad \kappaf = 1 \qquad \gammaM = c^2 \qquad \gammaF = 1\]
Here \(\epsilon_0\) and \(\mu_0\) are constants determined by experiment; the current (2020) CODATA recommended values are
\[\epsilon_0 = \SI{8.8541878128(13)e-12}{\farad\per\meter} \qquad \mu_0 = \SI{1.25663706212(19)e-6}{\henry\per\meter}\]
The SI is now a truly rationalised system. Before its redefinition in 2019 it used to
be an unrationalised system in disguise, because \(\mu_0\) was defined as
exactly \(4\pi \cdot \SI{e-7}{\henry\per\meter}\).
We can make this explicit by defining a scaling constant
\(a = \SI{e7}{\ampere\squared\per\newton}\).
\[\alphaG = \frac{4 \pi c^2}{a} \qquad \alphaA = \frac{4 \pi}{a} \qquad \kappac = 1 \qquad \kappaf = 1 \qquad \gammaM = c^2 \qquad \gammaF = 1\]
%
\subsection{Electrostatic units -- ESU}
Electrostatic units are defined by taking Coulomb's constant to be unity.
\[\alphaG = 4 \pi \qquad \alphaA = \frac{4\pi}{c^2} \qquad \kappac = 1 \qquad \kappaf = 1 \qquad \gammaM = c^2 \qquad \gammaF = 1\]
This system is explicitly unrationalised.
%
\subsection{Electromagnetic units -- EMU}
Electromagnetic units are defined in order to make Ampère's force constant unity.
The name electromagnetic is not very apt, since Ampère's force law is only valid
in the magnetostatic limit.
\[\alphaG = 4 \pi c^2 \qquad \alphaA = 4\pi \qquad \kappac = 1 \qquad \kappaf = 1 \qquad \gammaM = c^2 \qquad \gammaF = 1\]
This system is explicitly unrationalised, but notice how it is the same as the old SI
units, except for the lack of the scaling factor \(a\).
The reason for this is that the SI adopted a rescaled version of EMU that were in
practical use among engineers.
%
\subsection{Gaussian units -- GU}
Gaussian units are defined as follows.
\[\alphaG = 4 \pi \qquad \alphaA = \frac{4\pi}{c} \qquad \kappac = 1 \qquad \kappaf = c \qquad \gammaM = c \qquad \gammaF = c\]
This system is explicitly unrationalised.
%
\subsection{Lorentz-Heaviside units -- LHU}
Lorentz-Heaviside units are obtained by rationalising Gaussian units.
\[\alphaG = 1 \qquad \alphaA = \frac{1}{c} \qquad \kappac = 1 \qquad \kappaf = c \qquad \gammaM = c \qquad \gammaF = c\]
Obviously, this system is rationalised.
%
\subsection{Symetrised Gaussian units -- SGU}
Gaussian units can be made fully symmetrical, with \(\alphaG = \alphaA = 4 \pi\),
by taking \(\kappac = c\). This also has the advantage of making current
density and charge density have the same units, like the electric and magnetic
field in the standard Gaussian system.
\[\alphaG = 4 \pi \qquad \alphaA = 4\pi \qquad \kappac = c \qquad \kappaf = c \qquad \gammaM = c \qquad \gammaF = c\]
This system is explicitly unrationalised.
%
\subsection{Symmetrised Lorentz-Heaviside units -- SLHU}
Lorentz-Heaviside units can also be made fully symmetrical, with \(\alphaG = \alphaA = 1\).
\[\alphaG = 1 \qquad \alphaA = 1 \qquad \kappac = c \qquad \kappaf = c \qquad \gammaM = c \qquad \gammaF = c\]
This system is rationalised.
%
%
\section{Magnetic monopoles}
Looking at Maxwell's equations, there appears to be a fundamental lack of symmetry
between the electric and magnetic fields.
This lack of symmetry is due to the fact that, as of now, no experiment has ever
observed particles which carry such a thing as \textbf{magnetic charge}, that is
to say some particle which is capable of generating a magnetic field while standing still.
All magnetic phenomena are related to the motion of \emph{electric} charges, whether
this motion is macroscopic (as is the case with current-carrying wires) or microscopic
(as is the case with bar magnets).\\[1em]
Despite the absence of evidence, there have been theoretical arguments for the existence of such particles,
and a correction to Maxwell's equations is in order to account for their presence.
From here on we denote electric charges and currents  with the subscript \(\mathrm{e}\)
and magnetic charges and currents with the subscript \(\mathrm{m}\).\\[1em]
After introducing the concept of magnetic charge \(Q_\mathrm{m}\), the concept of
magnetic current quickly follows. While we could, in principle, take a different
proportionality constant between the current and the time derivative of charge,
this would not be particularly useful, we thus use \(\kappac\) as we
did for electric charge.
\[I_\mathrm{m} = - \frac{1}{\kappac} \pderiv{Q_\mathrm{m}}{t}\]
Naturally, we can define the corresponding densities.
\[Q_\mathrm{m} = \iiint_\Omega \rho_\mathrm{m}\of{\vec{\xi},t} \de^3\xi
\qquad I_\mathrm{m} = \iint_\Sigma \vec{j}_\mathrm{m}\of{\vec{\xi},t} \cdot \vec{n} \de^2\xi\]
The densities must satisfy a continuity equation, the proof of which is identical
to the one given for the electric case, and shall not be repeated.
\[\frac{1}{\kappac} \pderiv{\rho_\mathrm{m}}{t} + \dvg{\vec{j}_\mathrm{m}} = 0\]
Due to the existence of a continuity equation, we also have the following relation
with the velocity field, once again the proof is identical to the previous.
\[\vec{j}_\mathrm{m}\of{\vec{r},t} = \frac{\vec{v}\of{\vec{r},t} \, \rho_\mathrm{m}\of{\vec{r},t}}{\kappac}\]
To correct Maxwell's equations we introduce two new coupling constants \(\beta_\mathrm{G}\)
and \(\beta_\mathrm{A}\) for Gauss's law and Ampère--Maxwell's law, respectively.
\begin{center}
  \begin{tabular}{ccccl}
    \(\mathrm{M}_\mathrm{I}\) & \textbf{Gauss's law (for \(\vefi\))} & \(\dvg{\vefi}\) & \(=\) & \(\alphaG \, \rho_\mathrm{e}\) \\[1em]
    \(\mathrm{M}_\mathrm{II}\) & \textbf{Gauss's law (for \(\vbfi\))} & \(\dvg{\vbfi}\) & \(=\) & \(\beta_\mathrm{G} \, \rho_\mathrm{m}\) \\[1em]
    \(\mathrm{M}_\mathrm{III}\) & \textbf{Faraday--Maxwell's law} & \(\crl{\vefi}\) & \(=\) & \(\displaystyle - \beta_\mathrm{A} \, \vec{j}_\mathrm{m} - \frac{1}{\gammaF} \pderiv{\vbfi}{t}\) \\[1em]
    \(\mathrm{M}_\mathrm{IV}\) & \textbf{Ampère--Maxwell's law} & \(\crl{\vbfi}\) & \(=\) & \(\displaystyle \alphaA \, \vec{j}_\mathrm{e} + \frac{1}{\gammaM} \pderiv{\vefi}{t}\) \\
  \end{tabular}
\end{center}
We must now find the correct expression for the Lorentz force of a magnetically
charged particle. To do so we attempt to find an analogue of the Faraday--Neumann--Lenz
law that relates the flux of the electric field to the work done per unit charge.
As before, we consider a closed loop delimiting a surface \(\Sigma\) and we take
the time derivative of the flux of the electric field through this surface.
\[\D{\Phi_\vefi}{t} = \D{}{t} \iint_{\Sigma\of{t}} \vefi\of{\vec{\xi},t} \cdot \vec{n}\of{\vec{\xi},t} \de^2\xi\]
We apply the Reynolds transport theorem in order to compute this derivative.
\[\D{\Phi_\vefi}{t} = \iint_{\Sigma\of{t}} \of{\pderiv{\vefi}{t} + \of{\dvg{\vefi}} \vec{v}} \cdot \vec{n} \, \de^2\xi
- \oint_{\partial\Sigma(t)} \of{\vec{v} \times \vefi} \cdot \vec{t}\,  \de\xi\]
This time the field is not solenoidal, using Gauss's law for the electric field and the Ampère--Maxwell law, we get the
following expression.
\[\D{\Phi_\vefi}{t} = \iint_{\Sigma\of{t}} \of{\gammaM \crl{\vbfi} - \alphaA \gammaM \vec{j}_\mathrm{e} +
\alphaG \, \rho_\mathrm{e} \, \vec{v}} \cdot \vec{n} \, \de^2\xi
- \oint_{\partial\Sigma(t)} \of{\vec{v} \times \vefi} \cdot \vec{t}\,  \de\xi\]
\[\D{\Phi_\vefi}{t} = \iint_{\Sigma\of{t}} \of{- \alphaA \gammaM \vec{j}_\mathrm{e} +
\alphaG \, \rho_\mathrm{e} \, \vec{v}} \cdot \vec{n} \, \de^2\xi
+ \oint_{\partial\Sigma(t)} \of{\gammaM \, \vbfi - \vec{v} \times \vefi} \cdot \vec{t}\,  \de\xi\]
We recall that \(\alphaA \gammaM = \alphaG \kappac\),
giving us the following expression.
\[\D{\Phi_\vefi}{t} = \alphaG \iint_{\Sigma\of{t}} \of{ \rho_\mathrm{e} \, \vec{v} - \kappac \, \vec{j}_\mathrm{e}} \cdot \vec{n} \, \de^2\xi + \gammaM \oint_{\partial\Sigma(t)} \of{ \vbfi - \frac{\vec{v} \times \vefi}{\gammaM}} \cdot \vec{t} \, \de\xi\]
Then we notice that the first integral vanishes, because \(\rho_\mathrm{e} \, \vec{v} = \kappac \, \vec{j}_\mathrm{e}\).
\[\D{\Phi_\vefi}{t} = \gammaM \oint_{\partial\Sigma(t)} \of{ \vbfi - \frac{\vec{v} \times \vefi}{\gammaM}} \cdot \vec{t} \, \de\xi\]
We have managed to find an expression similar to the Faraday--Neumann--Lenz law,
but some care must be taken in naming the right hand side of this equation. We shall
call this \textbf{magnetomotive force} \(\emf_\mathrm{m}\), but we remark that this term is already
in use for a distinct concept in magnetic circuits.
\begin{equation}
  \D{\Phi_\vefi}{t} = \gammaM \, \emf_\mathrm{m}
\end{equation}
We notice there is a difference in sign with the Faraday--Neumann--Lenz law
for the electromotive force, this is in line with the difference in sign between the
Faraday--Maxwell and the Ampère--Maxwell laws. By analogy with the electromotive force,
we can reasonably expect the Lorentz force for a moving magnetically charge particle to have the following expression.
\begin{equation}
  \vec{F}\of{\vec{r},t} = q_\mathrm{m} \of{\vbfi\of{\vec{r},t} - \frac{\vec{v} \times \vefi\of{\vec{r},t}}{\gammaM}}
\end{equation}
If a particle has both an electric and magnetic charge, then the Lorentz force must
include both contributions.
\begin{equation}
  \vec{F} = q_\mathrm{e} \of{\vefi + \frac{\vec{v} \times \vbfi}{\gammaF}} + q_\mathrm{m} \of{\vbfi - \frac{\vec{v} \times \vefi}{\gammaM}}
\end{equation}

\newpage
\section{Other laws}
\textbf{Coulomb's law}
\[\vec{F}_{12} = k_\mathrm{C} \iiint_{\Omega_1} \iiint_{\Omega_2} \frac{\rho_1\!\!\of{\vec{\xi_1}} \rho_2\!\!\of{\vec{\xi_2}}}{\nrm{\vec{r}_1 - \vec{r}_2}^3} \of{\vec{r}_1 - \vec{r}_2} \de^3\xi_1 \de^3\xi_2\]
\[\vefi(\vec{r}) = k_\mathrm{C} \iiint_\Omega \frac{\rho\!\of{\vec{\xi}}}{\nrm{\vec{r} - \vec{\xi}}^3} \of{\vec{r} - \vec{\xi}} \de^3\xi\]
\textbf{Ampère's law}
\[\vec{F}_{12} = k_\mathrm{A} \iiint_{\Omega_1} \iiint_{\Omega_2} \frac{\vec{j}_1\!\!\of{\vec{\xi_1}} \times \of{\vec{j}_2\!\!\of{\vec{\xi_2}} \times \of{\vec{\xi}_1 - \vec{\xi}_2}}}{\nrm{\vec{\xi}_1 - \vec{\xi}_2}^3} \de^3\xi_1 \de^3\xi_2\]
\textbf{Biot-Savart's law}
\[\vbfi(\vec{r}) = k_\mathrm{S} \iiint_\Omega \frac{\vec{j}\!\of{\vec{\xi}} \times \of{\vec{r} - \vec{\xi}}}{\nrm{\vec{r} - \vec{\xi}}^3} \de^3\xi\]

\newpage
\section{Continuum mechanics}
Consider a body undergoing deformation, a reference configuration in which the
coordinates of the points are given by \(\vec{X}\) (Lagrangian or material picture) and the
current configuration in which they are given by \(\vec{x}\) (Eulerian or spatial picture).\\[1em]
A \textbf{motion} in the body is given by a function \(\vec{\chi}\) mapping the coordinates
in the reference configuration to those in the current configuration.
\[\vec{x} = \vec{\chi}\of{\vec{X},t}\]
We take this function to be invertible; we denote its inverse by \(\vec{\xi}\).
\[\vec{X} = \vec{\xi}\of{\vec{x},t}\]
We moreover take \(\vec{\chi}\) to be continuously differentiable and name its
gradient (a rank 2 tensor) the \textbf{material deformation gradient} \(\vec{F}\).
\[\vec{F} = \pderiv{\vec{x}}{\vec{X}} = \Grd{\vec{x}}\]
In the same way, we name the gradient of \(\vec{\xi}\) the \textbf{spatial
deformation gradient} \(\vec{H}\).
\[\vec{H} = \pderiv{\vec{X}}{\vec{x}} = \grd{\vec{X}}\]
These two tensors are such that the following relations on the differentials are true.
\[\de\vec{x} = \vec{F} \cdot \de\vec{X} \qquad \de\vec{X} = \vec{H} \cdot \de\vec{x}\]
The spatial deformation gradient is the inverse of the material deformation gradient,
depending on the order in which they are contracted, we obtain either the Lagrangian
or Eulerian identity tensor.
\[\vec{F} \cdot \vec{H} = \vec{I}_\mathrm{E} \qquad \vec{H} \cdot \vec{F} = \vec{I}_\mathrm{L}\]
Another useful property of the material deformation gradient is that its determinant,
usually denoted by \(\mathcal{J}\) is a measure of the change of a volume element
(trivial, since \(\vec{F}\) is the Jacobian of the coordinate change).
\[\mathcal{J} = \det \vec{F} \qquad \de^3x = \mathcal{J} \de^3X\]
A useful relation to know is the derivative of \(\mathcal{J}\) with regards to \(\vec{F}\),
which is just a restatement of the formula for the derivative of a determinant.
\[\pderiv{\mathcal{J}}{\vec{F}} = \mathcal{J} \, \tr{\vec{H}}\]
Less trivial is \textbf{Nanson's formula}, used to change the oriented area elements,
here \(\vec{n}\) is the normal vector in the current configuration and \(\vec{N}\)
is the normal vector in the reference configuration.
\[\vec{n} \, \de^2x = \mathcal{J} \tr{\vec{H}} \cdot \vec{N} \, \de^2X\]
We now take the time derivative of the material deformation gradient.
\[\dot{\vec{F}} = \D{}{t} \pderiv{\vec{x}}{\vec{X}} = \pderiv{}{\vec{X}} \D{\vec{x}}{t} = \pderiv{\vec{V}}{\vec{X}} = \Grd{\vec{V}}\]
We have introduced the \textbf{velocity} \(\vec{V}\of{\vec{X},t}\) as the time derivative of the
position in the current configuration.
The time derivative of the material deformation gradient is thus the material velocity gradient.
It is often more useful to express things in terms of a spatial gradient,
in which case we denote the velocity by \(\vec{v}\of{\vec{x},t}\) because we consider it in
the Eulerian picture, rather than in the Lagrangian.
\[\dot{\vec{F}} = \pderiv{\vec{v}}{\vec{x}} \cdot \pderiv{\vec{x}}{\vec{X}} = \vec{L} \cdot \vec{F}\]
This led us to defining the \textbf{spatial velocity gradient} \(\vec{L}\).
\[\vec{L} = \pderiv{\vec{v}}{\vec{x}} = \grd{\vec{v}}\]
We can use this result to compute the time derivative of the spatial deformation gradient, we
do this by invoking the formula for the derivative of matrix inverse.
\[\dot{\vec{H}} = - \vec{H} \cdot \dot{\vec{F}} \cdot \vec{H} = - \vec{H} \cdot \vec{L} \cdot \vec{F} \cdot \vec{H} = - \vec{H} \cdot \vec{L}\]
And we can compute the time derivative of the determinant \(\mathcal{J}\) as well.
\[\dot{\mathcal{J}} = \pderiv{\mathcal{J}}{\vec{F}} : \dot{\vec{F}} = \mathcal{J} \tr{\vec{H}} : \vec{L} \cdot \vec{F} = \mathcal{J} \trace{\vec{L}} = \mathcal{J} \, \dvg{\vec{v}}\]
Where we have used the property that \(\vec{A}:\vec{B} = \trace\of{\tr{\vec{A}}\cdot\vec{B}}\)
and the fact that the trace is conserved under transformations, so \(\trace\of{\vec{H}\cdot\vec{L}\cdot{\vec{F}}} = \trace{\vec{L}}\).
%
%
\section{Proof of the Reynolds Transport Theorem}
%
Consider the flux of some vector field \(\vec{a}\of{\vec{x},t}\) through a surface
\(\Sigma\of{t}\), and take its time derivative, which will depend both on the intrinsic
change in the vector field, and in the change of shape of the surface.
\[\D{\Phi_\vec{a}}{t} = \D{}{t} \iint_{\Sigma\of{t}} \vec{a}\of{\vec{x},t} \cdot \vec{n}\of{\vec{x},t} \de^2x\]
We use Nanson's formula to bring the integral to the reference configuration, where
the domain of intgration is independent of time, we denote by \(\vec{A}\) the material
representation of the vector field: \(\vec{A}\of{\vec{X},t} = \vec{a}\of{\vec{x}\of{\vec{X},t},t}\).
\[\D{\Phi_\vec{a}}{t} = \D{}{t} \iint_{\Sigma_0} \vec{A} \cdot \mathcal{J} \tr{\vec{H}} \cdot \vec{N}\,\de^2X\]
We can now bring the derivative inside the integral, giving us the following.
\[\D{\Phi_\vec{a}}{t} = \iint_{\Sigma_0} \of{\dot{\vec{A}} \cdot \mathcal{J} \tr{\vec{H}} +
    \vec{A} \cdot \dot{\mathcal{J}} \tr{\vec{H}} +
    \vec{A} \cdot \mathcal{J} \tr{\dot{\vec{H}}}} \cdot \vec{N}\,\de^2X
\]
Substituting the relevant quantities we find the following.
\begin{align*}
    \D{\Phi_\vec{a}}{t} &= \iint_{\Sigma_0} \of{\dot{\vec{A}} \cdot \mathcal{J} \tr{\vec{H}} +
    \vec{A} \cdot \of{\mathcal{J} \trace \vec{L}} \tr{\vec{H}} -
    \vec{A} \cdot \mathcal{J} \of{\tr{\vec{L}} \cdot \tr{\vec{H}}}} \cdot \vec{N}\,\de^2X \\
    &= \iint_{\Sigma_0} \of{\dot{\vec{A}} + \vec{A} \trace \vec{L} - \vec{A} \cdot \tr{\vec{L}}} \cdot \mathcal{J} \tr{\vec{H}} \cdot \vec{N}\,\de^2X
\end{align*}
We can now apply Nanson's formula backwards to bring the integral back to the
current configuration.
\[\D{\Phi_\vec{a}}{t}= \iint_{\Sigma\of{t}} \of{\dot{\vec{a}} + \vec{a} \trace \vec{L} - \vec{a} \cdot \tr{\vec{L}}} \cdot \vec{n}\,\de^2x\]
We expand \(\dot{\vec{a}}\) as the material derivative and replace \(\vec{L}\) with the
relevant expressions in terms of \(\vec{v}\).
\[\D{\Phi_\vec{a}}{t}= \iint_{\Sigma\of{t}} \of{\pderiv{\vec{a}}{t} + \vec{v} \cdot \grd{\vec{a}} + \vec{a} \of{\dvg{\vec{v}}} - \vec{a} \cdot \grd{\vec{v}}} \cdot \vec{n}\,\de^2x\]
We now recall the following vector identity for the curl of a cross product.
\[\crl{\of{\vec{a}\times\vec{v}}} = \vec{a} \of{\dvg{\vec{v}}} - \vec{v} \of{\dvg{\vec{a}}} + \vec{v} \cdot \grd{\vec{a}} - \vec{a} \cdot \grd{\vec{v}}\]
We notice that most of the terms appear in our expression, we thus obtain one of
the many possible expressions of the transport theorem for the flux of a vector field.
\[\D{\Phi_\vec{a}}{t}= \iint_{\Sigma\of{t}} \of{\pderiv{\vec{a}}{t} + \vec{v} \of{\dvg{\vec{a}}} + \crl{\of{\vec{a}\times\vec{v}}}} \cdot \vec{n}\,\de^2x\]
Using the curl theorem we can also bring the last term in terms of a circulation.
\[\D{\Phi_\vec{a}}{t}= \iint_{\Sigma\of{t}} \of{\pderiv{\vec{a}}{t} + \vec{v} \of{\dvg{\vec{a}}}} \cdot \vec{n}\,\de^2x
  + \oint_{\partial\Sigma\of{t}} \of{\vec{a}\times\vec{v}} \cdot \vec{t} \, \de x\]
This form is especially useful for solenoidal fields, where \(\dvg{\vec{a}}\) disappears.
\[\D{\Phi_\vec{a}}{t}= \iint_{\Sigma\of{t}} \pderiv{\vec{a}}{t} \cdot \vec{n}\,\de^2x
  + \oint_{\partial\Sigma\of{t}} \of{\vec{a}\times\vec{v}} \cdot \vec{t} \, \de x\]

\end{document}
