Consider a small point charge \(q\), which we call a \textbf{test charge}
standing still in space, we define the \textbf{electric field} \(\vefi\) as the
ratio between the force measured and the charge, that is to say it is the force
per unit charge.
\[\vec{F}\of{\vec{r},t} = q \vefi\of{\vec{r},t}\]
Consider now a thin stationary wire tracing a curve \(\gamma\) in a region of space devoid
of electric fields. Along this wire travels a small current \(i\). We can't simply
define the \textbf{magnetic field} \(\vbfi\) as force per unit current, because
it is found experimentally that it depends upon the direction the current is
travelling in. In mathematical terms the force acting on the wire is both a functional
depending on the curve \(\gamma\) and a function of time. For a given parametrization
\(\xi\) of the curve \(\gamma\) we get the following expression, sometimes known as the \textbf{Laplace force}.
\[\vec{F}[\gamma]\of{t} = i \int_\gamma \vec{t}\of{\xi} \times \vbfi\of{\vec{r},t} \, \de\xi\]
We can also view this in terms of local quantities, that is to say charge and
current density.
The conversion is trivial for the electric force; if instead of a point charge
we have a distribution of charge following a certain density \(\rho\), the total
force on the distribution is given by the following expression.
\[\vec{F}(t) = \iiint_\Omega \rho(\vec{\xi},t) \vefi\of{\vec{\xi},t} \de^3\xi\]

The derivatives of the delta function are the following.
\begin{align*}
  \pderiv{}{x} \delta\of{\vec{r}-\vec{v}t} &= \delta'\!\of{x-v_xt}\delta\!\of{y-v_yt}\delta\!\of{z-v_zt} \\
  \pderiv{}{y} \delta\of{\vec{r}-\vec{v}t} &= \delta\!\of{x-v_xt}\delta'\!\of{y-v_yt}\delta\!\of{z-v_zt} \\
  \pderiv{}{z} \delta\of{\vec{r}-\vec{v}t} &= \delta\!\of{x-v_xt}\delta\!\of{y-v_yt}\delta'\!\of{z-v_zt} \\
  \pderiv{}{t} \delta\of{\vec{r}-\vec{v}t} &= - v_x \, \delta'\!\of{x-v_xt}\delta\!\of{y-v_yt}\delta\!\of{z-v_zt} \\
                                              &\quad - v_y \, \delta\!\of{x-v_xt}\delta'\!\of{y-v_yt}\delta\!\of{z-v_zt} \\
                                              &\quad - v_z \, \delta\!\of{x-v_xt}\delta\!\of{y-v_yt}\delta'\!\of{z-v_zt}
\end{align*}
And the derivatives of \(\vec{j}\) and \(\rho\) are quickly obtained from them.
\begin{align*}
  \pderiv{j_x}{x} &= \frac{q v_x}{\kappa} \delta'\!\of{x-v_xt}\delta\!\of{y-v_yt}\delta\!\of{z-v_zt} \\
  \pderiv{j_y}{y} &= \frac{q v_y}{\kappa} \delta\!\of{x-v_xt}\delta'\!\of{y-v_yt}\delta\!\of{z-v_zt} \\
  \pderiv{j_z}{z} &= \frac{q v_z}{\kappa} \delta\!\of{x-v_xt}\delta\!\of{y-v_yt}\delta'\!\of{z-v_zt} \\
  \pderiv{\rho}{t} &= - q v_x \, \delta'\!\of{x-v_xt}\delta\!\of{y-v_yt}\delta\!\of{z-v_zt} \\
                                              &\quad - q v_y \, \delta\!\of{x-v_xt}\delta'\!\of{y-v_yt}\delta\!\of{z-v_zt} \\
                                              &\quad - q v_z \, \delta\!\of{x-v_xt}\delta\!\of{y-v_yt}\delta'\!\of{z-v_zt}
\end{align*}



We can rewrite the continuity equation to have a more relativistic appearance.
\[\pderiv{}{\of{ct}} \frac{c\rho}{\kappa} + \dvg{\vec{j}} = 0\]
Then we make it manifestly covariant by defining the \textbf{4-current} \(j^\alpha\).
\[j^\alpha = \begin{pmatrix}\dfrac{c\rho}{\kappa} \\[1em] \vec{j}\end{pmatrix}\]
In terms of the 4-current, the continuity equation is written as follows.
\[\partial_\alpha j^\alpha = 0\]


For indices, we convene that Greek letters span both the temporal and spatial components of
the tensors (0 to 3), while Latin letters only span the spatial compoments (1 to 3);
we shall also be using Einstein's summation convention.
We recall the distinction between \emph{covariant} vectors, represented as rows
or with lower indices, that transform with the same matrix that transforms the basis
and \emph{contravariant} vectors, represented as columns or with upper indices,
that transform with the inverse of the matrix that transform the basis.
Switching between covariant and contravariant representation can be done by \\[1em]

= \begin{pmatrix} 0 & \dfrac{\kappa}{c} \efi_x & \dfrac{\kappa}{c} \efi_y & \dfrac{\kappa}{c} \efi_z \\[1em]
-\dfrac{\kappa}{c} \efi_x & 0 & -\bfi_z & \bfi_y \\[1em]
-\dfrac{\kappa}{c} \efi_y & \bfi_z & 0 & -\bfi_x \\[1em]
-\dfrac{\kappa}{c} \efi_z & -\bfi_y & -\bfi_x & 0 \\
\end{pmatrix}

We verify that this is consistent with the continuity equations by computing the
divergence of \(\vec{j}\) and the time derivative of \(\rho\).
In order to do this we recall that the Dirac delta of a vector argument can be
broken up into the product of Dirac deltas in the components.
This is necessary to compute the derivatives.
\[\delta\of{\vec{r}-\vec{v}t} = \delta\!\of{x-v_xt}\delta\!\of{y-v_yt}\delta\!\of{z-v_zt}\]
Computing all relevant derivatives it can be seen that the continuity equation
is satisfied, meaning that our ansatz for the current density was correct.
\[\frac{1}{\kappa} \pderiv{\rho}{t} + \pderiv{j_x}{x} + \pderiv{j_y}{y} + \pderiv{j_z}{z} = 0\]


\subsection{Covariance of Maxwell's equations}
%
Since light is an electromagnetic wave, a good theory of electromagnetism must be
consistent with special relativity. This can easily be achieved by writing the
defining equation of the theory in a manner that is \emph{manifestly covariant}.
In this document we shall use the \((+,-,-,-)\) sign convention for the metric tensor.\\[1em]
We recall that the \textbf{4-gradient} is defined as follows, in covariant components.
\[\vec{\partial} = \partial_\alpha = \begin{pmatrix}\displaystyle \frac{1}{c} \pderiv{}{t} & \grd{}\end{pmatrix}\]
We define the \textbf{4-current} \(\vec{J}\) as follows, in contravariant components.
\begin{equation}
  \vec{J} = J^\alpha = \begin{pmatrix}\dfrac{c}{\kappa_\mathrm{c}}\rho \\[1em] \vec{j}\end{pmatrix}
\end{equation}
The factor \(\frac{c}{\kappa_\mathrm{c}}\) is necessary to achieve consistency between the
units of the temporal and spatial components of the vector. Looking at equation
\eqref{eq::continuity} with this definition in mind, the continuity
equation can be written in a manifestly covariant manner as follows.
\begin{equation}
  \vec{\partial} \cdot \vec{J} = \partial_\alpha J^\alpha = 0
\end{equation}
It is impossible to replace the electric and magnetic field with a 4-vector, but
they can be represented by a skew-symmetric second rank tensor, called the \textbf{Faraday tensor}.
\begin{equation}
  \vfrd = \frd^{\alpha\beta} = \begin{pmatrix} 0 & -\dfrac{\kappa_\mathrm{f}}{c} \vefi \\[1em] \dfrac{\kappa_\mathrm{f}}{c} \vefi & \hat{\vbfi} \\ \end{pmatrix}
\end{equation}
In the definition, we have used \(\hat{\vbfi}\) to denote the cross product matrix
of \(\vbfi\); this matrix has the property that, for any vector \(\vec{w}\),
\(\vec{w} \cdot \hat{\vbfi}\) = \(\vec{w} \times \vbfi\).
Once again the factor \(\frac{\kappa_\mathrm{f}}{c}\) is necessary to have consistent units
in all parts of the tensor.\\[1em]
We now assume that we can write Maxwell's equations in the following form.
\[\vec{\partial} \cdot \vfrd = \alpha_1 \vec{J} \qquad \partial_\alpha \frd^{\alpha\beta} = \alpha_1 J^\beta\]
We attempt to verify our assumption by computing \(\vec{\partial} \cdot \vfrd\).
\[
\vec{\partial} \cdot \vfrd = \begin{pmatrix}\displaystyle \frac{1}{c} \pderiv{}{t} & \grd{}\end{pmatrix} \begin{pmatrix} 0 & -\dfrac{\kappa_\mathrm{f}}{c} \vefi \\[1em] \dfrac{\kappa_\mathrm{f}}{c} \vefi & \hat{\vbfi} \\ \end{pmatrix}
 = \begin{pmatrix}\displaystyle \frac{\kappa_\mathrm{f}}{c} \dvg{\vefi} \\[1em] \displaystyle - \frac{\kappa_\mathrm{f}}{c^2} \pderiv{\vefi}{t} + \dvg{\hat{\vbfi}}\end{pmatrix}
 = \begin{pmatrix}\displaystyle \frac{\kappa_\mathrm{f}}{c} \dvg{\vefi} \\[1em] \displaystyle - \frac{\kappa_\mathrm{f}}{c^2} \pderiv{\vefi}{t} + \crl{\vbfi}\end{pmatrix}
\]
Taking \(\vec{\partial} \cdot \vfrd = \alpha_1 \vec{J}\) we obtain the following pair of equations.
\[\begin{cases}
\dfrac{\kappa_\mathrm{f}}{c} \dvg{\vefi} = \alpha_1 \dfrac{c}{\kappa_\mathrm{c}} \rho \\[1em]
- \displaystyle \frac{\kappa_\mathrm{f}}{c^2} \pderiv{\vefi}{t} + \crl{\vbfi} = \alpha_1 \vec{j}
\end{cases} \qquad \begin{cases}
\dvg{\vefi} = \dfrac{\alpha_1 \, c^2}{\kappa_\mathrm{f} \kappa_\mathrm{c}} \rho \\[1em]
\displaystyle \crl{\vbfi} = \alpha_1 \vec{j} + \frac{\kappa_\mathrm{f}}{c^2} \pderiv{\vefi}{t}
\end{cases}\]
We observe the second equation. In order for it to be Ampère-Maxwell's law
we need to take \(\alpha_1 = \alpha_\mathrm{A}\). Knowing this we also
obtain the following two conditions.
\[\alpha_\mathrm{G} = \dfrac{c^2}{\kappa_\mathrm{f} \kappa_\mathrm{c}} \alpha_\mathrm{A} \qquad \gamma_\mathrm{M} = \frac{\kappa_\mathrm{f}}{c^2} \]
We notice that our assumption only covers two of Maxwell's equations.
In order to cover the other two, we define the \emph{dual} of the Faraday tensor.
\begin{equation}
  \vfrdg = \frdg^{\alpha\beta} = \begin{pmatrix} 0 & \vbfi \\[1em] - \vbfi & \dfrac{\kappa_\mathrm{f}}{c} \hat{\vefi} \\ \end{pmatrix}
\end{equation}
This time we assume that \(\vec{\partial} \cdot \vfrdg = 0 \). We verify this as before.
\[
\vec{\partial} \cdot \vfrdg = \begin{pmatrix}\displaystyle \frac{1}{c} \pderiv{}{t} & \grd{}\end{pmatrix} \begin{pmatrix} 0 & \vbfi \\[1em] - \vbfi & \dfrac{\kappa_\mathrm{f}}{c} \hat{\vefi} \\ \end{pmatrix}
 = \begin{pmatrix}\displaystyle -\dvg{\vbfi} \\[1em] \displaystyle \frac{1}{c} \pderiv{\vbfi}{t} + \frac{\kappa_\mathrm{f}}{c} \dvg{\hat{\vefi}} \end{pmatrix}
 = \begin{pmatrix}\displaystyle -\dvg{\vbfi} \\[1em] \displaystyle \frac{1}{c} \pderiv{\vbfi}{t} + \frac{\kappa_\mathrm{f}}{c} \crl{\vefi} \end{pmatrix}
\]
The equations we get resemble Gauss's law for magnetism and Faraday--Maxwell's law.
\[\begin{cases}
-\dvg{\vbfi} = 0 \\[1em]
\displaystyle \frac{1}{c} \pderiv{\vbfi}{t} + \frac{\kappa_\mathrm{f}}{c} \crl{\vefi} = 0
\end{cases} \qquad \begin{cases}
\dvg{\vbfi} = 0 \\[1em]
\displaystyle \crl{\vefi} = - \frac{1}{\kappa_\mathrm{f}} \pderiv{\vbfi}{t}
\end{cases}\]
We obtain one final condition.
\[\gamma_\mathrm{F} = \frac{1}{\kappa_\mathrm{f}}\]

\section{Electric and magnetic fields}
We now give the definition of the electric and magnetic fields, in order to simplify
this process we consider the \textbf{force density} \(\vec{f}\), a local property
defined by the amount of force per unit volume.
The total force is therefore given by the following expression.
\[\vec{F}(t) = \iiint_\Omega \vec{f}\of{\vec{\xi},t} \de^3 \xi\]
The \textbf{electric field} \(\vefi\) is defined by the electric contribution to
the force density, that is the contribution that is independent of the motion of
the charges.
The coupling constant is simply the charge density, and the field is in the same
direction as the force.
\[\vec{f}_\mathrm{e}\of{\vec{r},t} = \rho\of{\vec{r},t}\vefi\of{\vec{r},t}\]
The \textbf{magnetic field} \(\vbfi\) provides the magnetic contribution to the
force density, that is the contribution that is dependent upon the motion of the
charges.
In this case, the force is observed to be perpendicular both to the field and to
the current density, therefore we must account for this fact in the definition.
The current density itself also plays the role of a coupling constant for this force.
\[\vec{f}_\mathrm{m}\of{\vec{r},t} = \vec{j}\of{\vec{r},t} \times \vbfi\of{\vec{r},t}\]
Since both \(\vec{f}\) and \(\vec{j}\) are vectors, it is necessary for \(\vbfi\)
to be a \emph{pseudovector}.\\[1em]
The \textbf{Lorentz force} provides a complete description of the interaction of
a charge distribution with the electromagnetic field by combining in itself both
the electric and magnetic terms that we have considered separately.
\begin{equation}
  \vec{f}\of{\vec{r},t} = \rho\of{\vec{r},t}\vefi\of{\vec{r},t} + \vec{j}\of{\vec{r},t} \times \vbfi\of{\vec{r},t}
\end{equation}
